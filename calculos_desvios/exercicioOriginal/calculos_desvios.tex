
\item Escreva um programa em \emph{Pascal} que leia um número e o imprima 
caso ele seja maior que 20.

\begin{center}
\begin{tabular}{|l|l|} \hline
Exemplo Entrada & Saída esperada \\ \hline
30.56           & 30.56          \\ \hline
20              &                \\ \hline
20.05           & 20.05          \\ \hline
\end{tabular}
\end{center}

\item Escreva um programa em \emph{Pascal} que leia dois valores numéricos 
inteiros e efetue a adição; se o resultado for maior que 10, imprima o 
primeiro valor. Caso contrário, imprima o segundo.

\begin{center}
\begin{tabular}{|l|l|} \hline
Exemplo Entrada & Saída esperada \\ \hline
7                &                \\
4                &   7             \\ \hline
7                &                \\
2                &   2             \\ \hline
3                &                \\
7                &   7             \\ \hline
\end{tabular}
\end{center}

\item Escreva um programa em \emph{Pascal} que imprima se um dado número $N$ 
inteiro (recebido através do teclado) é PAR ou ÍMPAR.

\begin{center}
\begin{tabular}{|l|l|} \hline
Exemplo Entrada & Saída esperada \\ \hline
5                & impar               \\ \hline
3                & impar               \\ \hline
2                & par               \\ \hline
\end{tabular}
\end{center}

\item Escreva um programa em \emph{Pascal} para determinar se um dado número 
$N$ (recebido através do teclado) é POSITIVO, NEGATIVO ou NULO.

\begin{center}
\begin{tabular}{|l|l|} \hline
Exemplo Entrada & Saída esperada \\ \hline
5               & positivo               \\ \hline
-3              & negativo               \\ \hline
0               & nulo               \\ \hline
\end{tabular}
\end{center}

\item Escreva um programa em \emph{Pascal} que leia dois números e efetue a 
adição. Caso o valor somado seja maior que 20, este deverá ser apresentado 
somando-se a ele mais 8; caso o valor somado seja menor ou igual a 20, 
este deverá ser apresentado subtraindo-se 5.

\begin{center}
\begin{tabular}{|l|l|} \hline
Exemplo Entrada & Saída esperada \\ \hline
13.14                &                \\
5                &    13.14            \\ \hline
-3                &                \\
-4                &   -12.00             \\ \hline
16                &                \\
5                &    20.00            \\ \hline
\end{tabular}
\end{center}

\item Escreva um programa em \emph{Pascal} que imprima qual o menor valor de 
dois números $A$ e $B$, lidos através do teclado.

\begin{center}
\begin{tabular}{|l|l|} \hline
Exemplo Entrada & Saída esperada \\ \hline
5.35                &                \\
4                & 4.00               \\ \hline
-3                &                \\
1                & -3.00               \\ \hline
6                &                \\ 
15                & 6.00               \\ \hline
\end{tabular}
\end{center}

\item Escreva um programa em \emph{Pascal} para determinar se um número 
inteiro $A$ é divisível por um outro número inteiro $B$. Esses valores 
devem ser fornecidos pelo usuário.

\begin{center}
\begin{tabular}{|l|l|} \hline
Exemplo Entrada & Saída esperada \\ \hline
5                &                \\
10               & nao               \\ \hline
4                &                \\ 
2                & sim               \\ \hline
7                &                \\ 
21               & nao              \\ \hline
\end{tabular}
\end{center}

\item Escreva um programa em \emph{Pascal} que leia um número inteiro e 
informe se ele é ou não divisível por 5.

\begin{center}
\begin{tabular}{|l|l|} \hline
Exemplo Entrada & Saída esperada \\ \hline
5                & sim               \\ \hline
-5                & sim               \\ \hline
3                &  nao              \\ \hline
\end{tabular}
\end{center}

\item Escreva um programa em \emph{Pascal} que receba um número inteiro e 
imprima se este é múltiplo de 3.

\begin{center}
\begin{tabular}{|l|l|} \hline
Exemplo Entrada & Saída esperada \\ \hline
5                & nao               \\ \hline
-3                & sim               \\ \hline
15                & sim               \\ \hline
\end{tabular}
\end{center}

\item Escreva um programa em \emph{Pascal} que leia um número e imprima a 
raiz quadrada do número caso ele seja positivo ou igual a zero e o quadrado 
do número caso ele seja negativo.

\begin{center}
\begin{tabular}{|l|l|} \hline
Exemplo Entrada & Saída esperada \\ \hline
0                & 0.00               \\ \hline
4                & 2.00               \\ \hline
-5               & 25.00               \\ \hline
\end{tabular}
\end{center}

\item Escreva um programa em \emph{Pascal} que leia um número e informe se ele 
é divisível por 3 e por 7.

\begin{center}
\begin{tabular}{|l|l|} \hline
Exemplo Entrada & Saída esperada \\ \hline
21               & sim               \\ \hline
7                & nao               \\ \hline
3                & nao               \\ \hline
-42              & sim               \\ \hline
\end{tabular}
\end{center}

\item A prefeitura de Contagem abriu uma linha de crédito para os funcionários 
estatutários. O valor máximo da prestação não poderá ultrapassar 30\% do 
salário bruto. Fazer um programa em \emph{Pascal} que permita entrar com o 
salário bruto e o valor da prestação, e informar se o empréstimo pode ou não 
ser concedido.

\begin{center}
\begin{tabular}{|l|l|} \hline
Exemplo Entrada & Saída esperada \\ \hline
500                &                \\
200                & nao               \\ \hline
1000.50                &                \\
250.10                & sim               \\ \hline
1000                &                \\
300                & sim               \\ \hline
\end{tabular}
\end{center}

\item Escreva um programa em \emph{Pascal} que dado quatro valores, 
$A$, $B$, $C$ e $D$, o programa imprima o menor e o maior valor.

\begin{center}
\begin{tabular}{|l|l|} \hline
Exemplo Entrada & Saída esperada \\ \hline
1                &                \\ 
2                &                \\ 
3                &                \\ 
4                & 1.00 4.00               \\ \hline
-3                &                \\ 
0               &                \\ 
1               &                \\ 
1               &  -3.00 1.00              \\ \hline
3.5                &                \\
3.7                &                \\
4.0                &                \\
5.5                & 3.50 5.50               \\ \hline
\end{tabular}
\end{center}

\item Dados três valores $A$, $B$ e $C$, construa um programa em \emph{Pascal},
 que imprima os valores de forma ascendente (do menor para o maior).

\begin{center}
\begin{tabular}{|l|l|} \hline
Exemplo Entrada & Saída esperada \\ \hline
1 2 1.5                & 1.00 1.50 2.00               \\ \hline
-3 -4 -5              & -5.00 -4.00 -3.00               \\ \hline
6 5 4                & 4.00 5.00 6.00               \\ \hline
\end{tabular}
\end{center}

\item Dados três valores $A$, $B$ e $C$, construa um programa em \emph{Pascal},
 que imprima os valores de forma descendente (do maior para o menor).

\begin{center}
\begin{tabular}{|l|l|} \hline
Exemplo Entrada & Saída esperada \\ \hline
1 2 1.5                & 2.00 1.50 1.00               \\ \hline
-5 -4 -3                & -3.00 -4.00 -5.00                \\ \hline
5 6 4                & 6.00 5.00 4.00               \\ \hline
\end{tabular}
\end{center}

\item Escreva um programa em \emph{Pascal} que leia dois números e imprimir 
o quadrado do menor número e raiz quadrada do maior número, se for possível.

\begin{center}
\begin{tabular}{|l|l|} \hline
Exemplo Entrada & Saída esperada \\ \hline
4                &                \\
3                & 9.00 2.00               \\ \hline
4.35                &                \\
3.50                & 12.25 2.09              \\ \hline
-4                &                \\ 
-16                & 256.00               \\ \hline
\end{tabular}
\end{center}

\item Escreva um programa em \emph{Pascal} que indique se um número digitado 
está compreendido entre 20 e 90 ou não (20 e 90 não estão na faixa de valores).

\begin{center}
\begin{tabular}{|l|l|} \hline
Exemplo Entrada & Saída esperada \\ \hline
50.50                & sim               \\ \hline
20                & nao               \\ \hline
90                & nao               \\ \hline
\end{tabular}
\end{center}

\item Escreva um programa em \emph{Pascal} que leia um número inteiro e 
informe se ele é divisível por 10, por 5 ou por 2 ou se não é divisível 
por nenhum deles.

\begin{center}
\begin{tabular}{|l|l|} \hline
Exemplo Entrada & Saída esperada \\ \hline
10                &  10 5 2              \\ \hline
5                &  5              \\ \hline
4                &  2              \\ \hline
7               &  nenhum              \\ \hline
\end{tabular}
\end{center}

\item Escreva um programa em \emph{Pascal} que leia um número e imprima se 
ele é igual a 5, a 200, a 400, se está no intervalo entre 500 e 1000, 
inclusive, ou se está fora dos escopos anteriores.

\begin{center}
\begin{tabular}{|l|l|} \hline
Exemplo Entrada & Saída esperada \\ \hline
5                & igual a 5               \\ \hline
200                & igual a 200               \\ \hline
400                & igual a 400               \\ \hline
750.50                & intervalo entre 500 e 1000               \\ \hline
1000                & intervalo entre 500 e 1000               \\ \hline
1500                & fora dos escopos               \\ \hline
\end{tabular}
\end{center}

\item A CEF concederá um crédito especial com juros de 2\% aos seus clientes de 
acordo com o saldo médio no último ano. Fazer um programa em \emph{Pascal} que 
leia o saldo médio de um cliente e calcule o valor do crédito de acordo com a 
tabela a seguir. Imprimir uma mensagem informando o valor de crédito.

\begin{tabular}{|l|l|}\hline
De 0 a 500 & nenhum credito \\ \hline
De 501 a 1000 & 30\% do valor do saldo medio \\ \hline
De 1001 a 3000 & 40\% do valor do saldo medio \\ \hline
Acima de 3001 &  50\% do valor do saldo medio\\ \hline
\end{tabular}

\begin{center}
\begin{tabular}{|l|l|} \hline
Exemplo Entrada & Saída esperada \\ \hline
300.50                & 0.00                \\ \hline
571                & 171.30               \\ \hline
1492.35                & 596.94               \\ \hline
3001.20               &  1500.60               \\ \hline
\end{tabular}
\end{center}

\item Escreva um programa em \emph{Pascal} que dada a idade de uma pessoa, 
determine sua classificação segundo a seguinte tabela:
\begin{itemize}
\item Maior de idade;
\item Menor de idade;
\item Pessoa idosa (idade superior ou igual a 65 anos).
\end{itemize}

\begin{center}
\begin{tabular}{|l|l|} \hline
Exemplo Entrada & Saída esperada \\ \hline
18                & maior               \\ \hline
15                & menor               \\ \hline
65                & idosa               \\ \hline
\end{tabular}
\end{center}

\item Escreva um programa em \emph{Pascal} que leia a idade de uma pessoa 
e informe a sua classe eleitoral:
\begin{itemize}
\item não eleitor (abaixo de 16 anos);
\item eleitor obrigatório (entre a faixa de 18 e menor de 65 anos);
\item eleitor facultativo (de 16 até 18 anos e maior de 65 anos, inclusive).
\end{itemize}

\begin{center}
\begin{tabular}{|l|l|} \hline
Exemplo Entrada & Saída esperada \\ \hline
15                & nao eleitor               \\ \hline
16                & facultativo               \\ \hline
17                & facultativo               \\ \hline
18                & obrigatorio               \\ \hline
19                & obrigatorio               \\ \hline
\end{tabular}
\end{center}

\item A confederação brasileira de natação irá promover eliminatórias para o próximo mundial. Fazer um programa em \emph{Pascal} que receba a idade de um nadador e imprima a sua categoria segundo a tabela a seguir:

\begin{tabular}{|l|l|} \hline
Infantil A & 5 -- 7 anos \\ \hline
Infantil B & 8 -- 10 anos  \\ \hline
Juvenil A & 11 -- 13 anos \\ \hline
Juvenil B & 14 -- 17 anos \\ \hline
Sênior &  Maiores de 18 anos \\ \hline
\end{tabular}

\begin{center}
\begin{tabular}{|l|l|} \hline
Exemplo Entrada & Saída esperada \\ \hline
4 & INVÁLIDO \\ \hline
7 & Infantil A \\ \hline
8 & Infantil B \\ \hline
10 & Infantil B \\ \hline
11 & Juvenil A \\ \hline
13 & Juvenil A \\ \hline
14 & Juvenil B \\ \hline
17 & Juvenil B \\ \hline
18 & Sênior \\ \hline
\end{tabular}
\end{center}

\item Dados três valores $A$, $B$ e $C$, escreva um programa em \emph{Pascal} 
para verificar se estes valores podem ser valores dos lados de um triângulo, 
se é um triângulo ESCALENO, um triângulo EQUILÁTERO ou um triângulo ISÓSCELES. 
Caso não sejam válidos, imprimir: ``INVALIDO''.

\begin{center}
\begin{tabular}{|l|l|} \hline
Exemplo Entrada & Saída esperada \\ \hline
5 & \\
5 & \\
5 & EQUILATERO \\ \hline
7 & \\
7 & \\
5 & ISOSCELES \\ \hline
3 & \\
4 & \\
5 & ESCALENO \\ \hline
5 & \\
4 & \\
15 & INVALIDO \\ \hline
\end{tabular}
\end{center}

\item Escreva um programa em \emph{Pascal} que leia as duas notas bimestrais 
de um aluno e determine a média das notas semestral. Através da média 
calculada o algoritmo deve imprimir a seguinte mensagem: APROVADO, 
REPROVADO ou em EXAME (a média é 7 para Aprovação, menor que 3 para 
Reprovação e as demais em Exame).

\begin{center}
\begin{tabular}{|l|l|} \hline
Exemplo Entrada & Saída esperada \\ \hline
3.1 & \\
2.5 & REPROVADO \\ \hline
3 & \\
3 & EXAME \\ \hline
10 & \\
0 & EXAME \\ \hline
6 & \\
8 & APROVADO \\ \hline
10 & \\
10 & APROVADO \\ \hline
\end{tabular}
\end{center}

\item Depois da liberação do governo para as mensalidades dos planos de saúde, as pessoas começaram a fazer pesquisas para descobrir um bom plano, não muito caro. Um vendedor de um plano de saúde apresentou a tabela a seguir. Escreva um programa em \emph{Pascal} que entre com a idade de uma pessoa e imprima o valor que ela deverá pagar, segundo a seguinte tabela:

\begin{tabular}{|l|l|}
Até 10 anos & R\$ 30.00 \\ \hline
Acima de 10 até 29 anos & R\$ 60.00 \\ \hline
Acima de 29 até 45 anos & R\$ 120.00 \\ \hline
Acima de 45 até 59 anos & R\$ 150.00 \\ \hline
Acima de 59 até 65 anos & R\$ 250.00 \\ \hline
Maior do que 65 anos & R\$ 400.00 \\ \hline
\end{tabular}

\begin{center}
\begin{tabular}{|l|l|} \hline
10 & R\$ 30,00 \\ \hline
29 & R\$ 60,00 \\ \hline
45 & R\$ 120,00 \\ \hline
59 & R\$ 150,00 \\ \hline
65 & R\$ 250,00 \\ \hline
66 & R\$ 400,00 \\ \hline
\end{tabular}
\end{center}

\item Escreva um programa em \emph{Pascal} que leia o um número inteiro entre 1 e 7 e escreva o dia da semana correspondente. Caso o usuário digite um número fora desse intervalo, deverá aparecer a seguinte  mensagem: INEXISTENTE

\begin{center}
\begin{tabular}{|l|l|} \hline
Exemplo Entrada & Saída esperada \\ \hline
0 & INEXISTENTE \\ \hline
1 & DOMINGO \\ \hline
2 & SEGUNDA \\ \hline
3 & TERÇA \\ \hline
4 & QUARTA \\ \hline
5 & QUINTA \\ \hline
6 & SEXTA \\ \hline
7 & SÁBADO \\ \hline
8 & INEXISTENTE \\ \hline
\end{tabular}
\end{center}
