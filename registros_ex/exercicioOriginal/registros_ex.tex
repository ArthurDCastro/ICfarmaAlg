\subsection{Exercícios}

\begin{enumerate}


\item Latitude e longitude s\~ao especificados em graus, ($^{o}$), minutos ('),
segundos (''), e dire\c c\~ao (N, S, L, O). Por exemplo, a cidade A fica
na latitude 22$^{o}$17'34''N e longitude 53$^{o}$41'9''O.

Defina um tipo em \emph{Pascal} 
cujo nome seja {\bf localizacao}, e que seja constituido
de {\em longitude} e {\em latitude}, ambas do tipo {\bf coordenadas}. Para
isto voce ter\'a que definir o tipo {\bf coordenadas} como sendo constituido
de {\em grau}, {\em minutos}, {\em segundos} e {\em direcao}.

\item Declare um vetor onde cada elemento \'e um registro com os campos: nome, DDD, telefone, CPF, Idade.
\item Considerando o vetor n\~ao ordenado, encontrar e imprimir o nome do cliente mais jovem. Fa\c ca um 
procedimento para isto
\item Ordenar por ordem de nome. Fa\c ca um procedimento para isto.
\item Dado um CPF, localizar se o nome est\'a no vetor e imprimir todos os dados. Fa\c ca um procedimento para
isto.
\item Fa\c ca um procedimento que receba por valor par\^ametros para nome, DDD, telefone, CPF, Idade 
e o insira no vetor (que j\'a est\'a ordenado) em ordem alfab\'etica. N\~ao vale usar um vetor auxiliar.
Retornar por refer\^encia o vetor alterado

\item Considere o arquivo de uma empresa (chamado de ``func.dat'' -- um arquivo
de registros) contendo
para cada funcionário seu número, seu nível salarial e seu departamento.
Como a administração desta empresa é feita a nível departamental é
importante que no arquivo os funcionários de cada um dos departamentos
estejam relacionados entre si e ordenados sequencialmente pelo seu número.
Como são frequentes as mudanças interdepartamentais no quadro de
funcionários, não é conveniente reestruturar o arquivo a cada uma destas
mudanças.  Desta maneira, o arquivo poderia ser organizado da seguinte
forma:

\begin{verbatim}
linha numFunc nivel departamento proximo
0     123     7     1            5
1     8765    12    1            -1
2     9210    4     2            -1
3     2628    4     3            6
4     5571    8     2            -1
5     652     1     1            9
6     7943    1     3            -1
7     671     5     3            12
8     1956    11    2            11
9     1398    6     1            10
10    3356    3     1            1
11    4050    2     2            4
12    2468    9     3            3
\end{verbatim}




Em um segundo arquivo (chamado ``depto.dat'' -- um arquivo de registros)
temos as seguintes informações:

\begin{verbatim}
codDepto nomeDepto      inicio
1        vendas         0
2        contabilidade  8
3        estoque        7
4        entrega        2
\end{verbatim}

Assim, o primeiro funcionário do departamento de vendas é o registro 0 do
arquivo de funcionários e os demais funcionários do mesmo departamento são
obtidos seguindo o campo {\tt proximo}.  Ou seja, os funcionários do
departamento de vendas são os funcionários nos registros: 0, 5, 9, 10 e 1.
Os funcionários do departamento de contabilidade são os funcionários nos
registros: 8, 11 e 4.

Escreva um programa em \emph{Pascal} que realize as seguintes operações:
\begin{itemize}
\item admissão de novo funcionário
\item demissão de funcionário
\item mudança de departamento por um funcionário
\end{itemize} 

Para estas operações devem ser lidas as informações:
\begin{itemize}
\item código do tipo da operação: 0 para fim, 1 para admissão, 2 para
  demissão e 3 para mudança de departamento
\item número do funcionário
\item nível salarial (somente no caso de admissão)
\item número do departamento ao qual o funcionário passa a pertencer (no
  caso de admissão e mudança)
\item número do departamento do qual o funcionário foi desligado (no caso
  de demissão e mudança)
\end{itemize}



O programa deve escrever as seguintes informações:
\begin{itemize}
\item os valores iniciais lidos dos arquivos
\item para cada operação: o tipo da operação realizada, os dados da
operação e a forma final dos dados (de funcionários e departamentos)
\end{itemize}

No final do programa novos arquivos ``func.dat'' e ``depto.dat'' são
gerados com os dados atualizados.

Detalhamento:
\begin{itemize}
\item a quantidade máxima de funcionários é 1000
\item a quantidade máxima de departamentos é 20
\item se a quantidade máxima for ultrapassada o programa deve dar uma
mensagem de erro
\item se for requisitada a remoção ou mudança de um funcionário não
  existente no departamento especificado o programa deve dar uma mensagem
  de erro.
\item quando for requisitada a inserção de um novo funcionário é preciso
  verificar se um funcionário com o mesmo número já existe.
\item se o código de operação for inválido o programa deve continuar lendo
um novo código até que ele seja 0 (zero), 1 (um), 2 (dois) ou 3 (três).
\end{itemize}


\item Uma empreiteira tem contratos para construir diversos tipos de casa: moderno,
fazenda, colonial, etc.   A quantidade de material empregada para cada
tipo de casa está armazenada em um arquivo chamado  ``material.dat''.
Este é um {\em arquivo de registros} contendo: o tipo de casa e as respectivas
quantidades de cada material necessárias para sua
construção. Existem no máximo 50 tipos de casas e 100 tipos distintos de
materiais. 

Cada tipo de casa é representado por um código inteiro no intervalo [1,50].
Por exemplo:  
\begin{itemize}
\item 1: moderno
\item 2: fazenda
\item 3: colonial, {\bf ...}
\end{itemize}

Cada tipo de material é representado por um código inteiro no intervalo
[1,100].  Por exemplo: 
\begin{itemize}
\item 1: ferro
\item 2: madeira
\item 3: vidro
\item 4: tinta
\item 5: tijolo, {\bf ...}
\end{itemize}

Em um segundo arquivo, chamado ``preco.dat'', encontram-se os
preços de cada um dos materiais.
Este também é um {\em arquivo de registros} contendo: o código do material
e o seu 
respectivo preço.   O código do material segue a mesma codificação
utilizada pelo arquivo de materiais.

Escreva um programa \emph{Pascal} que leia os dados do arquivo ``material.dat'' em
uma matriz e o dados do arquivo ``preco.dat'' em um vetor, como ilustrado
abaixo.

\item Usando as estruturas de dados abaixo escreva um procedimento em
\emph{Pascal} que recebe como parâmetro uma estrutura do tipo 
\verb+TAGENDA+ e ordena de forma crescente o vetor \verb+pessoa+ 
dessa estrutura tomando como referência para a ordenação  
o campo \verb+nome+ da estrutura \verb+TPESSOA+. Ou seja,
ordena uma agenda pessoal de telefones e endereços em ordem 
crescente do nome das pessoas presentes na agenda.
Você \emph{deve} usar a função $compara( r, s )$, que recebe dois 
parâmetros do tipo \verb+string+ e retorna 0 se $r$ e $s$ são
iguais, 1 se $r$ é lexicograficamente maior que $s$ e $-1$ 
se $r$ é lexicograficamente menor que $s$. Um nome $n_1$ é 
lexicograficamente maior que um nome $n_2$ se $n_1$ aparece 
depois de $n_2$ numa ordenação alfabética crescente desses
nomes.
\begin{center}
\begin{small}
\begin{verbatim}
Const
  MAX = 1000;
Type
   TPESSOA = record
                nome: string;
                telefone: string;
                endereco: string
             end; 
   TAGENDA = record
                pessoa: array [1..MAX] of TPESSOA;
                tamanho: integer
             end;
\end{verbatim}
\end{small}
\end{center}

\item Uma matriz é dita esparsa quando a maioria dos seus elementos possui valor $0.0$
(zero). Neste caso, a representação da matriz sob a forma tradicional (um
\emph{array} bidimensional) implica em uma utilização ineficiente da memória.
Por isso, matrizes esparsas são freqüentemente representadas como vetores de
elementos não nulos, sendo que cada elemento contém suas coordenadas e seu
valor.

Exemplo:
\begin{displaymath}
    \mathbf{M} = \left[ \begin{array}{cccc}
                         0 & 0 & 0 & 1.2 \\
                         7.3 & 0 & 99 & 0 \\
                         0 & 2 & 0 & 0 \\
                         0 & 17 & 0 & 0
                        \end{array} \right] \Longleftrightarrow   
    \mathbf{M_e} = \left[ \begin{array}{ccccc}
                         \begin{array}{|c|c|}
                                1 & 4 \\ \hline \multicolumn{2}{|c|}{1.2}
                               \end{array} &
                         \begin{array}{|c|c|}
                                2 & 1 \\ \hline \multicolumn{2}{|c|}{7.3}
                               \end{array} &
                         \begin{array}{|c|c|}
                                2 & 3 \\ \hline \multicolumn{2}{|c|}{99}
                               \end{array} &
                         \begin{array}{|c|c|}
                                3 & 2 \\ \hline \multicolumn{2}{|c|}{2}
                               \end{array} &
                         \begin{array}{|c|c|}
                                4 & 2 \\ \hline \multicolumn{2}{|c|}{17}
                               \end{array}
                    \end{array} \right]
\end{displaymath}
Para representar estas matrizes em \emph{Pascal}, podemos definir as
seguintes estruturas de dados:
\begin{lstlisting}
Const MAX = 1000; MAX_ESP = MAX*MAX/10;
Type t_matriz = record
        lin,col : integer;
        dados : array [1..MAX, 1..MAX] of real;
     end;
     elemento = record
        l,c : integer;
        val : real;
     end;
     t_matrizesp = record
        tam : integer;
        dados : array [1..MAX_ESP] of elemento;
     end;
\end{lstlisting}

Utilizando as estruturas de dados definidas acima, faça:
\begin {enumerate}

\item  Escreva uma função que transforme uma matriz do tipo
\verb|t_matriz| em uma matriz do tipo \verb|t_matrizesp|.

\item  Escreva uma função que transforme uma matriz do tipo
\verb|t_matrizesp| em uma matriz do tipo \verb|t_matriz|.
 
\item  Escreva uma função que receba duas matrizes do tipo 
\verb|t_matrizesp| e imprima o resultado da \textbf{soma} destas matrizes. O
resultado deve ser impresso na forma bidimensional, com os valores de cada
linha separados por espaços.

\end{enumerate}

\item Verdadeiro ou falso: 
\begin{itemize}
   \item  um record deve ter pelo menos dois campos
   \item  os campos de um record tem que ter nomes diferentes
   \item  um record deve ter um nome diferente de qualquer um dos seus campos.
\end{itemize}

\item Suponha que a linguagem \emph{Pascal} foi modificada para permitir que o 
   símbolo ponto "." possa fazer parte de identificadores. Qual problema
   isto poderia causar? Dê um exemplo.

\item Esta definição é legal? Porque não? 

\begin{verbatim}
   TYPE 
        Ponto = RECORD
                    quantidade: integer;
                    corte: Tamanho;
                END;
        Tamanho = (mini, medio, maxi);
\end{verbatim}

\item 
 Latitudes e  longitudes são especificadas em graus  (o), minutos ('),
   segundos ('') e direção (norte, sul, leste, oeste). Suponha que uma
   cidade   esteja   na    latitude   22o17'34''   norte   e   longitude
   53o41'9''oeste.  Armazene esta localização  na variável  Cidade, como
   declarada abaixo:
    
\begin{verbatim}
   TYPE 
        TipoDirecao = (norte, sul, leste, oeste);
        Coordenadas = RECORD
                          graus: 0..180;
                          minutos, segundos: 0..60;
                          direcao: TipoDirecao;
                      END;
        Localizacao = RECORD
                          latitude, longitude: Coordenadas;
                      END
   VAR
        Cidade: Localizacao;
\end{verbatim}

\item Suponha que temos as seguintes definições e declarações abaixo:

\begin{verbatim}
   TYPE 
        NumTelefone = RECORD
                          Area, Prefixo, Numero: integer;
                      END;
   VAR
        Casa, Escritorio, Celular: NumTelefone;
\end{verbatim}

   Escreva código \emph{Pascal} par testar  se Escritorio e Celular estão sobre
   o mesmo código de área e para atribuir o mesmo número do Celular como
   sendo o do Escritório.

\item Criar registros para as seguintes configurações da vida real:
\begin{itemize}
  \item Pessoas com nome, endereço, telefone, sexo, idade e salário;
   \item Planetas  com massa, distância  média do sol, número  de satélites,
     nome, diâmetro. 
   \item Cursos,  com alunos e suas  notas, períodos em  que são oferecidos,
     professor, horário.
\end{itemize}

\item Uma matriz é dita \textit{esparsa} se o 
número de elementos não nulos for bastante pequeno com relação
ao número total de elementos da matriz. Você deve fazer um programa
que leia do teclado uma matriz qualquer de números \textit{reais} e crie um 
vetor que armazene somente os elementos não nulos da matriz. Cada
posição do vetor deve ser um registro contendo o valor do elemento,
e a posição desse elemento na matriz (linha e coluna).
Imprima o resultado. A maneira de definir o vetor faz parte da prova.
Exemplo:
Seja a matriz $4 \times 5$ abaixo:

\begin{center}
\begin{tabular}{|l|l|l|l|l|}  \hline
0.0  & 0.0 & 3.0 & 0.0  & 1.7 \\ \hline
-1.1 & 0.0 & 0.0 & 0.0  & 0.0 \\ \hline
0.0  & 0.0 & 0.0 & -2.1 & 0.0 \\ \hline
0.0  & 0.0 & 2.5 & 0.0  & 0.0 \\ \hline
\end{tabular}
\end{center}

Seu programa deverá construir um 
vetor que permita imprimrir as seguintes informações:

\begin{itemize}
\item o elemento 3.0 está na posição (1,3) do vetor;
\item o elemento 1.7 está na posição (1,5) do vetor;
\item o elemento -1.1 está na posição (2,1) do vetor;
\item o elemento -2.1 está na posição (3,4) do vetor;
\item o elemento 2.5 está na posição (4,3) do vetor;
\end{itemize}

\item Considere uma estrutura de vetores para representar uma matriz 
esparsa, tal como você definiu na questão anterior. Faça
um procedimento para imprimir a matriz completa, contendo todos
os elementos nulos e não nulos 
(não precisa fazer o programa todo, basta o procedimento).
Por exemplo, dado vetor com as informações descritas no 
exemplo da questão anterior, a saída do seu programa deve
ser exatamente a matriz apresentada no início do exemplo.

\item Considere o tipo PGM para imagens como definido na seção~\ref{pgm}.
Nas questões que seguem, considere as seguintes estruturas de dados e 
assinaturas de funções e procedimentos:

\begin{lstlisting}
const MAX=10000;

type
     matriz = array [1..MAX,1..MAX] of integer;

     vetor  = array [1..MAX] of integer;

     imagem = record
                 col, lin, maior: integer;
                 m: matriz;
     end;

     imgcompactada = record
                 tam: integer;
                 v: vetor;
     end;
  
function calcula_valor_medio (var I: imagem): integer;
(* funcao que retorna o valor medio dos pixels da imagem, isto eh
   a soma de todos os elementos dividido pelo numero de elementos *)

procedure ler (var I: imagem);
(* procedimento que le uma imagem no formato PGM *)

procedure imprime_imagem (var I: imagem);
(* procedimento que imprime uma imagem no formato PGM *)
procedure binariza (var I: imagem; limiar: integer);
(* procedimento que transforma a imagem de tons de cinza para preto e branco
   para isto, os pixels que forem maiores que o limiar devem se tornar brancos
   e os que forem menores ou iguais a este mesmo limiar devem se tornar pretos *)

procedure compacta_imagem (var I: imagem; var C: imgcompactada);
(* procedimento que recebe uma imagem no formato PGM e cria um vetor C
   que eh uma representacao compactada desta                           *)

procedure imprime_img_compactada (var C: imgcompactada);
(* procedure que recebe uma imagem compactada e a imprime no formato PGM *)
\end{lstlisting}

\begin {enumerate}

\item  Implemente estas funções e procedimentos e faça um programa que 
receba um certo número $N$ de imagens PGM em tons de cinza (onde 0
representa preto e branco é representado pelo maior valor da imagem) e
imprima a imagem binarizada, isto é, em preto e branco (onde 0 
representa preto e 1 representa branco). Note que o limiar é obtido
pelo valor médio dos pixels.

\item 
Implemente um procedimento que gere um vetor que representa a matriz 
binarizada de forma compacta. Para isto, use a seguinte ideia: como a
matriz só tem zeros e uns, vamos substituir sequências de uns pelo número
de uns consecutivos. Os elementos vão sendo colocados no vetor, de maneira
linear, cada linha seguinte é concatenada à anterior. Veja o exemplo:
Exemplo:
\begin{itemize}
\vspace*{-2mm}      
    \item \textbf{Imagem binarizada}:
    \begin{verbatim}
    P2
    11 10
    1
    1  1  1  1  0  1  1  1  1  1  0 
    1  1  0  1  1  1  1  1  1  1  1 
    0  0  1  0  0  0  1  1  1  0  0
    1  1  1  1  1  1  1  1  1  1  1
    0  0  0  1  1  1  1  1  0  0  0
    0  0  0  1  1  1  1  1  1  1  1
    1  1  1  0  1  1  1  1  1  1  1
    1  1  1  1  1  1  1  0  1  1  1
    1  1  1  1  1  1  1  1  1  0  0
    0  1  1  1  1  1  1  1  1  1  1
    \end{verbatim}

\vspace*{-4mm}      
    \item \textbf{Imagem compactada}:
    \begin{verbatim}
36
4 0 5 0 2 0 8 0 0 1 0 0 0 3 0 0 11 0 0 0 5 0 0 0 0 0 0 11 0 14 0 12 0 0 0 10
\end{verbatim}
\end{itemize}

Isto é, a primeira linha da matriz possui 4 uns consecutivos seguido de um
zero e outros 5 uns consecutivos, por isto, o vetor contém seus primeiros
elementos ``4, 0 e 5''. Preste atenção antes de escrever o código. Você pode 
definir, se precisar, funções, procedimentos ou estruturas de dados adicionais.
\end{enumerate}



\end{enumerate}
