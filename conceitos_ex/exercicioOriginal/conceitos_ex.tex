\section{Exercícios}

Baixe o mini guia da linguagem \emph{Pascal}, disponível em:

\begin{center}
\url{http://www.inf.ufpr.br/cursos/ci055/pascal.pdf},
\end{center}

\noindent também disponível online em:

\begin{center}
\url{http://wiki.inf.ufpr.br/marcos/doku.php?id=pascal}.
\end{center}

Você vai precisar estudá-lo para poder resolver boa parte dos
exercícios desta seção, pois como já informado, este material complementa
partes dependentes da versão da linguagem \emph{Pascal}, versão \emph{Free
Pascal}, que naturalmente evolui com o tempo. 

\vspace*{\baselineskip}

\noindent
\textbf{Os exemplos de entrada e saída que acompanham os enunciados
devem ser vistos como \emph{um} caso de entrada, você deve testar
com diferentes entradas para se certificar que o seu programa funciona
para diferentes entradas.}

\begin{enumerate}

%%% expressões aritméticas
\subsection{Expressões aritméticas}
\item Considere o seguinte programa incompleto em \emph{Pascal}:

\begin{lstlisting}
program tipos;
var 
     A: <tipo>;
     B: <tipo>;
     C: <tipo>;
     D: <tipo>;
     E: <tipo>;
begin
     A := 1 + 2 * 3;
     B := 1 + 2 * 3 / 7;
     C := 1 + 2 * 3 div 7;
     D := 3 div 3 * 4.0;
     E := A + B * C - D
end.
\end{lstlisting}

Você deve completar este programa indicando, para cada variável de $A$ até
$E$, qual é o tipo correto desta variável. Algumas delas podem ser tanto inteiras 
como reais, enquanto que algumas só podem ser de um tipo específico.
Para resolver este exercício você precisa estudar sobre os operadores
inteiros e reais e também sobre a ordem de precedência de operadores
que aparecem em uma expressão aritimética. Sua solução estará correta se seu 
programa compilar.

\item Escreva um programa em \emph{Pascal} que leia 
6 valores reais para as variáveis $A, B, C, D, E, F$ e 
imprima o valor de $X$ após o cálculo 
 da seguinte expressão aritmética:

\[ 
X = \frac{\frac{A + B}{C - D}E}{\frac{F}{AB} + E}
\]

Seu programa deve assumir que nunca haverá divisões por zero
para as variáveis dadas como entrada. Note que neste programa
a variável $X$ deve ser do tipo \emph{real}, enquanto que 
as outras variáveis podem ser tanto da família \emph{ordinal}
(\emph{integer, longint, etc}) como também podem ser do tipo
\emph{real}.

\begin{center}
\begin{tabular}{|l|l|} \hline
Exemplo Entrada & Saída esperada \\ \hline
1 2 3 4 5 6     & -1.8750000000000000E+000  \\ \hline
1 -1 1 -1 1 -1  & 0.0000000000000000E+000   \\ \hline
3 5 8 1 1 2     & 1.0084033613445378E+000   \\ \hline
\end{tabular}
\end{center}

\item Escreva em \emph{Pascal} as seguintes expressões
aritméticas usando o mínimo possível de parênteses.
Para resolver este exercício você precisa estudar sobre precedência
de operadores em uma expressão aritmética. Dica: para elevar um número
ao quadrado multiplique este número por ele mesmo ($x^2 = x * x$).

\begin{enumerate}
\item 
     \[ \frac{W^2}{Ax^2 + Bx +C} \]

\item 
     \[ \frac{\frac{P_1 + P_2}{Y - Z}R}{\frac{W}{AB} + R} \]
\end{enumerate}

\noindent
Observe que os compiladores não suportam o uso de subescritos, que são
utilizados na notação matemática. Então no lugar de $P_1$ e $P_2$, você
pode dar os nomes para as variáveis de $p1$ e $p2$ respectivamente.

\item Faça um programa em \emph{Pascal}
que some duas horas. A entrada deve ser feita lendo-se
dois inteiros por linha, em duas linhas, e a saída
deve ser feita no formato especificado no exemplo
abaixo:

\begin{center}
\begin{tabular}{|l|l|} \hline
Exemplo Entrada & Saída esperada \\ \hline
12 52           &                \\ 
7 13            & 12:52 + 7:13 = 20:05 \\ \hline
20 15           &                \\ 
1 45            & 20:15 + 1:45 = 22:00  \\ \hline
0 0             &                \\ 
8 35            & 0:0 + 8:35 = 8:35  \\ \hline
\end{tabular}
\end{center}

Você deve observar que o comando de impressão deve imprimir os espaços em branco e
os símbolos ``+'' e ``='' conforme o enunciado exige.

\item Dado um número inteiro que representa uma quantidade de segundos,
   determinar o seu valor equivalente em graus, minutos e segundos. Se
   a quantidade de segundos for insuficiente para dar um valor em graus,
   o valor em graus deve ser 0 (zero). A mesma observação vale em
   relação aos minutos e segundos. 

\begin{center}
\begin{tabular}{|l|l|} \hline
Exemplo Entrada & Saída esperada \\ \hline
3600            & 1, 0, 0        \\ \hline
3500            & 0, 58, 20      \\ \hline
7220            & 2, 0, 20       \\ \hline
\end{tabular}
\end{center}

\item Fazer um programa em \emph{Pascal} que troque o conteúdo de duas 
variáveis. Exemplo:

\begin{center}
\begin{tabular}{|l|l|} \hline
Exemplo Entrada & Saída esperada \\ \hline
3 7             & 7 3            \\ \hline
-5 15           & 15 -5          \\ \hline
2 10            & 10 2           \\ \hline
\end{tabular}
\end{center}

\item (*) Desafio:
Fazer um programa em \emph{Pascal}
que troque o conteúdo de duas variáveis \emph{inteiras}
sem utilizar variáveis auxiliares. Pense em fazer contas
de adição e/ou subtração com os números. 




%%% expressões booleanas
\subsection{Expressões booleanas}
  Para resolver estes exercícios
  você precisa estudar sobre sobre expressões lógicas (ou booleanas) e sobre a 
  ordem de precedência dos operadores lógicos (NOT, AND, OR).


\item Indique qual o resultado das expressões abaixo, sendo: 

\[
  a=6; b=9.5; d=14; p=4; q=5; r=10; z=6.0 ; sim= TRUE.
\]

\begin {enumerate}
\item \verb|sim AND (q \ge p)|
\item \verb|(0 \le b) AND (z > a) OR (a = b)|
\item \verb|(0 \le b) AND ((z > a) OR (a = b))|
\item \verb|(0 \le b) OR ((z > a) AND (a = b))|
\end {enumerate}




%%% expressões aritméticas e booleanas
\subsection{Expressões aritméticas e booleanas}
  Para resolver estes exercícios
  você precisa estudar sobre os operadores
  inteiros e reais e também sobre a ordem de precedência de operadores
  que aparecem em uma expressão aritimética. Adicionalmente, você precisa
  estudar sobre expressões lógicas (ou booleanas) e sobre a ordem de
  precedência dos operadores relacionais $(=, <>, \le, \ge, >, <)$ e
  lógicos (NOT, AND, OR).

\item Indique qual o resultado das expressões abaixo, sendo: 

\[
  a=6; b=9.5; d=14; p=4; q=5; r=10; z=6.0 ; sim= TRUE.
\]

\begin {enumerate}
\item \verb|NOT sim AND (z DIV b + 1 = r)|
\item \verb|(x + y > z) AND sim OR (d \ge b)|
\item \verb|(x + y <> z) AND (sim OR (d \ge b))|
\end {enumerate}

\item Indique qual o resultado das expressões abaixo, sendo: 

\[
  a=5; b=3; d=7;  p=4; q=5; r=2; x=8; y=4; z=6; sim=TRUE.
\]

\begin{enumerate}
\item \verb|(z DIV a + b * a) - d DIV 2|
\item \verb|p / r mod q - q / 2|
\item \verb|(z DIV y + 1 = x) AND sim OR (y >= x)|
\end{enumerate}



%%% acompanhamento de programas
\subsection{Acompanhamento de programas}
\item Dado o programa em \emph{Pascal} abaixo, mostre o acompanhamento de sua 
execução para três valores de entrada (valores pequenos, por exemplo para 
$x=0$, $x=10$ e $x=-1$).
Em seguida, descreva o que o programa faz.

\begin{lstlisting}
program questao1;
var
  m, x, y: integer;
begin
  read(x);
  y := 0;
  m := 1;
  while x > 0 do
    begin
      y := y + (x mod 2) * m;
      x := x div 2;
      m := m * 10;
    end;
  writeln(y)
end.
\end{lstlisting}



%%% programas simples, com 1 if... 
\subsection{Programas com um desvio condicional}

\item Faça um programa em \emph{Pascal} que leia um número $n$ do teclado
e decida se ele é positivo ou negativo. Seu programa deve imprimir
a mensagem ``par'' ou ``impar'' conforme o caso. Exemplo:

\begin{center}
\begin{tabular}{|l|l|} \hline
Exemplo Entrada & Saída esperada \\ \hline
5                & impar         \\ \hline
4                & par           \\ \hline
15               & impar         \\ \hline
\end{tabular}
\end{center}

\item Faça um programa em \emph{Pascal} que leia dois números $n, m$ do teclado
e decida se ele o primeiro é maior do que o segundo. Seu programa deve imprimir
a mensagem ``primeiro eh maior'' ou ``segundo eh maior ou igual'' conforme o caso. Exemplo:

\begin{center}
\begin{tabular}{|l|l|} \hline
Exemplo Entrada & Saída esperada \\ \hline
5 2             & primeiro eh maior               \\ \hline
2 5             & segundo eh maior ou igual       \\ \hline
5 5             & segundo eh maior ou igual       \\ \hline
\end{tabular}
\end{center}

\item Faça um programa em \emph{Pascal} que leia três números $x, y, z$ 
do teclado e decida se $x \le y < z$. 
Seu programa deve imprimir a mensagem ``esta no intervalo'' 
ou ``nao esta no intervalo'' conforme o caso. Exemplo:

\begin{center}
\begin{tabular}{|l|l|} \hline
Exemplo Entrada & Saída esperada \\ \hline
3 5 8           & esta no intervalo \\ \hline
3 8 8           & nao esta no intervalo \\ \hline
4 12 5          & nao esta no intervalo \\ \hline
\end{tabular}
\end{center}

\item Faça um programa em \emph{Pascal} que leia três números $x, y, z$ 
do teclado e decida se $x > y$ ou se $y < z$. 
Seu programa deve imprimir a mensagem ``sim'' em caso afirmativo e  
``nao'' caso contrário. Exemplo:

\begin{center}
\begin{tabular}{|l|l|} \hline
Exemplo Entrada & Saída esperada \\ \hline
3 5 8                & sim                \\ \hline
3 8 8                & nao               \\ \hline
4 12 5               & nao               \\ \hline
\end{tabular}
\end{center}

\item Escreva um programa em \emph{Pascal} que leia
6 valores reais para as variáveis $A, B, C, D, E, F$ e
imprima o valor de $X$ após o cálculo
 da seguinte expressão aritmética:

     \[ X = \frac{\frac{A + B}{C - D}E}{\frac{F}{AB} + E} \]

Seu programa deve imprimir a mensagem ``divisao por zero''
caso o denominador seja zero. Caso isso não ocorra seu programa
irá abortar neste caso, o que não é correto.

Exemplos de entrada e saída:

\begin{center}
\begin{tabular}{|l|l|} \hline
Exemplo Entrada & Saída esperada \\ \hline
1 2 3 4 5 6     & -1.8750000000000000E+000  \\ \hline
0 0 0 0 0 0     & divisao por zero   \\ \hline
1 1 2 2 1 3     & divisao por zero   \\ \hline
\end{tabular}
\end{center}



%%% programas simples, com 1 laço... 
\subsection{Programas com um laço}

\item Fazer um programa em \emph{Pascal} para
    ler uma massa de dados onde cada  linha da entrada contém um número  par.
    Para
    cada número lido,  calcular o seu sucessor par,  imprimindo-os dois a
    dois em  listagem de saída. A  última linha de dados  contém o número
    zero, o qual não deve ser processado e serve apenas para indicar o final
    da leitura dos dados. Exemplo:

\begin{center}
\begin{tabular}{|l|l|} \hline
Exemplo Entrada & Saída esperada \\ \hline
12 6 26 86 0    & 12 14          \\ 
                & 6 8            \\ 
                & 26 28          \\ 
                & 86 88          \\ \hline
-2 -5 -1 0      & -2 0           \\
                & -5 -3          \\ 
                & -1 1           \\ \hline
1 2 3 4 5 0     & 1 3            \\
                & 2 4            \\
                & 3 5            \\
                & 4 6            \\
                & 5 7            \\ \hline
\end{tabular}
\end{center}

\item Fazer um programa em \emph{Pascal} para
    ler  uma massa de  dados contendo a  definição de várias  equações do
    segundo grau da forma $Ax^{2} + Bx + C = 0$. Cada linha de dados contém a
    definição de uma equação por meio dos valores de $A$, $B$ e $C$ do conjunto
    dos  números reais.  A última  linha  informada ao  sistema contém  3
    (três) valores  zero (exemplo  0.0 0.0 0.0).  Após a leitura  de cada
    linha o  programa deve tentar calcular  as duas raízes  da equação. A
    listagem de saída, em cada  linha, deverá conter
    os valores das duas raízes reais. Considere
    que o usuário entrará somente com valores $A$, $B$ e $C$ tais que a equação
    garantidamente tenha duas raízes reais. 

\begin{center}
\begin{tabular}{|l|l|} \hline
Exemplo Entrada & Saída esperada \\ \hline
1.00 -1.00 -6.00 & -3.00 2.00 \\
1.00 0.00 -1.00  & -1.00 1.00 \\
1.00 0.00 0.00   & 0.00 0.00  \\ 
0.00 0.00 0.00   &            \\ \hline
\end{tabular}
\end{center}

\item Fazer um programa em \emph{Pascal} que receba dois números inteiros $N$ e
  $M$ como entrada e retorne como saída $N \ mod \ M$ (o resto da
  divisão inteira de $N$ por $M$) usando para isto apenas
  operações de subtração. O seu programa
  deve considerar que o usuário entra com $N$ sempre maior do que $M$.

\begin{center}
\begin{tabular}{|l|l|} \hline
Exemplo Entrada & Saída esperada \\ \hline
30 7            & 2              \\ \hline
3 2             & 1              \\ \hline
12 3            & 0              \\ \hline
\end{tabular}
\end{center}

\item Fazer um programa em \emph{Pascal} que leia um número $n > 0$ do
teclado e imprima a tabuada de $n$ de 1 até 10.

\begin{center}
\begin{tabular}{|l|l|} \hline
Exemplo Entrada & Saída esperada \\ \hline
5               & 5 x 1 = 5 \\
                & 5 x 2 = 10 \\
                & 5 x 3 = 15 \\
                & 5 x 4 = 20 \\
                & 5 x 5 = 25 \\
                & 5 x 6 = 30 \\
                & 5 x 7 = 35 \\
                & 5 x 8 = 40 \\
                & 5 x 9 = 45 \\
                & 5 x 10 = 50 \\ \hline
\end{tabular}
\end{center}

\end{enumerate}

\newpage

%%% exercícios complementares (do David Menotti Gomes)
\section{Exercícios complementares}

Estes exercícios\footnote{Os enunciados aqui apresentados foram compilados
pelo professor David Menotti Gomes e gentilmente cedidos para que 
constem neste material com o objetivo de servir de prática para os alunos
interessados em complementar seu domínio na programação em \emph{Pascal}.
Os autores modificaram minimamente o texto para fins de padronização com 
o restante dos enunciados deste livro. O professor David informou que 
os exemplos de execução foram adicionados pelo professor Marcelo da Silva, 
da Universidade Federal de Ouro Preto.}
complemantam os anteriores e podem ser feitos por aqueles
que querem um reforço nos conceitos deste capítulo que
é fundamental para compreensão do restante desta disciplina. Muitos
dos problemas aqui propostos são similares, os estudantes podem resolver
os problemas até se sentirem confiantes que compreenderam os conceitos
básicos de entrada e saída e expressões aritméticas e booleanas.

A maior parte dos problemas pode ser resolvida com base em conceitos 
básicos de Matemática e Física, mas sempre apresentamos as fórmulas
necessárias, pois o que está sendo solicitado são implementações de
conceitos fundamentais do ensino básico, que deveriam ser de conhecimento
dos alunos. 

\subsection{Programas com cálculos simples}
\begin{enumerate}

\item Escreva um programa em \emph{Pascal} que leia um número inteiro e 
imprima o seu sucessor e seu antecessor, na mesma linha.

\begin{center}
\begin{tabular}{|l|l|} \hline
Exemplo de entrada & Saída esperada \\ \hline
1               & 2 0 \\ \hline
100             & 101 99 \\ \hline
-3              & -2 -4 \\ \hline
\end{tabular}
\end{center}

\item Escreva um programa em \emph{Pascal} que leia dois números inteiros e 
imprima o resultado da soma destes dois valores. Antes do resultado, deve ser 
impressa a seguinte mensagem ``SOMA= ''.


\begin{center}
\begin{tabular}{|l|l|} \hline
Exemplo de entrada & Saída esperada \\ \hline
1 2             & SOMA= 3 \\ \hline
100 -50         & SOMA= 50 \\ \hline
-5 -40          & SOMA= -45 \\ \hline
\end{tabular}
\end{center}

\item Escreva um programa em \emph{Pascal} que leia dois números reais, um 
será o valor de um produto e outro o valor de desconto que esse produto está 
recebendo. Imprima quantos reais o produto custa na promoção.

\begin{center}
\begin{tabular}{|l|l|l|} \hline
\multicolumn{2}{|c|}{Exemplo de entrada} & Saída esperada \\ \hline
Valor original & Desconto & Valor na promoção \\ \hline
500.00         & 50.00  & 450.00 \\ \hline
10500.00       & 500.00 & 10000.00\\ \hline
90.00          & 0.80   & 89.20 \\ \hline
\end{tabular}
\end{center}

\item Escreva um programa em \emph{Pascal} que leia dois números reais e 
imprima a média aritmética entre esses dois valores.

\begin{center}
\begin{tabular}{|l|l|} \hline
Exemplo de entrada & Saída esperada \\ \hline
1.2 2.3         & 1.75 \\ \hline
750 1500        & 1125.00  \\ \hline
8900 12300      & 10600.00 \\ \hline
\end{tabular}
\end{center}

\item Escreva um programa em \emph{Pascal} que leia um número real e imprima a 
terça parte deste número.

\begin{center}
\begin{tabular}{|l|l|} \hline
Exemplo de entrada & Saída esperada \\ \hline
3               & 1.00 \\ \hline
10              & 3.33  \\ \hline
90              & 30.00 \\ \hline
\end{tabular}
\end{center}

\item Uma P.A. (progressão aritmética) fica determinada pela sua razão ($r$) 
e pelo primeiro termo ($a_1$). Escreva um programa em \emph{Pascal} que seja 
capaz de determinar o enésimo ($n$) termo ($a_n$) de uma P.A., dado a razão 
($r$) e o primeiro termo ($a_1$). Seu programa deve ler $n, r, a_1$ do teclado
e imprimir $a_n$.

\[
a_n = a_1 + (n-1)\times r.
\]

\begin{center}
\begin{tabular}{|l|l|l|l|} \hline
\multicolumn{3}{|c|}{Exemplo de entrada} & Saída esperada \\ \hline
$n$ & $r$ & $a_1$   & $a_n$               \\ \hline
8 & 1 & 3       & 10                \\ \hline
100 & 10 & 1    & 991                \\ \hline
5 & -2 & 0      & -98                \\ \hline
\end{tabular}
\end{center}

\item Dada a razão ($r$) de uma P.A. (progressão aritmética) e um termo 
qualquer, $k$ ($a_k$). Escreva um programa em \emph{Pascal} para calcular 
o enésimo termo $n$ ($a_n$). Seu programa deve ler $k, a_k, r, n$ do teclado
e imprimir $a_n$.

\[
a_n = a_k + (n-r) \times r
\]

\begin{center}
\begin{tabular}{|l|l|l|l|l|} \hline
\multicolumn{4}{|c|}{Exemplo de entrada} & Saída esperada \\ \hline
$k$ & $a_k$ & $r$ & n   & $a_n$             \\ \hline
1 & 5 & 2 & 10          & 23                \\ \hline
10 & 20 & 2 & 5         & 10                \\ \hline
100 & 500 & 20 & 90     & 300               \\ \hline
\end{tabular}
\end{center}

\item Uma P.G. (progressão geométrica) fica determinada pela sua razão ($q)$ 
e pelo primeiro termo ($a_1$). Escreva um programa em \emph{Pascal} que seja 
capaz de determinar o enésimo $n$ termo ($a_n$) de uma P.G., dado a razão ($q$) 
e o primeiro termo ($a_1$). Seu programa deve ler $a_1, q, n$ do teclado
e imprimir $a_n$.

\[
a_n = a_1 \times q^{(n-1)}.
\]

\begin{center}
\begin{tabular}{|l|l|l|l|} \hline
\multicolumn{3}{|c|}{Exemplo de entrada} & Saída esperada \\ \hline
$a_1$ & $q$ & $n$   & $a_n$               \\ \hline
1 & 1 & 100         & 1.00                \\ \hline
2 & 2 & 10          & 1024.00             \\ \hline
5 & 3 & 2           & 15.00               \\ \hline
\end{tabular}
\end{center}

\item Dada a razão ($q$) de uma P.G. (progressão geométrica) e um termo 
qualquer, $k$ ($a_k$). Escreva um programa em Pascal para calcular o enésimo 
termo $n$ ($an$). Seu programa deve ser $k, a_k, q, n$ do teclado e imprimir
$a_n$.

\begin{center}
\begin{tabular}{|l|l|l|l|l|} \hline
\multicolumn{4}{|c|}{Exemplo de entrada} & Saída esperada \\ \hline
$k$ & $a_k$ & $q$ & $n$  & $a_n$               \\ \hline
2 & 2 & 1 & 1        & 2                \\ \hline
1 & 5 & 2 & 10       & 2560.00             \\ \hline
2 & 100 & 10 & 20    & 100000000000000000000.00               \\ \hline
\end{tabular}
\end{center}

\item Uma P.G. (progressão geométrica) fica determinada pela sua razão ($q$) 
e pelo primeiro termo ($a_1$). Escreva um programa em \emph{Pascal} que 
seja capaz de determinar o enésimo termo ($a_n$) de uma P.G., dado a razão 
($q$) e o primeiro termo ($a_1$). Seu programa deve ler $a_1, q, n$ do 
teclado e imprimir $a_n$.

\begin{center}
\begin{tabular}{|l|l|l|l|} \hline
\multicolumn{3}{|c|}{Exemplo de entrada} & Saída esperada \\ \hline
$a_1$ & $q$ & $n$   & $a_n$               \\ \hline
1 & 1 & 100         & 1.00                \\ \hline
2 & 2 & 10          & 1024.00             \\ \hline
10 & 2 & 20         & 5242880.00          \\ \hline
\end{tabular}
\end{center}

\item Considere que o número de uma placa de veículo é composto por quatro 
algarismos. Escreva um programa em \emph{Pascal} que leia este número  do
teclado e apresente o algarismo correspondente à casa das unidades.

\begin{center}
\begin{tabular}{|l|l|} \hline
Exemplo de entrada & Saída esperada \\ \hline
2569                & 9               \\ \hline
1000                & 0               \\ \hline
1305                & 5               \\ \hline
\end{tabular}
\end{center}

\item Considere que o número de uma placa de veículo é composto por quatro 
algarismos. Escreva um programa em \emph{Pascal} que leia este número  do
teclado e apresente o algarismo correspondente à casa das dezenas.

\begin{center}
\begin{tabular}{|l|l|} \hline
Exemplo de entrada & Saída esperada \\ \hline
2569                & 6               \\ \hline
1000                & 0               \\ \hline
1350                & 5               \\ \hline
\end{tabular}
\end{center}

\item Considere que o número de uma placa de veículo é composto por quatro 
algarismos. Escreva um programa em \emph{Pascal} que leia este número  do
teclado e apresente o algarismo correspondente à casa das centenas.

\begin{center}
\begin{tabular}{|l|l|} \hline
Exemplo de entrada & Saída esperada \\ \hline
2500                & 5               \\ \hline
2031                & 0               \\ \hline
6975                & 9               \\ \hline
\end{tabular}
\end{center}

\item Considere que o número de uma placa de veículo é composto por quatro 
algarismos. Escreva um programa em \emph{Pascal} que leia este número  do
teclado e apresente o algarismo correspondente à casa do milhar.

\begin{center}
\begin{tabular}{|l|l|} \hline
Exemplo de entrada & Saída esperada \\ \hline
2569                & 2               \\ \hline
1000                & 1               \\ \hline
0350                & 0               \\ \hline
\end{tabular}
\end{center}

\item Você é um vendedor de carros é só aceita pagamentos à vista. As vezes 
é necessário ter que dar troco, mas seus clientes não gostam de notas miúdas. i
Para agradá-los você deve criar um programa em \emph{Pascal} que recebe o valor
do troco que deve ser dado ao cliente e retorna o número de notas de R\$100 
necessárias para esse troco.

\begin{center}
\begin{tabular}{|l|l|} \hline
Exemplo de entrada & Saída esperada \\ \hline
500                & 5               \\ \hline
360                & 3               \\ \hline
958                & 9               \\ \hline
\end{tabular}
\end{center}

\item Certo dia o professor de Johann Friederich Carl Gauss (aos 10 anos de 
idade) mandou que os alunos somassem os números de 1 a 100. Imediatamente 
Gauss achou a resposta – 5050 – aparentemente sem a soma de um em um. 
Supõe-se que já aí, Gauss, houvesse descoberto a fórmula de uma soma de uma 
progressão aritmética.

Agora você, com o auxílio dos conceitos de algoritmos e da linguagem 
\emph{Pascal} deve construir um programa que realize a soma de uma P.A. 
de $n$ termos, dado o primeiro termo $a1$ e o último termo $an$.
A impressão do resultado deve ser formatada com duas casas na direita.

\begin{center}
\begin{tabular}{|l|l|l|l|} \hline
\multicolumn{3}{|c|}{Exemplo de entrada} & Saída esperada \\ \hline
$n$ & $a_1$ & $a_n$   & $soma$               \\ \hline
100 & 1 & 100         & 5050.00                \\ \hline
10 & 1 & 10          & 55.00             \\ \hline
50 & 30 & 100         & 3250.00          \\ \hline
\end{tabular}
\end{center}

\item A sequência $A, B, C, \ldots$ determina uma Progressão Aritmética (P.A.). 
O termo médio ($B$) de uma P.A. é determinado pela média aritmética de seus 
termos, sucessor ($C$) e antecessor ($A$). Com base neste enunciado construa 
um programa em \emph{Pascal} que calcule e imprima o termo médio ($B$) 
através de $A$ e $C$, que devem ser lidos do teclado.

\[
B = \frac{A+B}{2}.
\]

\begin{center}
\begin{tabular}{|l|l|l|} \hline
\multicolumn{2}{|c|}{Exemplo de entrada} & Saída esperada \\ \hline
$A$ & $C$    & $B$               \\ \hline
1 & 3        & 2.00                \\ \hline
2 & 2        & 2.00             \\ \hline
100 & 500    & 300.00          \\ \hline
\end{tabular}
\end{center}

\item A sequência $A, B, C, \ldots$ determina uma Progressão Geométrica (P.G.), 
o termo médio ($B$) de uma P.G. é determinado pela média geométrica de seus 
termos, sucessor ($C$) e antecessor ($A$). Com base neste enunciado escreva 
um programa em \emph{Pascal} que calcule e imprima o termo médio ($B$) 
através de $A$, $C$, que devem ser lidos do teclado.

\begin{center}
\begin{tabular}{|l|l|l|} \hline
\multicolumn{2}{|c|}{Exemplo de entrada} & Saída esperada \\ \hline
$A$ & $C$    & $B$               \\ \hline
1 & 3        & 1.73                \\ \hline
10 & 100        & 31.62             \\ \hline
90 & 80    & 84.85          \\ \hline
\end{tabular}
\end{center}

\item O produto de uma série de termos de uma Progressão Geométrica (P.G.) 
pode ser calculado pela fórmula abaixo:

\[
P = a_1^n \times q^{\frac{n(n-1)}{2}}.
\]

Agora, escreva um programa em \emph{Pascal} para determinar o produto dos 
$n$ primeiros termos de uma P.G de razão $q$. Seu programa deverá ler
$a_1, q, n$ do teclado e imprimir $P$.
(ATENÇÃO PARA O TIPO DE VARIÁVEL!)

\begin{center}
\begin{tabular}{|l|l|l|l|} \hline
\multicolumn{3}{|c|}{Exemplo de entrada} & Saída esperada \\ \hline
$a_1$ & $q$ & $n$   & $P$               \\ \hline
5 & 1 & 10         & 9765625.00                \\ \hline
1 & 1 & 10          & 1.00             \\ \hline
2 & 2 & 5         & 32768.00          \\ \hline
\end{tabular}
\end{center}

\item A soma dos termos de uma Progressão Geométrica (P.G.) finita pode ser 
calculada pela fórmula abaixo:

\[
S_n = \frac{a_1 (q^n - 1)}{q - 1}
\]

Agora, escreva um programa em \emph{Pascal} para determinar a soma dos $n$ 
termos de uma P.G de razão $q$, iniciando no termo $a_1$. Seu programa
deverá ler $a_1, q, n$ do teclado e imprimir $S_n$.

\begin{center}
\begin{tabular}{|l|l|l|l|} \hline
\multicolumn{3}{|c|}{Exemplo de entrada} & Saída esperada \\ \hline
$a_1$ & $q$ & $n$   & $S_n$               \\ \hline
2 & 3 & 6         & 728.00                \\ \hline
0 & 5 & 10          & 0.00             \\ \hline
150 & 30 & 2         & 4650.00          \\ \hline
\end{tabular}
\end{center}

\item Criar um programa em \emph{Pascal} para calcular e imprimir o valor do 
volume de uma lata de óleo, utilizando a fórmula:

\[
V = 3.14159 \times r^2 \times h,
\]

onde $V$ é o volume, $r$ é o raio e $h$ é a altura. Seu programa
deve ler $r, h$ do teclado e imprimir $V$.

\begin{center}
\begin{tabular}{|l|l|l|} \hline
\multicolumn{2}{|c|}{Exemplo de entrada} & Saída esperada \\ \hline
$r$ & $h$   & $V$               \\ \hline
5 & 100          & 7853.98                \\ \hline
25 & 25.5           & 69704.03             \\ \hline
10 & 50.9         & 15990.69          \\ \hline
\end{tabular}
\end{center}

\item Fazer um programa em \emph{Pascal} que efetue o cálculo do salário 
líquido de um professor. Os dados fornecidos serão: valor da hora aula, 
número de aulas dadas no mês e percentual de desconto do INSS.

\begin{center}
\begin{tabular}{|l|l|l|l|} \hline
\multicolumn{3}{|c|}{Exemplo de entrada} & Saída esperada \\ \hline
valor hora aula & número de aulas & percentual INSS   & Salário bruto               \\ \hline
6.25 & 160 & 1.3         & 987.00                \\ \hline
20.5 & 240 & 1.7          & 4836.36             \\ \hline
13.9 & 200 & 6.48         & 2599.86          \\ \hline
\end{tabular}
\end{center}

\item Em épocas de pouco dinheiro, os comerciantes estão procurando aumentar 
suas vendas oferecendo desconto aos clientes. Escreva um programa em 
\emph{Pascal}  que possa entrar com o valor de um produto e imprima o novo 
valor tendo em vista que o desconto foi de 9\%. Além disso, imprima o valor 
do desconto.

\begin{center}
\begin{tabular}{|l|l|l|} \hline
\multicolumn{2}{|c|}{Exemplo de entrada} & Saída esperada \\ \hline
valor do produto (R\$) & novo valor (R\$) & valor do desconto (R\$) \\ \hline
100 & 91.00          & 9.00                \\ \hline
1500 & 1365.00       & 135.00             \\ \hline
60000 & 54600.00     & 5400.00          \\ \hline
\end{tabular}
\end{center}

\item Todo restaurante, embora por lei não possa obrigar o cliente a pagar, 
cobra 10\% de comissão para o garçom. Crie um programa em \emph{Pascal} que 
leia o valor gasto com despesas realizadas em um restaurante e imprima o i
valor da gorjeta e o valor total com a gorjeta.

\begin{center}
\begin{tabular}{|l|l|} \hline
Exemplo de entrada & Saída esperada \\ \hline
75                & 82.50              \\ \hline
125               & 137.50               \\ \hline
350.87            & 385.96               \\ \hline
\end{tabular}
\end{center}

\item Criar um programa em \emph{Pascal} que leia um valor de hora 
(hora:minutos), calcule e imprima o total de minutos se passaram desde o 
início do dia (0:00h). A entrada será dada por dois números separados
na mesma linha, o primeiro número representa as horas e o segundo os minutos.

\begin{center}
\begin{tabular}{|l|l|l|} \hline
\multicolumn{2}{|c|}{Exemplo de entrada} & Saída esperada \\ \hline
hora & minuto & total de minutos \\ \hline
1 & 0          & 60                \\ \hline
14 & 30       & 870             \\ \hline
23 & 55     & 1435          \\ \hline
\end{tabular}
\end{center}

\item Criar um programa em \emph{Pascal} que leia o valor de um depósito e o 
valor da taxa de juros. Calcular e imprimir o valor do rendimento do depósito 
e o valor total depois do rendimento.

\begin{center}
\begin{tabular}{|l|l|l|l|} \hline
\multicolumn{2}{|c|}{Exemplo de entrada} & \multicolumn{2}{|c|}{Saída esperada} \\ \hline
depósito & taxa de juros & rendimento & total \\ \hline
200 & 0.5 & 1.00   & 201.00            \\ \hline
1050 & 1 & 10.5    & 1060.5           \\ \hline
2300.38 & 0.06 & 1.38   & 2301.38          \\ \hline
\end{tabular}
\end{center}

\item Para vários tributos, a base de cálculo é o salário mínimo. Fazer um 
programa em \emph{Pascal} que leia o valor do salário mínimo e o valor do 
salário de uma pessoa. Calcular e imprimir quantos salários mínimos essa 
pessoa ganha.

\begin{center}
\begin{tabular}{|l|l|l|} \hline
\multicolumn{2}{|c|}{Exemplo de entrada} & Saída esperada \\ \hline
salário mínimo (R\$) & salário (R\$) & salário em salários mínimos (R\$) \\ \hline
450.89 & 2700.00 & 5.99            \\ \hline
1000.00& 1000.00 & 1.00           \\ \hline
897.50& 7800.00 & 8.69          \\ \hline
\end{tabular}
\end{center}

\item Criar um programa em \emph{Pascal} que efetue o cálculo da quantidade de 
litros de combustível gastos em uma viagem, sabendo-se que o carro faz 12 km 
com um litro. Deverão ser fornecidos o tempo gasto na viagem e a velocidade 
média.  $Distancia = Tempo \times Velocidade$.  $Litros = Distancia / 12$.
O algoritmo deverá apresentar os valores da Distância percorrida e a 
quantidade de Litros utilizados na viagem.

\begin{center}
\begin{tabular}{|l|l|l|l|} \hline
\multicolumn{3}{|c|}{Exemplo de entrada} & Saída esperada \\ \hline
tempo gasto & velocidade média & distância percorrida & litros \\ \hline
60 & 100  & 6000.00 & 500.00             \\ \hline
1440 & 80 & 115200.00 & 9600.00           \\ \hline
5 & 90 & 450.00 & 37.50        \\ \hline
\end{tabular}
\end{center}

\item Um vendedor de uma loja de sapatos recebe como pagamento 20\% de comissão 
sobre as vendas do mês e R\$5.00 por cada par de sapatos vendidos. Faça
 um programa em \emph{Pascal} que, dado o 
total de vendas do mês e o número de sapatos vendidos, imprima quanto será o 
salário daquele mês do vendedor.

\begin{center}
\begin{tabular}{|l|l|l|} \hline
\multicolumn{2}{|c|}{Exemplo de entrada} & Saída esperada \\ \hline
total de vendas (R\$) & sapatos vendidos & salário (R\$) \\ \hline
50000.00 & 100 & 10500.00            \\ \hline
2000.00 & 30 & 550.00           \\ \hline
1000000.00 & 500 & 202500.00          \\ \hline
\end{tabular}
\end{center}

\item Você está endividado e quer administrar melhor sua vida financeira. 
Para isso, crie um programa em \emph{Pascal} que recebe o valor de uma dívida 
e o juros mensal, então calcule e imprima o valor da dívida no mês seguinte.

\begin{center}
\begin{tabular}{|l|l|l|} \hline
\multicolumn{2}{|c|}{Exemplo de entrada} & Saída esperada \\ \hline
valor da dívida (R\$) & juros/mês & dívida (R\$) \\ \hline
100.00 & 10  & 110.00            \\ \hline
1500.00 & 3 &  1545.00          \\ \hline
10000.00 & 0.5 & 10050.00          \\ \hline
\end{tabular}
\end{center}

\item Antes de o racionamento de energia ser decretado, quase ninguém falava 
em quilowatts; mas, agora, todos incorporaram essa palavra em seu vocabulário. 
Sabendo-se que 100 quilowatts de energia custa um sétimo do salário mínimo, 
fazer um programa em \emph{Pascal} que receba o valor do salário mínimo e a 
quantidade de quilowatts gasta por uma residência e imprima:
\begin{itemize}
\item o valor em reais de cada quilowatt;
\item o valor em reais a ser pago;
\end{itemize}

\begin{center}
\begin{tabular}{|l|l|l|l|} \hline
\multicolumn{2}{|c|}{Exemplo de entrada} & \multicolumn{2}{|c|}{Saída esperada} \\ \hline
salário mínimo (R\$) & quilowatts & valor do quilowatt (R\$) & valor pago (R\$) \\ \hline
750.00 & 200 & 1.07 & 214.29           \\ \hline
935.00 & 150 & 1.34 & 200.36         \\ \hline
1200.00 & 250 & 1.71 & 428.57        \\ \hline
\end{tabular}
\end{center}


\end{enumerate}

\subsection{Programas com cálculos e desvios condicionais}
\begin{enumerate}

\item Escreva um programa em \emph{Pascal} que leia um número e o imprima 
caso ele seja maior que 20.

\begin{center}
\begin{tabular}{|l|l|} \hline
Exemplo Entrada & Saída esperada \\ \hline
30.56           & 30.56          \\ \hline
20              &                \\ \hline
20.05           & 20.05          \\ \hline
\end{tabular}
\end{center}

\item Escreva um programa em \emph{Pascal} que leia dois valores numéricos 
inteiros e efetue a adição; se o resultado for maior que 10, imprima o 
primeiro valor. Caso contrário, imprima o segundo.

\begin{center}
\begin{tabular}{|l|l|} \hline
Exemplo Entrada & Saída esperada \\ \hline
7                &                \\
4                &   7             \\ \hline
7                &                \\
2                &   2             \\ \hline
3                &                \\
7                &   7             \\ \hline
\end{tabular}
\end{center}

\item Escreva um programa em \emph{Pascal} que imprima se um dado número $N$ 
inteiro (recebido através do teclado) é PAR ou ÍMPAR.

\begin{center}
\begin{tabular}{|l|l|} \hline
Exemplo Entrada & Saída esperada \\ \hline
5                & impar               \\ \hline
3                & impar               \\ \hline
2                & par               \\ \hline
\end{tabular}
\end{center}

\item Escreva um programa em \emph{Pascal} para determinar se um dado número 
$N$ (recebido através do teclado) é POSITIVO, NEGATIVO ou NULO.

\begin{center}
\begin{tabular}{|l|l|} \hline
Exemplo Entrada & Saída esperada \\ \hline
5               & positivo               \\ \hline
-3              & negativo               \\ \hline
0               & nulo               \\ \hline
\end{tabular}
\end{center}

\item Escreva um programa em \emph{Pascal} que leia dois números e efetue a 
adição. Caso o valor somado seja maior que 20, este deverá ser apresentado 
somando-se a ele mais 8; caso o valor somado seja menor ou igual a 20, 
este deverá ser apresentado subtraindo-se 5.

\begin{center}
\begin{tabular}{|l|l|} \hline
Exemplo Entrada & Saída esperada \\ \hline
13.14                &                \\
5                &    13.14            \\ \hline
-3                &                \\
-4                &   -12.00             \\ \hline
16                &                \\
5                &    20.00            \\ \hline
\end{tabular}
\end{center}

\item Escreva um programa em \emph{Pascal} que imprima qual o menor valor de 
dois números $A$ e $B$, lidos através do teclado.

\begin{center}
\begin{tabular}{|l|l|} \hline
Exemplo Entrada & Saída esperada \\ \hline
5.35                &                \\
4                & 4.00               \\ \hline
-3                &                \\
1                & -3.00               \\ \hline
6                &                \\ 
15                & 6.00               \\ \hline
\end{tabular}
\end{center}

\item Escreva um programa em \emph{Pascal} para determinar se um número 
inteiro $A$ é divisível por um outro número inteiro $B$. Esses valores 
devem ser fornecidos pelo usuário.

\begin{center}
\begin{tabular}{|l|l|} \hline
Exemplo Entrada & Saída esperada \\ \hline
5                &                \\
10               & nao               \\ \hline
4                &                \\ 
2                & sim               \\ \hline
7                &                \\ 
21               & nao              \\ \hline
\end{tabular}
\end{center}

\item Escreva um programa em \emph{Pascal} que leia um número inteiro e 
informe se ele é ou não divisível por 5.

\begin{center}
\begin{tabular}{|l|l|} \hline
Exemplo Entrada & Saída esperada \\ \hline
5                & sim               \\ \hline
-5                & sim               \\ \hline
3                &  nao              \\ \hline
\end{tabular}
\end{center}

\item Escreva um programa em \emph{Pascal} que receba um número inteiro e 
imprima se este é múltiplo de 3.

\begin{center}
\begin{tabular}{|l|l|} \hline
Exemplo Entrada & Saída esperada \\ \hline
5                & nao               \\ \hline
-3                & sim               \\ \hline
15                & sim               \\ \hline
\end{tabular}
\end{center}

\item Escreva um programa em \emph{Pascal} que leia um número e imprima a 
raiz quadrada do número caso ele seja positivo ou igual a zero e o quadrado 
do número caso ele seja negativo.

\begin{center}
\begin{tabular}{|l|l|} \hline
Exemplo Entrada & Saída esperada \\ \hline
0                & 0.00               \\ \hline
4                & 2.00               \\ \hline
-5               & 25.00               \\ \hline
\end{tabular}
\end{center}

\item Escreva um programa em \emph{Pascal} que leia um número e informe se ele 
é divisível por 3 e por 7.

\begin{center}
\begin{tabular}{|l|l|} \hline
Exemplo Entrada & Saída esperada \\ \hline
21               & sim               \\ \hline
7                & nao               \\ \hline
3                & nao               \\ \hline
-42              & sim               \\ \hline
\end{tabular}
\end{center}

\item A prefeitura de Contagem abriu uma linha de crédito para os funcionários 
estatutários. O valor máximo da prestação não poderá ultrapassar 30\% do 
salário bruto. Fazer um programa em \emph{Pascal} que permita entrar com o 
salário bruto e o valor da prestação, e informar se o empréstimo pode ou não 
ser concedido.

\begin{center}
\begin{tabular}{|l|l|} \hline
Exemplo Entrada & Saída esperada \\ \hline
500                &                \\
200                & nao               \\ \hline
1000.50                &                \\
250.10                & sim               \\ \hline
1000                &                \\
300                & sim               \\ \hline
\end{tabular}
\end{center}

\item Escreva um programa em \emph{Pascal} que dado quatro valores, 
$A$, $B$, $C$ e $D$, o programa imprima o menor e o maior valor.

\begin{center}
\begin{tabular}{|l|l|} \hline
Exemplo Entrada & Saída esperada \\ \hline
1                &                \\ 
2                &                \\ 
3                &                \\ 
4                & 1.00 4.00               \\ \hline
-3                &                \\ 
0               &                \\ 
1               &                \\ 
1               &  -3.00 1.00              \\ \hline
3.5                &                \\
3.7                &                \\
4.0                &                \\
5.5                & 3.50 5.50               \\ \hline
\end{tabular}
\end{center}

\item Dados três valores $A$, $B$ e $C$, construa um programa em \emph{Pascal},
 que imprima os valores de forma ascendente (do menor para o maior).

\begin{center}
\begin{tabular}{|l|l|} \hline
Exemplo Entrada & Saída esperada \\ \hline
1 2 1.5                & 1.00 1.50 2.00               \\ \hline
-3 -4 -5              & -5.00 -4.00 -3.00               \\ \hline
6 5 4                & 4.00 5.00 6.00               \\ \hline
\end{tabular}
\end{center}

\item Dados três valores $A$, $B$ e $C$, construa um programa em \emph{Pascal},
 que imprima os valores de forma descendente (do maior para o menor).

\begin{center}
\begin{tabular}{|l|l|} \hline
Exemplo Entrada & Saída esperada \\ \hline
1 2 1.5                & 2.00 1.50 1.00               \\ \hline
-5 -4 -3                & -3.00 -4.00 -5.00                \\ \hline
5 6 4                & 6.00 5.00 4.00               \\ \hline
\end{tabular}
\end{center}

\item Escreva um programa em \emph{Pascal} que leia dois números e imprimir 
o quadrado do menor número e raiz quadrada do maior número, se for possível.

\begin{center}
\begin{tabular}{|l|l|} \hline
Exemplo Entrada & Saída esperada \\ \hline
4                &                \\
3                & 9.00 2.00               \\ \hline
4.35                &                \\
3.50                & 12.25 2.09              \\ \hline
-4                &                \\ 
-16                & 256.00               \\ \hline
\end{tabular}
\end{center}

\item Escreva um programa em \emph{Pascal} que indique se um número digitado 
está compreendido entre 20 e 90 ou não (20 e 90 não estão na faixa de valores).

\begin{center}
\begin{tabular}{|l|l|} \hline
Exemplo Entrada & Saída esperada \\ \hline
50.50                & sim               \\ \hline
20                & nao               \\ \hline
90                & nao               \\ \hline
\end{tabular}
\end{center}

\item Escreva um programa em \emph{Pascal} que leia um número inteiro e 
informe se ele é divisível por 10, por 5 ou por 2 ou se não é divisível 
por nenhum deles.

\begin{center}
\begin{tabular}{|l|l|} \hline
Exemplo Entrada & Saída esperada \\ \hline
10                &  10 5 2              \\ \hline
5                &  5              \\ \hline
4                &  2              \\ \hline
7               &  nenhum              \\ \hline
\end{tabular}
\end{center}

\item Escreva um programa em \emph{Pascal} que leia um número e imprima se 
ele é igual a 5, a 200, a 400, se está no intervalo entre 500 e 1000, 
inclusive, ou se está fora dos escopos anteriores.

\begin{center}
\begin{tabular}{|l|l|} \hline
Exemplo Entrada & Saída esperada \\ \hline
5                & igual a 5               \\ \hline
200                & igual a 200               \\ \hline
400                & igual a 400               \\ \hline
750.50                & intervalo entre 500 e 1000               \\ \hline
1000                & intervalo entre 500 e 1000               \\ \hline
1500                & fora dos escopos               \\ \hline
\end{tabular}
\end{center}

\item A CEF concederá um crédito especial com juros de 2\% aos seus clientes de 
acordo com o saldo médio no último ano. Fazer um programa em \emph{Pascal} que 
leia o saldo médio de um cliente e calcule o valor do crédito de acordo com a 
tabela a seguir. Imprimir uma mensagem informando o valor de crédito.

\begin{tabular}{|l|l|}\hline
De 0 a 500 & nenhum credito \\ \hline
De 501 a 1000 & 30\% do valor do saldo medio \\ \hline
De 1001 a 3000 & 40\% do valor do saldo medio \\ \hline
Acima de 3001 &  50\% do valor do saldo medio\\ \hline
\end{tabular}

\begin{center}
\begin{tabular}{|l|l|} \hline
Exemplo Entrada & Saída esperada \\ \hline
300.50                & 0.00                \\ \hline
571                & 171.30               \\ \hline
1492.35                & 596.94               \\ \hline
3001.20               &  1500.60               \\ \hline
\end{tabular}
\end{center}

\item Escreva um programa em \emph{Pascal} que dada a idade de uma pessoa, 
determine sua classificação segundo a seguinte tabela:
\begin{itemize}
\item Maior de idade;
\item Menor de idade;
\item Pessoa idosa (idade superior ou igual a 65 anos).
\end{itemize}

\begin{center}
\begin{tabular}{|l|l|} \hline
Exemplo Entrada & Saída esperada \\ \hline
18                & maior               \\ \hline
15                & menor               \\ \hline
65                & idosa               \\ \hline
\end{tabular}
\end{center}

\item Escreva um programa em \emph{Pascal} que leia a idade de uma pessoa 
e informe a sua classe eleitoral:
\begin{itemize}
\item não eleitor (abaixo de 16 anos);
\item eleitor obrigatório (entre a faixa de 18 e menor de 65 anos);
\item eleitor facultativo (de 16 até 18 anos e maior de 65 anos, inclusive).
\end{itemize}

\begin{center}
\begin{tabular}{|l|l|} \hline
Exemplo Entrada & Saída esperada \\ \hline
15                & nao eleitor               \\ \hline
16                & facultativo               \\ \hline
17                & facultativo               \\ \hline
18                & obrigatorio               \\ \hline
19                & obrigatorio               \\ \hline
\end{tabular}
\end{center}

\item A confederação brasileira de natação irá promover eliminatórias para o próximo mundial. Fazer um programa em \emph{Pascal} que receba a idade de um nadador e imprima a sua categoria segundo a tabela a seguir:

\begin{tabular}{|l|l|} \hline
Infantil A & 5 -- 7 anos \\ \hline
Infantil B & 8 -- 10 anos  \\ \hline
Juvenil A & 11 -- 13 anos \\ \hline
Juvenil B & 14 -- 17 anos \\ \hline
Sênior &  Maiores de 18 anos \\ \hline
\end{tabular}

\begin{center}
\begin{tabular}{|l|l|} \hline
Exemplo Entrada & Saída esperada \\ \hline
4 & INVÁLIDO \\ \hline
7 & Infantil A \\ \hline
8 & Infantil B \\ \hline
10 & Infantil B \\ \hline
11 & Juvenil A \\ \hline
13 & Juvenil A \\ \hline
14 & Juvenil B \\ \hline
17 & Juvenil B \\ \hline
18 & Sênior \\ \hline
\end{tabular}
\end{center}

\item Dados três valores $A$, $B$ e $C$, escreva um programa em \emph{Pascal} 
para verificar se estes valores podem ser valores dos lados de um triângulo, 
se é um triângulo ESCALENO, um triângulo EQUILÁTERO ou um triângulo ISÓSCELES. 
Caso não sejam válidos, imprimir: ``INVALIDO''.

\begin{center}
\begin{tabular}{|l|l|} \hline
Exemplo Entrada & Saída esperada \\ \hline
5 & \\
5 & \\
5 & EQUILATERO \\ \hline
7 & \\
7 & \\
5 & ISOSCELES \\ \hline
3 & \\
4 & \\
5 & ESCALENO \\ \hline
5 & \\
4 & \\
15 & INVALIDO \\ \hline
\end{tabular}
\end{center}

\item Escreva um programa em \emph{Pascal} que leia as duas notas bimestrais 
de um aluno e determine a média das notas semestral. Através da média 
calculada o algoritmo deve imprimir a seguinte mensagem: APROVADO, 
REPROVADO ou em EXAME (a média é 7 para Aprovação, menor que 3 para 
Reprovação e as demais em Exame).

\begin{center}
\begin{tabular}{|l|l|} \hline
Exemplo Entrada & Saída esperada \\ \hline
3.1 & \\
2.5 & REPROVADO \\ \hline
3 & \\
3 & EXAME \\ \hline
10 & \\
0 & EXAME \\ \hline
6 & \\
8 & APROVADO \\ \hline
10 & \\
10 & APROVADO \\ \hline
\end{tabular}
\end{center}

\item Depois da liberação do governo para as mensalidades dos planos de saúde, as pessoas começaram a fazer pesquisas para descobrir um bom plano, não muito caro. Um vendedor de um plano de saúde apresentou a tabela a seguir. Escreva um programa em \emph{Pascal} que entre com a idade de uma pessoa e imprima o valor que ela deverá pagar, segundo a seguinte tabela:

\begin{tabular}{|l|l|}
Até 10 anos & R\$ 30.00 \\ \hline
Acima de 10 até 29 anos & R\$ 60.00 \\ \hline
Acima de 29 até 45 anos & R\$ 120.00 \\ \hline
Acima de 45 até 59 anos & R\$ 150.00 \\ \hline
Acima de 59 até 65 anos & R\$ 250.00 \\ \hline
Maior do que 65 anos & R\$ 400.00 \\ \hline
\end{tabular}

\begin{center}
\begin{tabular}{|l|l|} \hline
10 & R\$ 30,00 \\ \hline
29 & R\$ 60,00 \\ \hline
45 & R\$ 120,00 \\ \hline
59 & R\$ 150,00 \\ \hline
65 & R\$ 250,00 \\ \hline
66 & R\$ 400,00 \\ \hline
\end{tabular}
\end{center}

\item Escreva um programa em \emph{Pascal} que leia o um número inteiro entre 1 e 7 e escreva o dia da semana correspondente. Caso o usuário digite um número fora desse intervalo, deverá aparecer a seguinte  mensagem: INEXISTENTE

\begin{center}
\begin{tabular}{|l|l|} \hline
Exemplo Entrada & Saída esperada \\ \hline
0 & INEXISTENTE \\ \hline
1 & DOMINGO \\ \hline
2 & SEGUNDA \\ \hline
3 & TERÇA \\ \hline
4 & QUARTA \\ \hline
5 & QUINTA \\ \hline
6 & SEXTA \\ \hline
7 & SÁBADO \\ \hline
8 & INEXISTENTE \\ \hline
\end{tabular}
\end{center}

\end{enumerate}

