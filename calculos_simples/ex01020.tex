\item A soma dos termos de uma Progressão Geométrica (P.G.) finita pode ser 
calculada pela fórmula abaixo:

\[
S_n = \frac{a_1 (q^n - 1)}{q - 1}
\]

Agora, escreva um programa em \emph{Pascal} para determinar a soma dos $n$ 
termos de uma P.G de razão $q$, iniciando no termo $a_1$. Seu programa
deverá ler $a_1, q, n$ do teclado e imprimir $S_n$.

\begin{center}
\begin{tabular}{|l|l|l|l|} \hline
\multicolumn{3}{|c|}{Exemplo de entrada} & Saída esperada \\ \hline
$a_1$ & $q$ & $n$   & $S_n$               \\ \hline
2 & 3 & 6         & 728.00                \\ \hline
0 & 5 & 10          & 0.00             \\ \hline
150 & 30 & 2         & 4650.00          \\ \hline
\end{tabular}
\end{center}