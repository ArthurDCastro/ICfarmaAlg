
\item Escreva um programa em \emph{Pascal} que leia um número inteiro e 
imprima o seu sucessor e seu antecessor, na mesma linha.

\begin{center}
\begin{tabular}{|l|l|} \hline
Exemplo de entrada & Saída esperada \\ \hline
1               & 2 0 \\ \hline
100             & 101 99 \\ \hline
-3              & -2 -4 \\ \hline
\end{tabular}
\end{center}

\item Escreva um programa em \emph{Pascal} que leia dois números inteiros e 
imprima o resultado da soma destes dois valores. Antes do resultado, deve ser 
impressa a seguinte mensagem ``SOMA= ''.


\begin{center}
\begin{tabular}{|l|l|} \hline
Exemplo de entrada & Saída esperada \\ \hline
1 2             & SOMA= 3 \\ \hline
100 -50         & SOMA= 50 \\ \hline
-5 -40          & SOMA= -45 \\ \hline
\end{tabular}
\end{center}

\item Escreva um programa em \emph{Pascal} que leia dois números reais, um 
será o valor de um produto e outro o valor de desconto que esse produto está 
recebendo. Imprima quantos reais o produto custa na promoção.

\begin{center}
\begin{tabular}{|l|l|l|} \hline
\multicolumn{2}{|c|}{Exemplo de entrada} & Saída esperada \\ \hline
Valor original & Desconto & Valor na promoção \\ \hline
500.00         & 50.00  & 450.00 \\ \hline
10500.00       & 500.00 & 10000.00\\ \hline
90.00          & 0.80   & 89.20 \\ \hline
\end{tabular}
\end{center}

\item Escreva um programa em \emph{Pascal} que leia dois números reais e 
imprima a média aritmética entre esses dois valores.

\begin{center}
\begin{tabular}{|l|l|} \hline
Exemplo de entrada & Saída esperada \\ \hline
1.2 2.3         & 1.75 \\ \hline
750 1500        & 1125.00  \\ \hline
8900 12300      & 10600.00 \\ \hline
\end{tabular}
\end{center}

\item Escreva um programa em \emph{Pascal} que leia um número real e imprima a 
terça parte deste número.

\begin{center}
\begin{tabular}{|l|l|} \hline
Exemplo de entrada & Saída esperada \\ \hline
3               & 1.00 \\ \hline
10              & 3.33  \\ \hline
90              & 30.00 \\ \hline
\end{tabular}
\end{center}

\item Uma P.A. (progressão aritmética) fica determinada pela sua razão ($r$) 
e pelo primeiro termo ($a_1$). Escreva um programa em \emph{Pascal} que seja 
capaz de determinar o enésimo ($n$) termo ($a_n$) de uma P.A., dado a razão 
($r$) e o primeiro termo ($a_1$). Seu programa deve ler $n, r, a_1$ do teclado
e imprimir $a_n$.

\[
a_n = a_1 + (n-1)\times r.
\]

\begin{center}
\begin{tabular}{|l|l|l|l|} \hline
\multicolumn{3}{|c|}{Exemplo de entrada} & Saída esperada \\ \hline
$n$ & $r$ & $a_1$   & $a_n$               \\ \hline
8 & 1 & 3       & 10                \\ \hline
100 & 10 & 1    & 991                \\ \hline
5 & -2 & 0      & -98                \\ \hline
\end{tabular}
\end{center}

\item Dada a razão ($r$) de uma P.A. (progressão aritmética) e um termo 
qualquer, $k$ ($a_k$). Escreva um programa em \emph{Pascal} para calcular 
o enésimo termo $n$ ($a_n$). Seu programa deve ler $k, a_k, r, n$ do teclado
e imprimir $a_n$.

\[
a_n = a_k + (n-r) \times r
\]

\begin{center}
\begin{tabular}{|l|l|l|l|l|} \hline
\multicolumn{4}{|c|}{Exemplo de entrada} & Saída esperada \\ \hline
$k$ & $a_k$ & $r$ & n   & $a_n$             \\ \hline
1 & 5 & 2 & 10          & 23                \\ \hline
10 & 20 & 2 & 5         & 10                \\ \hline
100 & 500 & 20 & 90     & 300               \\ \hline
\end{tabular}
\end{center}

\item Uma P.G. (progressão geométrica) fica determinada pela sua razão ($q)$ 
e pelo primeiro termo ($a_1$). Escreva um programa em \emph{Pascal} que seja 
capaz de determinar o enésimo $n$ termo ($a_n$) de uma P.G., dado a razão ($q$) 
e o primeiro termo ($a_1$). Seu programa deve ler $a_1, q, n$ do teclado
e imprimir $a_n$.

\[
a_n = a_1 \times q^{(n-1)}.
\]

\begin{center}
\begin{tabular}{|l|l|l|l|} \hline
\multicolumn{3}{|c|}{Exemplo de entrada} & Saída esperada \\ \hline
$a_1$ & $q$ & $n$   & $a_n$               \\ \hline
1 & 1 & 100         & 1.00                \\ \hline
2 & 2 & 10          & 1024.00             \\ \hline
5 & 3 & 2           & 15.00               \\ \hline
\end{tabular}
\end{center}

\item Dada a razão ($q$) de uma P.G. (progressão geométrica) e um termo 
qualquer, $k$ ($a_k$). Escreva um programa em Pascal para calcular o enésimo 
termo $n$ ($an$). Seu programa deve ser $k, a_k, q, n$ do teclado e imprimir
$a_n$.

\begin{center}
\begin{tabular}{|l|l|l|l|l|} \hline
\multicolumn{4}{|c|}{Exemplo de entrada} & Saída esperada \\ \hline
$k$ & $a_k$ & $q$ & $n$  & $a_n$               \\ \hline
2 & 2 & 1 & 1        & 2                \\ \hline
1 & 5 & 2 & 10       & 2560.00             \\ \hline
2 & 100 & 10 & 20    & 100000000000000000000.00               \\ \hline
\end{tabular}
\end{center}

\item Uma P.G. (progressão geométrica) fica determinada pela sua razão ($q$) 
e pelo primeiro termo ($a_1$). Escreva um programa em \emph{Pascal} que 
seja capaz de determinar o enésimo termo ($a_n$) de uma P.G., dado a razão 
($q$) e o primeiro termo ($a_1$). Seu programa deve ler $a_1, q, n$ do 
teclado e imprimir $a_n$.

\begin{center}
\begin{tabular}{|l|l|l|l|} \hline
\multicolumn{3}{|c|}{Exemplo de entrada} & Saída esperada \\ \hline
$a_1$ & $q$ & $n$   & $a_n$               \\ \hline
1 & 1 & 100         & 1.00                \\ \hline
2 & 2 & 10          & 1024.00             \\ \hline
10 & 2 & 20         & 5242880.00          \\ \hline
\end{tabular}
\end{center}

\item Considere que o número de uma placa de veículo é composto por quatro 
algarismos. Escreva um programa em \emph{Pascal} que leia este número  do
teclado e apresente o algarismo correspondente à casa das unidades.

\begin{center}
\begin{tabular}{|l|l|} \hline
Exemplo de entrada & Saída esperada \\ \hline
2569                & 9               \\ \hline
1000                & 0               \\ \hline
1305                & 5               \\ \hline
\end{tabular}
\end{center}

\item Considere que o número de uma placa de veículo é composto por quatro 
algarismos. Escreva um programa em \emph{Pascal} que leia este número  do
teclado e apresente o algarismo correspondente à casa das dezenas.

\begin{center}
\begin{tabular}{|l|l|} \hline
Exemplo de entrada & Saída esperada \\ \hline
2569                & 6               \\ \hline
1000                & 0               \\ \hline
1350                & 5               \\ \hline
\end{tabular}
\end{center}

\item Considere que o número de uma placa de veículo é composto por quatro 
algarismos. Escreva um programa em \emph{Pascal} que leia este número  do
teclado e apresente o algarismo correspondente à casa das centenas.

\begin{center}
\begin{tabular}{|l|l|} \hline
Exemplo de entrada & Saída esperada \\ \hline
2500                & 5               \\ \hline
2031                & 0               \\ \hline
6975                & 9               \\ \hline
\end{tabular}
\end{center}

\item Considere que o número de uma placa de veículo é composto por quatro 
algarismos. Escreva um programa em \emph{Pascal} que leia este número  do
teclado e apresente o algarismo correspondente à casa do milhar.

\begin{center}
\begin{tabular}{|l|l|} \hline
Exemplo de entrada & Saída esperada \\ \hline
2569                & 2               \\ \hline
1000                & 1               \\ \hline
0350                & 0               \\ \hline
\end{tabular}
\end{center}

\item Você é um vendedor de carros é só aceita pagamentos à vista. As vezes 
é necessário ter que dar troco, mas seus clientes não gostam de notas miúdas. i
Para agradá-los você deve criar um programa em \emph{Pascal} que recebe o valor
do troco que deve ser dado ao cliente e retorna o número de notas de R\$100 
necessárias para esse troco.

\begin{center}
\begin{tabular}{|l|l|} \hline
Exemplo de entrada & Saída esperada \\ \hline
500                & 5               \\ \hline
360                & 3               \\ \hline
958                & 9               \\ \hline
\end{tabular}
\end{center}

\item Certo dia o professor de Johann Friederich Carl Gauss (aos 10 anos de 
idade) mandou que os alunos somassem os números de 1 a 100. Imediatamente 
Gauss achou a resposta – 5050 – aparentemente sem a soma de um em um. 
Supõe-se que já aí, Gauss, houvesse descoberto a fórmula de uma soma de uma 
progressão aritmética.

Agora você, com o auxílio dos conceitos de algoritmos e da linguagem 
\emph{Pascal} deve construir um programa que realize a soma de uma P.A. 
de $n$ termos, dado o primeiro termo $a1$ e o último termo $an$.
A impressão do resultado deve ser formatada com duas casas na direita.

\begin{center}
\begin{tabular}{|l|l|l|l|} \hline
\multicolumn{3}{|c|}{Exemplo de entrada} & Saída esperada \\ \hline
$n$ & $a_1$ & $a_n$   & $soma$               \\ \hline
100 & 1 & 100         & 5050.00                \\ \hline
10 & 1 & 10          & 55.00             \\ \hline
50 & 30 & 100         & 3250.00          \\ \hline
\end{tabular}
\end{center}

\item A sequência $A, B, C, \ldots$ determina uma Progressão Aritmética (P.A.). 
O termo médio ($B$) de uma P.A. é determinado pela média aritmética de seus 
termos, sucessor ($C$) e antecessor ($A$). Com base neste enunciado construa 
um programa em \emph{Pascal} que calcule e imprima o termo médio ($B$) 
através de $A$ e $C$, que devem ser lidos do teclado.

\[
B = \frac{A+B}{2}.
\]

\begin{center}
\begin{tabular}{|l|l|l|} \hline
\multicolumn{2}{|c|}{Exemplo de entrada} & Saída esperada \\ \hline
$A$ & $C$    & $B$               \\ \hline
1 & 3        & 2.00                \\ \hline
2 & 2        & 2.00             \\ \hline
100 & 500    & 300.00          \\ \hline
\end{tabular}
\end{center}

\item A sequência $A, B, C, \ldots$ determina uma Progressão Geométrica (P.G.), 
o termo médio ($B$) de uma P.G. é determinado pela média geométrica de seus 
termos, sucessor ($C$) e antecessor ($A$). Com base neste enunciado escreva 
um programa em \emph{Pascal} que calcule e imprima o termo médio ($B$) 
através de $A$, $C$, que devem ser lidos do teclado.

\begin{center}
\begin{tabular}{|l|l|l|} \hline
\multicolumn{2}{|c|}{Exemplo de entrada} & Saída esperada \\ \hline
$A$ & $C$    & $B$               \\ \hline
1 & 3        & 1.73                \\ \hline
10 & 100        & 31.62             \\ \hline
90 & 80    & 84.85          \\ \hline
\end{tabular}
\end{center}

\item O produto de uma série de termos de uma Progressão Geométrica (P.G.) 
pode ser calculado pela fórmula abaixo:

\[
P = a_1^n \times q^{\frac{n(n-1)}{2}}.
\]

Agora, escreva um programa em \emph{Pascal} para determinar o produto dos 
$n$ primeiros termos de uma P.G de razão $q$. Seu programa deverá ler
$a_1, q, n$ do teclado e imprimir $P$.
(ATENÇÃO PARA O TIPO DE VARIÁVEL!)

\begin{center}
\begin{tabular}{|l|l|l|l|} \hline
\multicolumn{3}{|c|}{Exemplo de entrada} & Saída esperada \\ \hline
$a_1$ & $q$ & $n$   & $P$               \\ \hline
5 & 1 & 10         & 9765625.00                \\ \hline
1 & 1 & 10          & 1.00             \\ \hline
2 & 2 & 5         & 32768.00          \\ \hline
\end{tabular}
\end{center}

\item A soma dos termos de uma Progressão Geométrica (P.G.) finita pode ser 
calculada pela fórmula abaixo:

\[
S_n = \frac{a_1 (q^n - 1)}{q - 1}
\]

Agora, escreva um programa em \emph{Pascal} para determinar a soma dos $n$ 
termos de uma P.G de razão $q$, iniciando no termo $a_1$. Seu programa
deverá ler $a_1, q, n$ do teclado e imprimir $S_n$.

\begin{center}
\begin{tabular}{|l|l|l|l|} \hline
\multicolumn{3}{|c|}{Exemplo de entrada} & Saída esperada \\ \hline
$a_1$ & $q$ & $n$   & $S_n$               \\ \hline
2 & 3 & 6         & 728.00                \\ \hline
0 & 5 & 10          & 0.00             \\ \hline
150 & 30 & 2         & 4650.00          \\ \hline
\end{tabular}
\end{center}

\item Criar um programa em \emph{Pascal} para calcular e imprimir o valor do 
volume de uma lata de óleo, utilizando a fórmula:

\[
V = 3.14159 \times r^2 \times h,
\]

onde $V$ é o volume, $r$ é o raio e $h$ é a altura. Seu programa
deve ler $r, h$ do teclado e imprimir $V$.

\begin{center}
\begin{tabular}{|l|l|l|} \hline
\multicolumn{2}{|c|}{Exemplo de entrada} & Saída esperada \\ \hline
$r$ & $h$   & $V$               \\ \hline
5 & 100          & 7853.98                \\ \hline
25 & 25.5           & 69704.03             \\ \hline
10 & 50.9         & 15990.69          \\ \hline
\end{tabular}
\end{center}

\item Fazer um programa em \emph{Pascal} que efetue o cálculo do salário 
líquido de um professor. Os dados fornecidos serão: valor da hora aula, 
número de aulas dadas no mês e percentual de desconto do INSS.

\begin{center}
\begin{tabular}{|l|l|l|l|} \hline
\multicolumn{3}{|c|}{Exemplo de entrada} & Saída esperada \\ \hline
valor hora aula & número de aulas & percentual INSS   & Salário bruto               \\ \hline
6.25 & 160 & 1.3         & 987.00                \\ \hline
20.5 & 240 & 1.7          & 4836.36             \\ \hline
13.9 & 200 & 6.48         & 2599.86          \\ \hline
\end{tabular}
\end{center}

\item Em épocas de pouco dinheiro, os comerciantes estão procurando aumentar 
suas vendas oferecendo desconto aos clientes. Escreva um programa em 
\emph{Pascal}  que possa entrar com o valor de um produto e imprima o novo 
valor tendo em vista que o desconto foi de 9\%. Além disso, imprima o valor 
do desconto.

\begin{center}
\begin{tabular}{|l|l|l|} \hline
\multicolumn{2}{|c|}{Exemplo de entrada} & Saída esperada \\ \hline
valor do produto (R\$) & novo valor (R\$) & valor do desconto (R\$) \\ \hline
100 & 91.00          & 9.00                \\ \hline
1500 & 1365.00       & 135.00             \\ \hline
60000 & 54600.00     & 5400.00          \\ \hline
\end{tabular}
\end{center}

\item Todo restaurante, embora por lei não possa obrigar o cliente a pagar, 
cobra 10\% de comissão para o garçom. Crie um programa em \emph{Pascal} que 
leia o valor gasto com despesas realizadas em um restaurante e imprima o i
valor da gorjeta e o valor total com a gorjeta.

\begin{center}
\begin{tabular}{|l|l|} \hline
Exemplo de entrada & Saída esperada \\ \hline
75                & 82.50              \\ \hline
125               & 137.50               \\ \hline
350.87            & 385.96               \\ \hline
\end{tabular}
\end{center}

\item Criar um programa em \emph{Pascal} que leia um valor de hora 
(hora:minutos), calcule e imprima o total de minutos se passaram desde o 
início do dia (0:00h). A entrada será dada por dois números separados
na mesma linha, o primeiro número representa as horas e o segundo os minutos.

\begin{center}
\begin{tabular}{|l|l|l|} \hline
\multicolumn{2}{|c|}{Exemplo de entrada} & Saída esperada \\ \hline
hora & minuto & total de minutos \\ \hline
1 & 0          & 60                \\ \hline
14 & 30       & 870             \\ \hline
23 & 55     & 1435          \\ \hline
\end{tabular}
\end{center}

\item Criar um programa em \emph{Pascal} que leia o valor de um depósito e o 
valor da taxa de juros. Calcular e imprimir o valor do rendimento do depósito 
e o valor total depois do rendimento.

\begin{center}
\begin{tabular}{|l|l|l|l|} \hline
\multicolumn{2}{|c|}{Exemplo de entrada} & \multicolumn{2}{|c|}{Saída esperada} \\ \hline
depósito & taxa de juros & rendimento & total \\ \hline
200 & 0.5 & 1.00   & 201.00            \\ \hline
1050 & 1 & 10.5    & 1060.5           \\ \hline
2300.38 & 0.06 & 1.38   & 2301.38          \\ \hline
\end{tabular}
\end{center}

\item Para vários tributos, a base de cálculo é o salário mínimo. Fazer um 
programa em \emph{Pascal} que leia o valor do salário mínimo e o valor do 
salário de uma pessoa. Calcular e imprimir quantos salários mínimos essa 
pessoa ganha.

\begin{center}
\begin{tabular}{|l|l|l|} \hline
\multicolumn{2}{|c|}{Exemplo de entrada} & Saída esperada \\ \hline
salário mínimo (R\$) & salário (R\$) & salário em salários mínimos (R\$) \\ \hline
450.89 & 2700.00 & 5.99            \\ \hline
1000.00& 1000.00 & 1.00           \\ \hline
897.50& 7800.00 & 8.69          \\ \hline
\end{tabular}
\end{center}

\item Criar um programa em \emph{Pascal} que efetue o cálculo da quantidade de 
litros de combustível gastos em uma viagem, sabendo-se que o carro faz 12 km 
com um litro. Deverão ser fornecidos o tempo gasto na viagem e a velocidade 
média.  $Distancia = Tempo \times Velocidade$.  $Litros = Distancia / 12$.
O algoritmo deverá apresentar os valores da Distância percorrida e a 
quantidade de Litros utilizados na viagem.

\begin{center}
\begin{tabular}{|l|l|l|l|} \hline
\multicolumn{3}{|c|}{Exemplo de entrada} & Saída esperada \\ \hline
tempo gasto & velocidade média & distância percorrida & litros \\ \hline
60 & 100  & 6000.00 & 500.00             \\ \hline
1440 & 80 & 115200.00 & 9600.00           \\ \hline
5 & 90 & 450.00 & 37.50        \\ \hline
\end{tabular}
\end{center}

\item Um vendedor de uma loja de sapatos recebe como pagamento 20\% de comissão 
sobre as vendas do mês e R\$5.00 por cada par de sapatos vendidos. Faça
 um programa em \emph{Pascal} que, dado o 
total de vendas do mês e o número de sapatos vendidos, imprima quanto será o 
salário daquele mês do vendedor.

\begin{center}
\begin{tabular}{|l|l|l|} \hline
\multicolumn{2}{|c|}{Exemplo de entrada} & Saída esperada \\ \hline
total de vendas (R\$) & sapatos vendidos & salário (R\$) \\ \hline
50000.00 & 100 & 10500.00            \\ \hline
2000.00 & 30 & 550.00           \\ \hline
1000000.00 & 500 & 202500.00          \\ \hline
\end{tabular}
\end{center}

\item Você está endividado e quer administrar melhor sua vida financeira. 
Para isso, crie um programa em \emph{Pascal} que recebe o valor de uma dívida 
e o juros mensal, então calcule e imprima o valor da dívida no mês seguinte.

\begin{center}
\begin{tabular}{|l|l|l|} \hline
\multicolumn{2}{|c|}{Exemplo de entrada} & Saída esperada \\ \hline
valor da dívida (R\$) & juros/mês & dívida (R\$) \\ \hline
100.00 & 10  & 110.00            \\ \hline
1500.00 & 3 &  1545.00          \\ \hline
10000.00 & 0.5 & 10050.00          \\ \hline
\end{tabular}
\end{center}

\item Antes de o racionamento de energia ser decretado, quase ninguém falava 
em quilowatts; mas, agora, todos incorporaram essa palavra em seu vocabulário. 
Sabendo-se que 100 quilowatts de energia custa um sétimo do salário mínimo, 
fazer um programa em \emph{Pascal} que receba o valor do salário mínimo e a 
quantidade de quilowatts gasta por uma residência e imprima:
\begin{itemize}
\item o valor em reais de cada quilowatt;
\item o valor em reais a ser pago;
\end{itemize}

\begin{center}
\begin{tabular}{|l|l|l|l|} \hline
\multicolumn{2}{|c|}{Exemplo de entrada} & \multicolumn{2}{|c|}{Saída esperada} \\ \hline
salário mínimo (R\$) & quilowatts & valor do quilowatt (R\$) & valor pago (R\$) \\ \hline
750.00 & 200 & 1.07 & 214.29           \\ \hline
935.00 & 150 & 1.34 & 200.36         \\ \hline
1200.00 & 250 & 1.71 & 428.57        \\ \hline
\end{tabular}
\end{center}

