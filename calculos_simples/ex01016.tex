\item Certo dia o professor de Johann Friederich Carl Gauss (aos 10 anos de
idade) mandou que os alunos somassem os números de 1 a 100. Imediatamente
Gauss achou a resposta – 5050 – aparentemente sem a soma de um em um.
Supõe-se que já aí, Gauss, houvesse descoberto a fórmula de uma soma de uma
progressão aritmética.

Agora você, com o auxílio dos conceitos de algoritmos e da linguagem
\emph{Pascal} deve construir um programa que realize a soma de uma P.A.
de $n$ termos, dado o primeiro termo $a1$ e o último termo $an$.
A impressão do resultado deve ser formatada com duas casas na direita.

\begin{center}
\begin{tabular}{|l|l|l|l|} \hline
\multicolumn{3}{|c|}{Exemplo de entrada} & Saída esperada \\ \hline
$n$ & $a_1$ & $a_n$   & $soma$               \\ \hline
100 & 1 & 100         & 5050.00                \\ \hline
10 & 1 & 10          & 55.00             \\ \hline
50 & 30 & 100         & 3250.00          \\ \hline
\end{tabular}
\end{center}
