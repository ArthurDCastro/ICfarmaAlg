\item O produto de uma série de termos de uma Progressão Geométrica (P.G.)
pode ser calculado pela fórmula abaixo:

\[
P = a_1^n \times q^{\frac{n(n-1)}{2}}.
\]

Agora, escreva um programa em \emph{Pascal} para determinar o produto dos
$n$ primeiros termos de uma P.G de razão $q$. Seu programa deverá ler
$a_1, q, n$ do teclado e imprimir $P$.
(ATENÇÃO PARA O TIPO DE VARIÁVEL!)

\begin{center}
\begin{tabular}{|l|l|l|l|} \hline
\multicolumn{3}{|c|}{Exemplo de entrada} & Saída esperada \\ \hline
$a_1$ & $q$ & $n$   & $P$               \\ \hline
5 & 1 & 10         & 9765625.00                \\ \hline
1 & 1 & 10          & 1.00             \\ \hline
2 & 2 & 5         & 32768.00          \\ \hline
\end{tabular}
\end{center}
