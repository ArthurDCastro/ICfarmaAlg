\item Uma P.G. (progressão geométrica) fica determinada pela sua razão ($q$) 
e pelo primeiro termo ($a_1$). Escreva um programa em \emph{Pascal} que 
seja capaz de determinar o enésimo termo ($a_n$) de uma P.G., dado a razão 
($q$) e o primeiro termo ($a_1$). Seu programa deve ler $a_1, q, n$ do 
teclado e imprimir $a_n$.

\begin{center}
\begin{tabular}{|l|l|l|l|} \hline
\multicolumn{3}{|c|}{Exemplo de entrada} & Saída esperada \\ \hline
$a_1$ & $q$ & $n$   & $a_n$               \\ \hline
1 & 1 & 100         & 1.00                \\ \hline
2 & 2 & 10          & 1024.00             \\ \hline
10 & 2 & 20         & 5242880.00          \\ \hline
\end{tabular}
\end{center}