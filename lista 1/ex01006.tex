\item Uma P.A. (progressão aritmética) fica determinada pela sua razão ($r$) 
e pelo primeiro termo ($a_1$). Escreva um programa em \emph{Pascal} que seja 
capaz de determinar o enésimo ($n$) termo ($a_n$) de uma P.A., dado a razão 
($r$) e o primeiro termo ($a_1$). Seu programa deve ler $n, r, a_1$ do teclado
e imprimir $a_n$.

\[
a_n = a_1 + (n-1)\times r.
\]

\begin{center}
\begin{tabular}{|l|l|l|l|} \hline
\multicolumn{3}{|c|}{Exemplo de entrada} & Saída esperada \\ \hline
$n$ & $r$ & $a_1$   & $a_n$               \\ \hline
8 & 1 & 3       & 10                \\ \hline
100 & 10 & 1    & 991                \\ \hline
5 & -2 & 0      & -98                \\ \hline
\end{tabular}
\end{center}