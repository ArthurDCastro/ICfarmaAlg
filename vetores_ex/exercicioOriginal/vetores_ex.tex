\subsection{Exercícios}

\begin{enumerate}

\item Faça  um programa que leia e  armazene em um vetor  uma sequência de
inteiros.  Em seguida  o programa  deve  ler uma  sequência de  inteiros
informados pelo usuário e, para cada  um deles, dizer se ele pertence ou
não ao vetor armazenado previamente. 

\item Faça um  programa que  leia duas  sequências de  $n$ inteiros  em dois
vetores distintos,  digamos, $v$ e  $w$ e verifique  se os dois  vetores são
idênticos. 

\item Faça um programa em  que leia dois vetores de números reais
e descubra se um deles é permutação do outro, isto é, se eles
tem os mesmos elementos, ainda que em ordem diferente.
A quantidade de elementos lidos em cada vetor é no  máximo 100,  e
cada sequência termina quando o valor 0 é digitado. 
Por   exemplo: 

        $[2,2,0,3,4]$   e  $[2,2,0,3,4]$:   sim.
  
        $[2,2,0,3,4]$ e  $[4,3,2,0,2]$: sim.
        
        $[2,2,0,3,4]$ e $[4,3,4,0,2]$: não. 
        
        $[3,0,5]$ e $[3,0,5,3]$: não.

Implemente três
versões deste problema:
\begin{itemize}
\item ordenando os vetores para em seguida compará-los;
\item sem ordenar os vetores;
\item crie uma
 função que  retorna  0 se $x$  não  pertence a  $v$ e  caso contrário
 retorna o índice  do vetor onde  $x$  se encontra. Use esta função para 
 resolver este problema. 
\end{itemize}




\item Faça  um  programa  que   leia  duas  sequências  de  inteiros,  não
necessariamente  contendo a  mesma quantidade  de números. Seu programa
deverá:
\begin{itemize} 
\item dizer  se a segunda sequência está contida na primeira. Exemplo:
\begin{verbatim}
v1: 7 3 2 3 2 6 4 7 
v2: 3 2 6
Saída: sim
\end{verbatim}
\item construir um terceiro  vetor, sem  destruir os  originais, que é  
      a concatenação  do primeiro com o segundo;
\begin{verbatim}
v1: 7 3 2 6
v2: 5 1 8 4 9
Saída: 1 2 3 4 5 6 7 8 9
\end{verbatim}
\item ordená-los, e em seguida imprimir todos os números ordenados 
      em ordem crescente. Exemplo:
\begin{verbatim}
v1: 7 3 2 6
v2: 5 1 8 4 9
Saída: 1 2 3 4 5 6 7 8 9
\end{verbatim}
\end{itemize}

\item Crie uma função que receba um vetor de inteiros de 
     tamanho $n$ e devolva o valor \emph{true} se o vetor estiver ordenado e 
     \emph{false} em caso contrário.

\item Aproveitando as soluções dos problemas anteriores, escreva um programa
     que leia dois vetores de inteiros $v$ e $w$, 
     de dimensões $m$ e $n$ 
     respectivamente, verifique se eles estão ordenados, ordene-os em caso 
     contrário e, em seguida, imprima a intercalação dos dois. \\
     Exemplo de intercalação: $v$: 1 4 6 9; $w$: 2, 3, 5, 7. \\
     Saída: 1, 2, 3, 4, 5, 6, 7, 9. 

\item Dados dois números naturais $m$ e $n$, uma frase com $m$ letras
e uma palavra com $n$ letras, escreva um procedimento que determine o número de
vezes que a palavra ocorre na frase e a posição em que cada ocorrência inicia.

    Exemplo:
    \begin{verbatim}
    Para M = 30, N = 3, a palavra ANA e a frase:
        ANA E MARIANA GOSTAM DE BANANA
    A palavra ANA ocorre 4 vezes, nas posições 1, 11, 26, 28.
    \end{verbatim}

\item Dada uma sequência de $N$ números, determinar quantos números dintintos
compõe  a sequência e  o número  de vezes  que cada  um deles  ocorre na
mesma. Exemplo:

N=5
1 2 3 2 3 
a sequência tem três números distintos, 1, 2 e 3. Ocorrências:
1 ocorre 1 vez
2 ocorre 2 vezes
3 ocorre 2 vezes

\item Dadas  duas  sequências  com  $n$  números  inteiros  entre  0  e  1,
interpretados como números binários:
\begin{enumerate}
     \item imprimir o valor decimal dos números;
     \item calcular a soma de ambos (em binário), usando o ``vai-um'';
     \item imprimir o valor decimal da soma.
\end{enumerate}


\item Escreva um programa em  que leia os seguintes valores:
um inteiro $B$, um inteiro $N$ ($1 \leq N \leq 10$), e $N$ valores inteiros.
A ideia é que estes valores sejam entendidos como a representação
de um número não negativo na base $B$. 
Estes valores deverão ser inseridos em um vetor de tamanho $N+1$, onde
a primeira posição armazena a base $B$ e as outras $N$ posições
o restante dos números lidos. 
%
Note que o  intervalo de valores possíveis para cada dígito na
base $B$ é $[0, B-1]$.
%
Seu programa deve retornar o valor em decimal do número representado
no vetor. Se o número representado no vetor não for válido na
base $B$ então deverá ser retornado o código de erro ``-1''.
Por exemplo, se $B=3$ o número 2102 na base 3 equivale ao valor decimal 65;
se $B=4$ o número $35$ é inválido na base 4.


\item Faça um programa que, 
dadas duas sequências com
$N$ números inteiros entre 0 e 9, interpretadas como dois números
inteiros de $N$ algarismos, calcular a sequência de números que
representa a soma dos dois inteiros, usando o ``vai-um''. Por exemplo:\\

\verb|        N=6,|\\
\verb|                4 3 4 2 5 1|\\
\verb|              + 7 5 2 3 3 7|\\
\verb|              1 1 8 6 5 8 8|



\item Dada  uma  sequência  $x_1$,  $x_2$, \ldots,  $x_n$  de  
números  inteiros,
determinar um segmento de soma  máxima.  Exemplo: na sequência 5, 2, -2,
-7, 3, 14,  10, -3, 9, -6, 4,  1, a soma do maior segmento  é 33, obtida
pela soma dos números de 3 até 9. 
 
\item Implemente um programa que leia um vetor de 1 milhão de inteiros
   em um vetor de inteiros. Gere um número aleatório e o procure
   neste vetor de duas maneiras diferentes: uma usando busca com 
   sentinela e outra usando busca binária. Seu programa deve imprimir
   uma tabela com o número de comparações feitas em cada um dos casos
   acima (busca com sentinela e busca binária). Desconsidere o tempo
   gasto com ordenação no caso da busca binária. A busca com sentinela
   deve ser feita em um vetor não ordenado. Gere 200 números aleatórios
   e imprima a média de comparações para cada um dos dois algoritmos
   sendo testados.

\item 
Suponha que um ex\'ercito tenha 20 regimentos e que eles est\~ao em processo
de forma\c c\~ao. Inicialmente o primeiro tem 1000 homens, o segundo 950,
o terceiro 900, e assim por diante, at\'e o vig\'esimo que tem 50. Suponhamos
que a cada semana 100 homens s\~ao enviados para cada regimento, e no final
da semana o maior regimento \'e enviado para o {\em front}. Imaginemos que o
general do quinto regimento \'e companheiro de xadrez do comandante supremo,
e que eles est\~ao no meio de uma partida. O comandante supremo ent\~ao
envia apenas 30 homens para o quinto regimento a cada semana, esperando com
isto poder acabar o jogo com seu colega.
Escreva um programa em 
que diga, a cada semana, qual \'e o regimento enviado ao {\em front} e mostre o {\em status} dos outros regimentos. O programa
deve tamb\'em determinar exatamente quantas semanas levar\'a o quinto regimento
para ser deslocado ao {\em front}. 

\item Suponha que voc\^e esteja usando o m\'etodo da ordenação 
por sele\c c\~ao. Qual das sequências abaixo requerir\'a o menor n\'umero
de trocas? Quantas? Qual requerir\'a o maior n\'umero de trocas? Quantas?
Explique.
\begin {enumerate}
\item 10, 9, 8, 7, 6, 5, 4, 3, 2, 1.
\item 5, 4, 3, 2, 1, 10, 9, 8, 7, 6.
\item 10, 1, 9, 2, 8, 3, 7, 4, 6, 5.
\item 2, 3, 4, 5, 6, 7, 8, 9, 10, 1.
\item 1, 10, 2, 9, 3, 8, 4, 7, 5, 6.
\end {enumerate}

\item Suponha que voc\^e tem uma vari\'avel do tipo vetor declarada como:
\emph{array [1..50] of real;}. 
Fa\c ca uma fun\c c\~ao 
que inicialize o vetor de modo que os elementos
de \'{\i}ndices \'{\i}mpares recebam o valor inicial -2.0 e os elementos
de \'{\i}ndices pares recebam o valor inicial 7.0. Sua função deve
fazer uso de um único comando de repetição, que incrementa de um em um, e de 
nenhum comando de desvio condicional.

\item Qual dos seguintes problemas requer o uso de vetores para uma solu\c c\~ao elegante?
\begin {enumerate}
\item Ler cerca de duzentos n\'umeros e imprimir os que est\~ao em uma certa faixa;
\item Computar a soma de uma sequ\^encia de n\'umeros;
\item Ler exatamente duzentos n\'umeros e orden\'a-los em ordem crescente;
\item Encontrar o segundo menor elemento de uma sequ\^encia de entrada;
\item Encontrar o menor inteiro de uma sequ\^encia de inteiros.
\end {enumerate}

\item Considere um vetor  declarado como: 
\emph{array [1..50] of integer} que tem a 
particularidade de todos os elementos estarem entre 1 e 30, sendo que
nenhum \'e repetido. Fa\c ca um programa
que ordene o vetor de maneira eficiente explorando esta característica e fazendo
o menor n\'umero poss\'{\i}vel de trocas.

\item   Dada uma sequência $x_1, x_2, \ldots, x_k$ de números
  reais, verifique  se existem dois segmentos  consecutivos iguais nesta
  sequência, isto é, se existem $i$ e $m$ tais que:
\[ x_i, x_{i+1}, \ldots, x_{i+m-1}=x_{i+m}, x_{i+m+1}, \ldots, x_{i+2m-1}.
\]
Imprima, caso existam,  os valores de $i$ e de  $m$. Caso contrário, não
imprima  nada. Exemplo:  Na sequência  7,9,5,4,5,4,8,6, existem  $i=3$ e
$m=2$. 

\item   Um coeficiente binomial, geralmente denotado
  $\binom{n}{k}$, representa o número de possíveis combinações de $n$
  elementos tomados $k$ a $k$. Um ``Triângulo de Pascal'', uma homenagem
  ao grande matemático Blaise Pascal, é uma tabela de valores de
  coeficientes combinatoriais para pequenos valores de $n$ e $k$. Os
  números que não são mostrados na tabela têm valor zero. Este triângulo
  pode ser construído automaticamente usando-se uma propriedade
  conhecida dos coeficientes binomiais, denominada ``fórmula da
  adição'':
$\binom{r}{k} = \binom{r-1}{k} + \binom{r-1}{k-1}.$
Ou seja, cada elemento do triângulo é a soma de dois elementos da linha
anterior, um da mesma coluna e um da coluna anterior. 
Veja  um  exemplo de  um  triângulo  de Pascal  com  7  linhas:
\[\begin{array}{ccccccc}
1 \\
1 & 1 \\
1 & 2 & 1 \\
1 & 3 & 3 & 1 \\
1 & 4 & 6 & 4 & 1 \\
1 & 5 & 10 & 10 & 5 & 1 \\
1 & 6 & 15 & 20 & 15 & 6 & 1 \\
\end{array}
\]
Faça um programa que imprima na tela um triângulo de Pascal
com 10 linhas.  Seu programa deve obrigatoriamente fazer uso de
exatamente dois vetores durante o processo de construção. Um deles
conterá a última linha ímpar gerada, enquanto que o outro conterá a
última linha par gerada. Lembre-se que os elementos que não aparecem na
tabela tem valor nulo. Você deve sempre ter o controle do tamanho da
última linha impressa (o tamanho útil dos vetores em cada passo). Você
deve também usar um procedimento para imprimir o vetor. Observe que não
há entrada de dados, os dois vetores são gerados, um a partir do outro.
O único elemento da primeira linha tem o valor 1. Você deve
obrigatoriamente declarar um tipo vetor com tamanho máximo
Tam\_max\_vetor, e o seu programa deverá tomar cuidado para manipular
corretamente vetores de tamanho menor do que o tamanho máximo, impedindo
que haja uma atribuição em posição ilegal de memória. 


\item Resolva o problema do triângulo de Pascal usando somente um vetor.

\item Seja um polinômio $p(x)= a_0 + a_1 x + a_2 x^2 + \ldots + a_n x^n$ de
  grau $n \ge 2$. Uma possível maneira de calcular uma raiz do polinômio
  é pelo ``método de Newton''. Este método consiste em se fornecer uma
  aproximação inicial para a raiz, isto é, um valor que não é a raiz
  exata, mas é um valor próximo.  Assim, se $x_0$ é esta
  aproximação inicial, $p(x_0)$ não é zero mas espera-se que
  seja próximo de zero.  A obtenção da raiz pelo método de Newton é
  feita pelo refinamento desta solução inicial, isto é, pela tentativa
  de minimizar o erro cometido. Isto é feito pela expressão seguinte:
\[x_{n+1}=x_{n}-\frac{p(x_{n})}{p\prime(x_{n})},\]  $n=0,1,2,\ldots$,  e
  onde $p\prime(x)$ é a primeira derivada de $p(x$). Usualmente, repete-se
  este refinamento até que $|x_{n+1}-x_{n}| < \epsilon, \epsilon>0$, ou
  até que $m$ iterações tenham sido executadas. \\
  Construa um programa em  que receba como dados de entrada 
  um polinômio $p(x)= a_0 + a_1 x + a_2 x^2 + \ldots + a_n x^n$ e
  uma  aproximação inicial  $x_0$ da  raiz de  $p(x)$, $\epsilon>0$  e o
  número  máximo de  iterações, e  calcule  uma aproximação  da raiz  de
  $p(x)$ pelo método de Newton. Utilize obrigatoriamente um procedimento
  que receba como parâmetro  um polinômio $p(x)$ (incluindo a informação
  sobre o grau do polinômio) e que calcule e retorne a função
  derivada  $p\prime(x)$.  Utilize também  uma  função  que receba  como
  parâmetros um polinômio $p(x)$ e um valor real $\overline{x}$ e retorne
  o   valor    do   polinômio    no   ponto   $\overline{x}$,    isto   é
  $p(\overline{x})$. Use  esta função para  calcular, a cada  iteração do
  método    de    Newton,    os    valores   de    $p(x_{n})$    e    de
  $p\prime(x_{n})$.  

\item Faça um programa em  que leia uma sequência de 10 
     letras (caracteres de A a Z), as armazene em um vetor de 10 posições 
     e imprima a lista de letras repetidas no vetor. Sendo assim, para os 
     dados: A J G A D F G A A, a saída deve ser: A G.

\item Escreva o programa da busca binária de um valor $x$ num 
     vetor de inteiros que, ao invés de achar a primeira ocorrência do valor 
     na lista, identifique e imprima o menor índice do vetor no qual o valor 
     ocorra.

\item Escreva um programa que leia uma sequência de {\em código de
operação} e {\em valor}, onde o {\em código de operação} é um inteiro com
os seguintes valores:
\begin{itemize}
\item 0 (zero): fim
\item 1 (um): inserção
\item 2 (dois): remoção
\end{itemize} 

O {\em valor} lido é um real que deve ser inserido em um vetor (caso a
operação seja 1), ou removido do vetor (caso a operação seja 2).   As
inserções no vetor devem ser realizadas de forma que o vetor esteja sempre
ordenado.   No final do programa o vetor resultante deve ser impresso.

Detalhamento:
\begin{itemize}
\item a quantidade máxima de valores que pode  ser inserida é 100;
\item se a quantidade máxima for ultrapassada o programa deve dar uma
mensagem de erro;
\item se for requisitada a remoção de um número não existente o programa
  deve dar uma mensagem de erro;
\item se o código de operação for inválido o programa deve continuar lendo
um novo código até que ele seja 0 (zero), 1 (um) ou 2 (dois).
\end{itemize}

\noindent
{\bf Exemplo de execução: }
\begin{small}
\begin{verbatim}
Entre com operacao (0=fim, 1=insercao, 2=remocao): 1 
Valor: 45.3 
Entre com operacao (0=fim, 1=insercao, 2=remocao): 1 
Valor: 34.3 
Entre com operacao (0=fim, 1=insercao, 2=remocao): 1 
Valor: 40.8
Entre com operacao (0=fim, 1=insercao, 2=remocao): 2 
Valor: 34.3
Entre com operacao (0=fim, 1=insercao, 2=remocao): 0
Vetor resultante
40.8   45.3 
\end{verbatim}
\end{small}

\begin{center}
\begin{tabular}{|p{1cm}|p{1cm}|p{1cm}|p{1cm}|p{1cm}|p{1cm}|l}
\multicolumn{1}{c}{1} & \multicolumn{1}{c}{2} & \multicolumn{1}{c}{3} & \multicolumn{1}{c}{4} & \multicolumn{1}{c}{5} & \multicolumn{1}{c}{6} & \\ \cline{1-6} 
&&&&&& \mbox{início}\\ \cline{1-6}
\multicolumn{1}{c}{} & \multicolumn{1}{c}{} & \multicolumn{1}{c}{} & \multicolumn{1}{c}{} & \multicolumn{1}{c}{} & \multicolumn{1}{c}{} & \\ \cline{1-6} 
45.3&&&&&& \mbox{após inserção de 45.3}\\ \cline{1-6}
\multicolumn{1}{c}{} & \multicolumn{1}{c}{} & \multicolumn{1}{c}{} & \multicolumn{1}{c}{} & \multicolumn{1}{c}{} & \multicolumn{1}{c}{} & \\ \cline{1-6} 
34.3&45.3&&&&& \mbox{após inserção de 34.3}\\ \cline{1-6}
\multicolumn{1}{c}{} & \multicolumn{1}{c}{} & \multicolumn{1}{c}{} & \multicolumn{1}{c}{} & \multicolumn{1}{c}{} & \multicolumn{1}{c}{} & \\ \cline{1-6} 
34.3&40.8&45.3&&&& \mbox{após inserção de 40.8}\\ \cline{1-6}
\multicolumn{1}{c}{} & \multicolumn{1}{c}{} & \multicolumn{1}{c}{} & \multicolumn{1}{c}{} & \multicolumn{1}{c}{} & \multicolumn{1}{c}{} & \\ \cline{1-6} 
40.8&45.3&&&&& \mbox{após remoção de 34.3}\\ \cline{1-6}
\end{tabular}
\end{center}

\item Escreva um programa 
que leia duas sequências de caracteres e
verifica se a segunda sequência é subpalavra da primeira.
Por exemplo, {\em todo} é subpalavra de {\em metodo} e
{\em ar} é subpalavra de {\em farmacia}.  Porém, {\em todo} não é
subpalavra de {\em todavia}.   A leitura das sequências deve ser feita
caracter por caracter e o final de cada sequência é sinalizada pelo
caracter '.'.    Se a segunda sequência é uma subpalavra, a saída do 
programa deve ser a posição na qual ela começa.  Caso contrário, escrever
a mensagem ``Nao eh subpalavra.''. 
{Observações:} 
\begin{itemize}
\item cada sequência tem no máximo 80 caracteres.
\item você não pode utilizar funções de manipulação de cadeias de
caracteres existentes no compilador, mas somente as funções para o
tipo {\tt char}.
\end{itemize}

\noindent
{Exemplo de execução: }
\small
\begin{verbatim}
Entre com duas palavras terminadas por ponto:
metodo.todo.

A segunda subpalavra comeca na posicao 3 da primeira.
\end{verbatim}



\item Escreva um programa que leia uma sequência de $n$ valores
reais ($n \leq 100$) e os insira num vetor.  A sequência termina quando o
valor lido for 0.  O programa deve escrever o valor da divisão 
da soma dos valores positivos pela soma dos valores
negativos que estão armazenados no vetor. Cuidado com divisões por zero.

\item Escreva uma função em  que substitui em um texto a primeira
ocorrência de  uma   palavra por   outra.  A  função  deve  retornar
\verb+true+ se a subs\-ti\-tui\-ção for bem sucedida e
\verb+false+ caso a palavra  não seja encontrada no  texto.  O texto e
as palavras são  representados  por vetores  do  tipo \verb+char+. Por
exemplo:
\vspace{-0.2cm}
\begin{center}
\begin{small}
\begin{verbatim}
        +---+---+---+---+---+---+---+---+---+---+
texto1  | e | x | e | m | p | r | o |   | u | n |
        +---+---+---+---+---+---+---+---+---+---+
          +---+---+---+---+---+
palavra1  | r | o |   | u | n |
          +---+---+---+---+---+
          +---+---+---+---+---+---+---+
palavra2  | l | o |   | d | o | i | s |
          +---+---+---+---+---+---+---+
        +---+---+---+---+---+---+---+---+---+---+---+---+
texto2  | e | x | e | m | p | l | o |   | d | o | i | s |
        +---+---+---+---+---+---+---+---+---+---+---+---+
\end{verbatim}
\end{small}
\end{center}
\vspace{-0.2cm}
A função recebe como parâmetros o texto, a palavra a ser substituída e
a nova palavra.  No exemplo, \verb+texto1+  mostra o estado inicial do
texto  e \verb+texto2+  o   estado do texto  após  a   substituição da
\verb+palavra1+ pela \verb+palavra2+.

Você pode usar, caso seja necessário, a função:
\begin{center}
\begin{small}
\begin{verbatim}
   buscachar(texto, pos, letra);
\end{verbatim}
\end{small}
\end{center}
que busca  um  caractere (\verb+letra+) a  partir   de uma determinada
posição (\verb+pos+) em um vetor que  contém o texto (\verb+texto+). A
função \verb+buscaletra+  retorna a posição  no  vetor \verb+texto+ da
primeira ocorrência  de  \verb+letra+, se \verb+letra+  não aparece no
texto a função retorna \verb+-1+.

\item Um \emph{algoritmo genético} é um procedimento computacional de busca,
inspirado no processo biológico de  evolução, que otimiza a solução de
um  pro\-ble\-ma.  O   problema é   modelado por: uma    população de
indivíduos que representam possíveis soluções; uma função que avalia a
qualidade da solução representada por cada indivíduo da população e um
conjunto  de operadores  genéticos.   Os   indivíduos  são dados   por
sequências   de genes que   representam características  da solução do
problema.  O procedimento consiste  em aplicar os operadores genéticos
sobre a população, gerando novos indivíduos e selecionar os mais aptos
para  constituirem uma  nova população.  Esse  processo é repetido até
que uma solução adequada seja obtida.  Dentre os operadores genéticos,
o mais importante é  o de recombinação genética  (\emph{crossover}) de
dois indivíduos.  Esse operador corta  em duas partes as sequências de
genes de dois indivíduos pais (\verb+pai1+  e \verb+pai2+) e gera dois
novos   indivíduos    filhos    (\verb+filho1+    e    \verb+filho2+).
\verb+filho1+ é dado pela contatenação da primeira  parte dos genes de
\verb+pai1+  com a segunda parte  de  \verb+pai2+ e \verb+filho2+ pela
concatenação  da primeira parte de \verb+pai2+  com a segunda parte de
\verb+pai1+.  O diagrama abaixo  exemplifica a operação  em indivíduos
representados por vetores de números inteiros  onde a primeira posição
contém o tamanho do vetor:
\vspace{-0.2cm}
\begin{center}
\begin{small}
\begin{verbatim}
                          corte1
        +----+---+---+---#---+---+---+---+---+---+---+---+
pai1    | 11 | 1 | 1 | 1 # 2 | 2 | 2 | 2 | 2 | 2 | 2 | 2 |
        +----+---+---+---#---+---+---+---+---+---+---+---+

                                  corte2
        +----+---+---+---+---+---#---+---+---+---+---+
pai2    | 10 | 3 | 3 | 3 | 3 | 3 # 4 | 4 | 4 | 4 | 4 |
        +----+---+---+---+---+---#---+---+---+---+---+

        +----+---+---+---+---+---+---+---+---+
filho1  |  8 | 1 | 1 | 1 | 4 | 4 | 4 | 4 | 4 |
        +----+---+---+---+---+---+---+---+---+

        +----+---+---+---+---+---+---+---+---+---+---+---+---+---+
filho2  | 13 | 3 | 3 | 3 | 3 | 3 | 2 | 2 | 2 | 2 | 2 | 2 | 2 | 2 |
        +----+---+---+---+---+---+---+---+---+---+---+---+---+---+
\end{verbatim}
\end{small}
\end{center}
Escreva um procedimento  em  que   execute a operação   de
recombinação descrita acima, usando  a estrutura de dados vetor.  O
procedimento deve  receber seis parâmetros, um vetor re\-pre\-sentando
o primeiro pai,   a posição  de   corte no   primeiro  pai, um   vetor
re\-pre\-sentando o segundo pai, a posição do corte  no segundo pai, e
dois vetores que receberão os novos indivíduos. No exemplo apresentado
a chamada do procedimento seria:
\vspace{-0.2cm}
\begin{center}
\begin{small}
\begin{verbatim}
   corte1 := 4;
   corte2 := 6;
   crossover(pai1, corte1, pai2, corte2, filho1, filho2);
\end{verbatim}
\end{small}
\end{center}
\vspace{-0.2cm}
Note que os vetores  devem iniciar na   posição zero e essa  posição é
usada para armazenar o tamanho do vetor. No caso do exemplo,
\verb+ pai1[0]=11+, 
\verb+ pai2[0]=10+,
\verb+filho1[0]=8 + e
\verb+ filho2[0]=13+.
Os pontos de corte devem estar dentro dos vetores:
\verb+1 < corte1 <= pai1[0] + e 
\verb+ 1 < corte2 <= pai2[0]+.

\item Escreva um procedimento em  
que implemente a subtração de números
binários. Considere que os números binários têm $N$ bits e que os bits
são armazenados em vetores de inteiros de $N$ posições indexadas de 1 a 
$N$. O primeiro bit do vetor representa o sinal do número, sendo zero (0) 
para o números positivos e um (1) para negativos. Os demais bits 
representam o valor absoluto do número. Por exemplo, para $N=11$, os 
números decimais $-13$, 12 e 1010 são representados pelos seguintes 
vetores:
\begin{center}
\begin{small}
\begin{verbatim}
        +---+---+---+---+---+---+---+---+---+---+---+
   -13: | 1 | 0 | 0 | 0 | 0 | 0 | 0 | 1 | 1 | 0 | 1 |
        +---+---+---+---+---+---+---+---+---+---+---+

        +---+---+---+---+---+---+---+---+---+---+---+
    12: | 0 | 0 | 0 | 0 | 0 | 0 | 0 | 1 | 1 | 0 | 0 |
        +---+---+---+---+---+---+---+---+---+---+---+

        +---+---+---+---+---+---+---+---+---+---+---+
  1010: | 0 | 1 | 1 | 1 | 1 | 1 | 1 | 0 | 0 | 1 | 0 |
        +---+---+---+---+---+---+---+---+---+---+---+
\end{verbatim}
\end{small}
\end{center}
O procedimento recebe dois vetores do mesmo tamanho como parâmetros e 
deve gerar como resultado um vetor que contenha a subtração do 
primeiro pelo segundo. Por exemplo $-12 - 1010 = -1022$:
\begin{center}
\begin{small}
\begin{verbatim}
        +---+---+---+---+---+---+---+---+---+---+---+ 
  -998: | 1 | 1 | 1 | 1 | 1 | 1 | 0 | 0 | 1 | 1 | 0 |
        +---+---+---+---+---+---+---+---+---+---+---+ 
\end{verbatim}
\end{small}
\end{center}
Caso o resultado tenha mais bits que o espaço disponível no vetor
o procedimento deve descartar os bits excedentes. 

\item Escreva um programa em  
que leia uma sequência de $N$ valores
reais não nulos ($N \leq 100$) e os insira em um vetor.
A sequência termina quando o valor lido for 0. O elemento zero não faz
parte do vetor. Leia do teclado um inteiro $p$ ($p \leq N$) e
considere que o elemento $V[p]$ como o pivô na operação
de rearranjar o vetor de tal maneira que todos os elementos à
esquerda de $V[p]$ sejam menores que ele e todos os da direta sejam
maiores ou iguais a ele. Por exemplo, considere o seguinte vetor dado
como entrada: 
\begin{center}
\begin{small}
\begin{verbatim}
    +------+------+------+------+------+------+------+
    | 99.7 | 32.6 | 2.45 | 13.4 | 26.7 | 12.2 | 0.51 | 
    +------+------+------+------+------+------+------+
\end{verbatim}
\end{small}
\end{center}
e $P$ valendo 3, o programa deve gerar como resultado um vetor onde
todos os elementos que estão à esquerda do valor $2.45$ no vetor sâo menores
que ele, enquanto que os da direita são maiores do que ele.

\item Escreva uma função em  que procura uma palavra dentro
de um texto. A função deve receber como parâmetros:
\begin{itemize}
\item um vetor do tipo \verb+texto+ que contém um texto;
\item o tamanho do vetor que contém o texto;
\item a posição inicial da busca dentro do vetor que contém o 
texto;
\item um vetor do tipo \verb+texto+ que contém uma palavra;
\item o tamanho do vetor que contém a palavra.
\end{itemize}
A função deve retornar um número inteiro indicando a posição
no texto onde a palavra foi encontrada pela primeira vez. 

Caso a palavra não seja encontrada ou algum erro ocorra o 
valor retornado deve ser zero. A busca pela palavra no texto 
deve iniciar na posição passada como parâmetro para a função.

O tipo \verb+texto+ é dado por:
\begin{center}
\begin{small}
\begin{verbatim}
   const
      TAMMAX = 10000;

   type
      texto = array [1..TAMMAX] of char;

\end{verbatim}
\end{small}
\end{center}

\item Escreva um programa em  que gere e imprima um
vetor de números reais de tamanho $N$, $1 \le N \le MAX$. A criação
do vetor deve ser feita da seguinte maneira:
\begin{itemize}
\item O tamanho $N$ do vetor deve ser lido do teclado; 
\item Os $N$ números reais são gerados aleatoriamente no
intervalo $[R_{min},R_{max}[$, com $R_{min}$ e $R_{max}$ lidos do teclado;
\item A posição em que cada elemento real é inserida no vetor também é gerada
      aleatoriamente;
\item Se uma posição $i$ sorteada já estiver ocupada, seu algoritmo deve
encontrar a primeira posição $j$ não ocupada, iniciando a partir de $i+1$ até
o final do vetor. Se todas as posição entre $i+1$ e o final do vetor
estiverem ocupadas, seu algoritmo deve pegar a primeira posição livre a
partir do início do vetor.
\end{itemize}

Dica: a função \verb|random| sem parâmetros retorna um número real no intervalo
$[0,1[$, e a função \verb|random(n)| retorna um número inteiro no intervalo
$[0,n[$.

\item Escreva um procedimento em  que remove um
  elemento de uma determinada posição $p$ de um vetor $v$ de $n$ números
  reais. O vetor não está ordenado. 
  Use a seguinte assinatura para o procedimento: 

\verb!   procedure remove(var v: vetor; var n: integer; p: integer);!


\item Escreva um procedimento em  que altere um
  vetor de $N$ números reais da seguinte forma: todos os elementos
  repetidos do vetor devem ir para o final do vetor, mas de maneira
  que estes últimos fiquem em ordem crescente. Exemplos:

\verb!   ENTRADA: 5 3 8 2 3 9 8 9 7 5 3               ENTRADA: 4 4 3 3 2 2!

\vspace*{-1mm}
\verb!   SAÍDA  : 5 3 8 2 9 7 3 3 5 8 9               SAÍDA  : 4 3 2 2 3 4!

\item Em uma festa estiveram presentes 150 pessoas. Cada uma delas recebeu
um crachá na entrada com um número entre 1 e 150, número que
representa a ordem de entrada de cada convidado.

Como em toda festa, cada um dos presentes cumprimentou outras pessoas
com apertos de mão. Ao final da festa, cada convidado sabia exatamente
quantas vezes tinha apertado a mão de outras pessoas.

Na saída, ao entregar o crachá ao recepcionista, cada convidado
informou o número do seu crachá e quantas vezes trocou apertos de mão
na festa.

Muito curioso, o recepcionista queria saber quantos convidados eram
muito populares no encontro, isto é, queria saber o número de pessoas
que apertaram a mão de pelo menos outros 120 convidados.

Faça um programa
que modele o problema do recepcionista e que produza como saída o
número de celebridades (cumprimentadas pelo menos 120 vezes) presentes
na festa.

\item Um procedimento chamado \texttt{nova\_geracao} 
recebe como parâmetros dois  vetores (origem e  destino) e o
tamanho  dos vetores.  Este  procedimento constrói  um  novo vetor  de
valores  0 ou  1  (destino) a  partir  do conteúdo  do primeiro  vetor
(origem). Seja $O$ o vetor origem e $D$ o vetor destino, a regra para 
a construção do novo vetor é dada por:
\begin{itemize}
\item 
se: $O[i-1] = 0,\ O[i] = 0,\ O[i+1] = 0$ então: $D[i] = 0$
\item 
se: $O[i-1] = 0,\ O[i] = 0,\ O[i+1] = 1$ então: $D[i] = 1$
\item 
se: $O[i-1] = 0,\ O[i] = 1,\ O[i+1] = 0$ então: $D[i] = 1$
\item 
se: $O[i-1] = 0,\ O[i] = 1,\ O[i+1] = 1$ então: $D[i] = 1$
\item 
se: $O[i-1] = 1,\ O[i] = 0,\ O[i+1] = 0$ então: $D[i] = 1$
\item 
se: $O[i-1] = 1,\ O[i] = 0,\ O[i+1] = 1$ então: $D[i] = 0$
\item 
se: $O[i-1] = 1,\ O[i] = 1,\ O[i+1] = 0$ então: $D[i] = 0$
\item 
se: $O[i-1] = 1,\ O[i] = 1,\ O[i+1] = 1$ então: $D[i] = 0$
\end{itemize}
Onde $i$ indica uma posição do vetor $D$. Considere o valor 0 para as
bordas externas do vetor origem $O$. Escreva o procedimento
\texttt{nova\_geracao} e separe a regra descrita acima em uma função
que dados os valores das 3 posições consecutivas do vetor origem
($O[i-1], O[i], O[i+1]$) calcula o valor correspondente no vetor
destino ($D[i]$).

\item Faça um programa em  que simule o 
tráfego em um trecho de uma rodovia de mão única, ou seja, uma rodovia na 
qual os veículos entram de um lado e saem do outro. 

\begin{itemize}
    \item A rodovia é representada por um vetor com \verb|TAM_RODOVIA| posições;
    \item A simulação ocorre durante \verb|MAX_TEMPO| iterações;
    \item Através da chamada do procedimento \\
          \verb|detecta_entrada(VAR tipo, placa, velocidade:INTEGER)|,\\
          o programador é 
          informado sobre a ocorrência ou não da entrada de um veículo na
          rodovia, bem como o tipo do veículo, sua placa e sua respectiva 
          velocidade, onde:
        \begin{itemize}
            \item \emph{tipo:} $0$ - nenhuma nova entrada, 
                               $1$ - entrou automóvel, 
                               $2$ - entrou caminhão;
            \item \emph{placa:} um número inteiro;
            \item \emph{velocidade:} a velocidade de deslocamento do veículo 
                   (em posições/unidade de tempo).
        \end{itemize}
    \item Veículos do tipo automóvel ocupam uma posição da rodovia. 
          Caminhões ocupam duas posições.
    \item Quando veículos mais rápidos alcançam veículos mais lentos, os 
          primeiros devem andar mais devagar, pois não podem ultrapassar.
\end{itemize}    
    

A cada unidade de tempo em que algum veículo sair da rodovia, seu programa
deve imprimir esta unidade de tempo e o número da placa do veículo que saiu.

Exemplo: (\verb|TAM_RODOVIA=7|, \verb|MAX_TEMPO=10|)
\begin{itemize}
    \item Entrada:
    \begin{itemize}
        \item \textbf{t=1:} $tipo=2, placa=35, velocidade=1$
        \item \textbf{t=2:} $tipo=0$
        \item \textbf{t=3:} $tipo=1, placa=27, velocidade=4$
        \item \textbf{t=4:} $tipo=0$
        \item \textbf{t=5:} $tipo=0$
        \item \textbf{t=6:} $tipo=1, placa=16, velocidade=2$
        \item \textbf{t=7:} $tipo=0$
        \item \textbf{t=8:} $tipo=0$
        \item \textbf{t=9:} $tipo=0$
        \item \textbf{t=10:} $tipo=0$
    \end{itemize}

    \item Representação gráfica:

    \begin{itemize}
        \item \textbf{t=1:} \begin{tabular}{|p{1cm}|p{1cm}|p{1cm}|p{1cm}|p{1cm}|p{1cm}|p{1cm}|}
            \hline $35_1$&$35_1$& & & & &\\ \hline \end{tabular}
        \item \textbf{t=2:} \begin{tabular}{|p{1cm}|p{1cm}|p{1cm}|p{1cm}|p{1cm}|p{1cm}|p{1cm}|}
            \hline &$35_1$& $35_1$& & & &\\ \hline \end{tabular}
        \item \textbf{t=3:} \begin{tabular}{|p{1cm}|p{1cm}|p{1cm}|p{1cm}|p{1cm}|p{1cm}|p{1cm}|}
            \hline $27_4$& & $35_1$&$35_1$ & & &\\ \hline \end{tabular}
        \item \textbf{t=4:} \begin{tabular}{|p{1cm}|p{1cm}|p{1cm}|p{1cm}|p{1cm}|p{1cm}|p{1cm}|}
            \hline & & $27_4$& $35_1$& $35_1$& &\\ \hline \end{tabular}
        \item \textbf{t=5:} \begin{tabular}{|p{1cm}|p{1cm}|p{1cm}|p{1cm}|p{1cm}|p{1cm}|p{1cm}|}
            \hline & & & $27_4$& $35_1$& $35_1$&\\ \hline \end{tabular} 
        \item \textbf{t=6:} \begin{tabular}{|p{1cm}|p{1cm}|p{1cm}|p{1cm}|p{1cm}|p{1cm}|p{1cm}|}
            \hline $16_2$& & & & $27_4$& $35_1$& $35_1$ \\ \hline \end{tabular} 
        \item \textbf{t=7:} \begin{tabular}{|p{1cm}|p{1cm}|p{1cm}|p{1cm}|p{1cm}|p{1cm}|p{1cm}|}
            \hline & &$16_2$ & & & $27_4$& $35_1$ \\ \hline \end{tabular} 
        \item \textbf{t=8:} \begin{tabular}{|p{1cm}|p{1cm}|p{1cm}|p{1cm}|p{1cm}|p{1cm}|p{1cm}|}
            \hline & & & & $16_2$& & $27_4$ \\ \hline \end{tabular}
        \item \textbf{t=9:} \begin{tabular}{|p{1cm}|p{1cm}|p{1cm}|p{1cm}|p{1cm}|p{1cm}|p{1cm}|}
            \hline & & & & & & $16_2$ \\ \hline \end{tabular}
        \item \textbf{t=10:} \begin{tabular}{|p{1cm}|p{1cm}|p{1cm}|p{1cm}|p{1cm}|p{1cm}|p{1cm}|}
            \hline & & & & & & \\ \hline \end{tabular}
    \end{itemize}

    \item Saída:
    \begin{itemize}
        \item \textbf{t=8:} $35$
        \item \textbf{t=9:} $27$
        \item \textbf{t=10:} $16$
    \end{itemize}
\end{itemize}

%\begin{lstlistings}[frame=trBL]
%Code
%\end{lstlistings}


\item Você deve incluir no enunciado da questão anterior a
existência de uma pista de ultrapassagem. Agora, veículos mais rápidos 
podem mover-se para a pista de ultrapassagem ao alcançarem veículos mais 
lentos, desde que não haja ninguém ocupando aquele trecho de pista. 
Eles devem retornar à pista original assim que tiverem completado 
a ultrapassagem, retomando a velocidade original. Você deve escrever unicamente
os procedimentos modificados ou novos que levam em conta este novo fato.

Exemplo da nova saída para a entrada original:
\begin{itemize}
    \item Representação gráfica:
    \begin{itemize}
        \item \textbf{t=1:} \begin{tabular}{|p{1cm}|p{1cm}|p{1cm}|p{1cm}|p{1cm}|p{1cm}|p{1cm}|}
            \hline & & & & & &\\
            \hline $35_1$&$35_1$& & & & &\\ \hline \end{tabular}
        \item \textbf{t=2:} \begin{tabular}{|p{1cm}|p{1cm}|p{1cm}|p{1cm}|p{1cm}|p{1cm}|p{1cm}|}
            \hline & & & & & &\\ 
            \hline &$35_1$& $35_1$& & & &\\ \hline \end{tabular}
        \item \textbf{t=3:} \begin{tabular}{|p{1cm}|p{1cm}|p{1cm}|p{1cm}|p{1cm}|p{1cm}|p{1cm}|}
            \hline & & & & & &\\ 
            \hline $27_4$& & $35_1$&$35_1$ & & &\\ \hline \end{tabular}
        \item \textbf{t=4:} \begin{tabular}{|p{1cm}|p{1cm}|p{1cm}|p{1cm}|p{1cm}|p{1cm}|p{1cm}|}
            \hline & & & $27_4$& & &\\ 
            \hline & & & $35_1$& $35_1$& &\\ \hline \end{tabular}
        \item \textbf{t=5:} \begin{tabular}{|p{1cm}|p{1cm}|p{1cm}|p{1cm}|p{1cm}|p{1cm}|p{1cm}|}
            \hline & & & & & &\\ 
            \hline & & & & $35_1$& $35_1$&$27_4$\\ \hline \end{tabular} 
        \item \textbf{t=6:} \begin{tabular}{|p{1cm}|p{1cm}|p{1cm}|p{1cm}|p{1cm}|p{1cm}|p{1cm}|}
            \hline & & & & & &\\ 
            \hline $16_2$& & & & & $35_1$& $35_1$ \\ \hline \end{tabular} 
        \item \textbf{t=7:} \begin{tabular}{|p{1cm}|p{1cm}|p{1cm}|p{1cm}|p{1cm}|p{1cm}|p{1cm}|}
            \hline & & & & & &\\ 
            \hline & &$16_2$ & & & & $35_1$ \\ \hline \end{tabular} 
        \item \textbf{t=8:} \begin{tabular}{|p{1cm}|p{1cm}|p{1cm}|p{1cm}|p{1cm}|p{1cm}|p{1cm}|}
            \hline & & & & & &\\ 
            \hline & & & & $16_2$& &  \\ \hline \end{tabular}
        \item \textbf{t=9:} \begin{tabular}{|p{1cm}|p{1cm}|p{1cm}|p{1cm}|p{1cm}|p{1cm}|p{1cm}|}
            \hline & & & & & &\\ 
            \hline & & & & & & $16_2$ \\ \hline \end{tabular}
        \item \textbf{t=10:} \begin{tabular}{|p{1cm}|p{1cm}|p{1cm}|p{1cm}|p{1cm}|p{1cm}|p{1cm}|}
            \hline & & & & & &\\ 
            \hline & & & & & & \\ \hline \end{tabular}
    \end{itemize}
    \item Saída:
    \begin{itemize}
        \item \textbf{t=6:} $27$
        \item \textbf{t=8:} $35$
        \item \textbf{t=10:} $16$
    \end{itemize}
\end{itemize}


\item Mateus, um engenheiro novato, está desenvolvendo uma notação posicional original
para representação de números inteiros. Ele chamou esta notação de UMC 
(Um método curioso). A notação UMC usa os mesmos dígitos da notação decimal,
isto é, de 0 a 9.
%
Para converter um número $A$ da notação UMC para a notação decimal deve-se
adicionar $K$ termos, onde $K$ é o número de dígitos de $A$ (na notação UMC).
O valor do $i$-ésimo termo correspondente ao $i$-ésimo dígito $a_i$, contando
da direita para a esquerda é $a_i \times i!$. 

Por exemplo, 719$_{UMC}$ é equivalente a 53$_{10}$, pois
$7 \times 3! + 1 \times 2! + 9 \times 1! = 53$.

Mateus está começando seus estudos em teoria dos números e 
provavelmente não sabe quais as propriedades que um sistema de numeração
deve ter, mas neste momento ele está interessado em converter
os números da notação UCM para a notação decimal. Você pode ajudá-lo?

\noindent
\textbf{Entrada}:
cada caso de teste é fornecido em uma linha simples que contém um 
número não vazio de no máximo 5 dígitos, representando um número 
em notação UMC. Este número não contém zeros a esquerda. O último
teste é sequido por uma linha contendo um zero.

\noindent
\textbf{Saída}:
para cada caso de teste imprimir uma linha simples contendo a representação
em notação decimal do correspondente número em UMC seguido do cálculo feito
para a conversão.

\noindent
\textbf{O programa}:
seu programa deve, para cada número da entrada, convertê-lo em um vetor
de inteiros, sendo que cada dígito do número é um elemento do vetor,
e fazer os cálculos usando este vetor. 

\noindent
Exemplos de entrada e saída:
\begin{verbatim}
ENTRADA                SAÍDA
719                    53 = 7 x 3! + 1 x 2! + 9 x 1!
1                      1 = 1 x 1!
15                     7 = 1 x 2! + 5 x 1!
110                    8 = 1 x 3! + 1 x 2! + 0 x 1!
102                    8 = 1 x 3! + 0 x 2! + 2 x 1!
0 
\end{verbatim}


\item Sabemos que nos compiladores  mais recentes, nos
  quais existe o tipo \textsf{string}, podemos realizar de maneira simples
   operações com  palavras.  Imagine, no entanto, que  estamos usando um
  compilador  \emph{Pascal}  no  qual  não  existe  este  tipo.  Neste  caso  o
  programador deve  implementar por  sua própria conta  os procedimentos
  com palavras. Neste exercício iremos considerar a seguinte 
  declaração alternativa para o tipo \textsf{string}: \\
\begin{center}
\verb+type palavra = array[1..MaxTam] of char;+ 
\end{center}
  Implemente uma função em \emph{Pascal} que receba
  como parâmetros duas variáveis do tipo \textsf{MeuString} e retorne -1 se a
  primeira palavra for lexicograficamente menor que a segunda, 0 se
  forem iguais, e +1 no caso que resta.



\item Faça um programa que leia um certo número indefinido de
vetores e que imprima o vetor original (O) e um vetor gerado (G)
após um processo de compactação
que consiste na eliminação de todos os elementos repetidos
em cada vetor. Considere que a entrada de dados é feita em
um vetor por linha, sendo que o primeiro elemento da linha é
o tamanho de cada vetor e os elementos restantes da linha são
os elementos do vetor. Quando o tamanho for zero significa que 
terminou a entrada de dados. Por exemplo, considere a seguinte
entrada:

\begin{verbatim}
5 2 4 7 -1 2
3 1 1 1 
7 3 4 5 3 4 5 1
0
\end{verbatim}

Deverá produzir como saída o seguinte:

\begin{verbatim}
O: 2 4 7 -1 2
G: 2 4 7 -1
O: 1 1 1
G: 1
O: 3 4 5 3 4 5 1
G: 3 4 5 1
\end{verbatim}

\item Considere uma sequência de dígitos binários como:

011100011

Uma maneira de criptografar essa sequência de bits é adicionar
à cada dígito a soma dos seus dígitos adjacentes. Por exemplo,
a sequência acima se tornaria:

123210122

Se $P$ é a sequência original e $Q$ é a sequência criptografada,
então $Q[i] = P[i-1] + P[i] + P[i+1]$ para todas as posições $i$
da sequência. Considerando uma sequência de tamanho $n$ e seus
índices variando de $0$ a $n-1$, os dígitos $P[-1]$ e $P[n]$ não
fazem parte da sequência original e são tratados como zeros na
operação de codificação.

Assumindo $P[0] = 0$ temos:
\begin{itemize}
\item
$Q[0] = P[0] + P[1] = 0 + P[1] = 1$, logo $P[1] = 1$.
\item
$Q[1] = P[0] + P[1] + P[2] = 0 + 1 + P[2] = 2$, logo $P[2] = 1$.
\item
$Q[2] = P[1] + P[2] + P[3] = 1 + 1 + P[3] = 3$, logo $P[3] = 1$.
\item
Repetindo a operação temos: $P[4] = 0$, $P[5] = 0$, $P[6] = 0$, $P[7] = 1$ e $P[8] = 1$.
\end{itemize}

Agora repetindo o mesmo processo para $P[0] = 1$ temos:
\begin{itemize}
\item
$Q[0] = P[0] + P[1] = 1 + P[1] = 1$, logo $P[1] = 0$.
\item
$Q[1] = P[0] + P[1] + P[2] = 1 + 0 + P[2] = 2$, logo $P[2] = 1$.
\item
$Q[2] = P[1] + P[2] + P[3] = 0 + 1 + P[3] = 3$, o que nos leva a conclusão que $P[3] = 2$.
Entretanto isso viola o fato da sequência original ser binária. Portanto não
existe uma decodificação possível considerando o primeiro dígito da sequência original
valendo $1$.
\end{itemize}
Note que este algoritmo pode gerar ou decodificar uma sequência criptografada em até duas 
possíveis sequências originais, uma iniciando com 0 e outra iniciando com 1.

Escreva um procedimento em  que receba como parâmetros um vetor de
números inteiros contendo a sequência criptografada e a decodifica em dois outros
vetores de números inteiros. Caso uma das decodificações não seja possível, como
no caso do exemplo para $P[0] = 1$, o vetor correspondente deve ser preenchido com -1
na posição inicial.

Outros exemplos:
\begin{itemize}
\item
$123210122 = 011100011, -1$
\item
$11 = 01, 10$
\item
$22111 = -1, 11001$
\item
$123210120 = -1, -1$
\item
$3 = -1, -1$
\item
$12221112222221112221111111112221111 = \\
01101001101101001101001001001101001, \\
10110010110110010110010010010110010$
\end{itemize}

\item Escrever um programa para ler um texto e imprimir uma distribuição de
   frequências para  palavras do texto  (quantas palavras de  uma letra,
   quantas de duas letras, etc.).


\item Escreva um programa em \emph{Pascal} que leia do teclado o gabarito de uma prova
de 20 questões de múltipla escolha, onde as respostas são inteiros de 1 a
5.  Em seguida, o programa deve ler o número de alunos que prestaram a prova
e, para cada aluno, a sua matrícula (um inteiro) e as respectivas respostas.
O programa deve calcular e escrever:
\begin{itemize}
\item a relação de alunos ordenados pela nota, supondo que cada questão vale 5 pontos;
\item para cada questão: quantos alunos acertaram a questão
\end{itemize}




\end{enumerate}
