\section{Exercícios}

\begin{enumerate}

\item Resolva o problema dos \emph{Vingadores}, que está nesta URL: \\
      \url{http://www.inf.ufpr.br/cursos/ci055/tad_conjunto/enunciado_tad.pdf}

\item Resolva o problema do \emph {Jogo da Vida}, que está nesta URL: \\
      \url{http://www.inf.ufpr.br/cursos/ci055/Util/vida/vida.html}

\item Fazendo uso das boas técnicas  de programação vistas durante o curso,
 faça um programa em  \emph{Pascal} que implemente  um \emph{jogo  da velha}:
 \begin{itemize}

  \item O jogo possui um tabuleiro composto de nove posições, na forma
  de uma matriz de  tamanho 3 por 3;  cada posição pode estar vazia ou
  pode ser ocupada pelo símbolo de um dos jogadores.

  \item Dois jogadores participam do jogo, sendo  que a cada um destes
  é associado um símbolo distinto, por exemplo: ``X'' e ``0''.

  \item A primeira jogada é efetuada pelo jogador X; em cada jogada um
  dos jogadores ocupa uma posição vazia  do tabuleiro; os jogadores se
  alternam a cada jogada.

  \item  Um dos jogadores vence  quando ocupa uma posição que completa
  uma seqüência   de  três símbolos   iguais em  uma  linha, coluna ou
  diagonal.

  \item O jogo termina empatado  quando todas as posições do tabuleiro
  foram ocupadas e não houve vencedor.
 \end{itemize}

\item Considere um jogo de Batalha Naval em que cada participante disporá seus 5 
barcos de 1, 2, 3, 4 e 5 células, respectivamente, no espaço de uma matriz 
de $N\times N (N \ge 100)$. 

Os barcos deverão estar ou na direção horizontal ou na vertical, deverão 
ser retos e não poderão se tocar. Não poderá haver barcos que passem pelas 
linhas e colunas marginais da matriz.

Escreva o programa principal para montar um jogo e fazer uma disputa entre 
dois adversários, especificando as chamadas às diferentes funções e 
procedimentos.

Escrever as funções e procedimentos necessários para o bom funcionamento do
seu programa principal acima.

Você deve documentar a lógica da solução de forma precisa. Em particular, 
descreva as estruturas de dados que você utilizar e a forma como elas serão 
usadas para resolver o problema.

%\item  implementar o caça-palavras
%
%\item  implementar o Sudoku
%
%\item implementar a TV da vovó
%
%\item implementar a espiral.

\end{enumerate}
