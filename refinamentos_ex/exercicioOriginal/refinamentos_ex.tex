\section{Exercícios}


\begin{enumerate}

\item Fazer um programa em \emph{Pascal} que leia do teclado dois números inteiros positivos e que imprima na saída um único número inteiro
que é a soma dos dois primeiros. Entretanto, seu programa não pode utilizar o operador de soma (+) da linguagem \emph{Pascal} para somar os dois inteiros lidos em uma única operação. Outrossim, o programa deve implementar a soma dos números dígito a dígito, iniciando pelo menos significativo até o mais significativo, considerando o ``vai um'', conforme costumamos
fazer manualmente desde o ensino fundamental.

\begin{verbatim}
Exemplo 1                    Exemplo 2
 11   ("vai um")              1111   ("vai um")
40912 (primeiro número)        52986 (primeiro número)
 1093 (segundo número)       1058021 (segundo número)
-----                        -------
42005 (soma)                 1111007 (soma)
\end{verbatim}

\item Um agricultor possui 1 (uma) espiga de milho. Cada espiga tem 150 grãos,
e cada grão pesa 1g (um grama). Escreva um programa em \emph{Pascal} para
determinar quantos anos serão necessários para que o agricultor colha
mais de cem toneladas de milho (1T = 1000Kg, 1Kg = 1000g), sendo que:

\begin{itemize}
\item A cada ano ele planta todos os grãos da colheita anterior
\item Há uma única colheita por ano
\item 10\% (dez por cento) dos grãos não germina (morre sem produzir)
\item Cada grão que germina produz duas espigas de milho
\end{itemize}

Assuma que a quantidade de terra disponível é sempre suficiente para o
plantio.

\item Modifique a questão anterior acrescentando na simulação os seguintes fatos:

\begin{itemize}
\item  Há 8 (oito) CASAIS de pombas (16 pombas) que moram na propriedade
       do agricultor.
\item Cada pomba come 30 grãos por dia, durante os 30 dias do ano em que
as espigas estão formadas antes da colheita;
\item A cada ano, cada casal gera 2 novos casais (4 pombas), que se
alimentarão e reproduzirão no ano seguinte;
\item Uma pomba vive tres anos;
\end{itemize}

Ao final do programa, imprima também o número de pombas que vivem na
propriedade quando o agricultor colher mais de 100T de milho

\item Considere um número inteiro com 9 dígitos. Suponha que o  último dígito seja o ``dígito verificador''
 do número formado pelos 8 primeiros. Faça um programa em \emph{Pascal} que leia uma massa de dados terminada por 0 (zero) e que
imprima os números que não são bem formados, isto é, aqueles que não satisfazem o dígito
verificador. Implemente o seguinte algoritmo para gerar o dígito verificador:

Conforme o esquema abaixo, cada dígito do número, começando da direita para a esquerda
(menos significativo para o mais significativo) é multiplicado, na ordem, por 2, depois 1,
depois 2, depois 1 e assim sucessivamente.

Número exemplo: 261533-4

\begin{center}
\begin{small}
\begin{verbatim}

  +---+---+---+---+---+---+   +---+
  | 2 | 6 | 1 | 5 | 3 | 3 | - | 4 |
  +---+---+---+---+---+---+   +---+
    |   |   |   |   |   |
   x1  x2  x1  x2  x1  x2
    |   |   |   |   |   |
   =2 =12  =1 =10  =3  =6
    +---+---+---+---+---+-> = (16 / 10) = 1, resto 6 => DV = (10 - 6) = 4 

\end{verbatim}
\end{small}
\end{center}

Ao invés de ser feita a somatória das multiplicações, será feita a somatória dos dígitos das multiplicações
(se uma multiplicação der 12, por exemplo, será somado 1 + 2 = 3).

A somatória será dividida por 10 e se o resto (módulo 10) for diferente de zero, o dígito será 10 menos este valor.

\item 
Escreva um programa \emph{Pascal} que leia dois valores inteiros positivos A e B.
Se A for igual a B, devem ser lidos novos valores até que sejam informados
valores distintos.  Se A for menor que B, o programa deve calcular e
escrever a soma dos números ímpares existentes entre A(inclusive) e
B(inclusive).  Se A for maior que B, o programa deve calcular e escrever a
média aritmética dos múltiplos de 3 existentes entre A(inclusive) e
B(inclusive).


\item 
Faça um  programa em \emph{Pascal} que dado
 uma sequência de números inteiros  terminada por zero (0), determinar
 quantos segmentos    de  números iguais consecutivos     compõem essa
 sequência.

Ex.: A sequência 2,2,3,3,5,1,1,1 é  formada por 4 segmentos de números
iguais.


\item Faça um programa em \emph{Pascal} que imprima a seguinte
sequência de números: 1, 1, 2, 2, 3, 3, 3, 4, 4, 4, 5, 5, 5, 5, 6, 6, 6, 6, 
7, 7, 7, 7, 7, 8, 8, 8, 8, 8, \ldots

\item Faça um programa em \emph{Pascal} que 
receba como entrada um dado 
inteiro $N$ e o imprima como um produto de primos. Exemplos:
$45 = 3 \times 3 \times 5$. $56 = 2 \times 2 \times 2 \times 7$.

\item Faça um programa em \emph{Pascal} que, dado
um número inteiro $N$, escreva o maior divisor de $N$ que é uma potência
de um dos números primos fatorados. Ex:

\hspace{2cm} $N=45 = 3^2.5^1 $ escreve $9=3^2$

\hspace{2cm} $N=145 = 5^2.7^1 $ escreve $25=5^2$

\hspace{2cm} $N=5616 = 2^4.3^3.13 $ escreve $27=3^3$

\end{enumerate}
