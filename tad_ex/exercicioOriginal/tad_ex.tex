\section{Exercícios}

\begin{enumerate}

\item Defina um Tipo Abstrato de Dados \emph{racional} que vai procurar
      abstrair números da forma $\frac{A}{B}$, onde $A$ e$B$ são
      números inteiros. Em seguida implemente as funções e procedimentos
      necessários para realizar as operações abaixo de maneira que os
      resultados sejam sempre simplificados, isto é, o resultado final
      deve indicar $\frac{3}{2}$ e não $\frac{6}{4}$, por exemplo. 
      Suas implementações não podem permitir que ocorram divisões por zero.
      \begin{enumerate}
          \item adição de duas frações;
          \item subtração de duas frações;
          \item multiplicação de duas frações;
          \item divisão de duas frações.
      \end{enumerate}

\item Sem olhar os códigos escritos neste capítulo, implemente sua 
      própria biblioteca que implementa o TAD pilha. Os algoritmos
      que estão ali são triviais mas servem para ver sua prática em 
      programação.

\item Sem olhar os códigos escritos neste capítulo, implemente sua 
      própria biblioteca que implementa o TAD conjunto. Alguns dos algoritmos
      que estão ali \emph{não} são triviais e é um ótimo exercício de 
      programação.

\item Resolva o exercício que está na seguinte URL: \\
\url{http://www.inf.ufpr.br/cursos/ci055/tad_conjunto/enunciado_tad.pdf}

\item Resolva o exercício que está na seguinte URL: \\
\url{http://www.inf.ufpr.br/cursos/ci055/Provas_antigas/final-20191.pdf}

\item Resolva o exercício que está na seguinte URL: \\
\url{http://www.inf.ufpr.br/cursos/ci055/Provas_antigas/p3-20191.pdf}

\item Considere um Tipo Abstrato de Dados \emph{Fila}, que define
uma estrutura na qual o primeiro que entra é o primeiro que sai
(FIFO, do inglês \emph{First In, First Out}). Pense em uma fila
de uma padaria, por exemplo. As principais operações que manipulam
filas são as seguintes:

\begin{itemize}
   \item criar uma fila (vazia);
   \item inserir elementos;
   \item remover elementos;
   \item retornar o tamanho da fila;
   \item saber se a fila está vazia/cheia;
   \item imprimir a fila como está;
   \item retornar o elemento do início da fila, removendo-o;
   \item retornar o elemento do início da fila, sem removê-lo.
\end{itemize}

\item Considere um Tipo Abstrato de Dados \emph{lista}, que define
uma estrutura que pode conter elementos em qualquer ordem ou mesmo
repetidos. Pense por exemplo em uma lista de compras ou de afazeres.
As principais operações que manipulam listas são as seguintes:
\begin{itemize}
   \item criar uma lista (vazia);
   \item inserir no início;
   \item inserir no fim;
   \item inserir na posição p;
   \item remover (do início, do fim, da posição p);
   \item retornar o tamanho da lista;
   \item saber se a lista está vazia/cheia;
   \item imprimir a lista como está;
   \item imprimir a lista ordenada;
   \item fundir duas listas;
   \item intercalar duas listas;
   \item pesquisar o elemento da posição p na lista;
   \item copiar uma lista em outra;
\end{itemize}

\item Implemente um programa que encontra bilhetes premiados do jogo da mega-sena, usando a sua implementação do tipo lista acima definido.

\item Defina o Tipo Abstrato de Dados \textsf{polinomio}. A estrutura tem
que ser capaz de armazenar o grau e os coeficientes de um polinômio
$a_0 + a_1x + a_2x^2 + \ldots + a_nx^n$. Implemente o conjunto de
operações abaixo na forma de funções e procedimentos:

\begin{enumerate}
\item inicializar a estrutura polinômio;
\item ler um polinômio de grau $n$;
\item dado $x \in R$, calcular o valor de $P(x)$;
\item obter o polinômio derivada de $P$, $P'$.
\item dado $x \in R$, calcular o valor de $P'(x)$;
\item obter o polinômio soma de $P$ e $Q$;
\item obter o polinômio multiplicação de $P$ e $Q$;
\end{enumerate}

Faça um programa em \emph{Pascal} que utilize o tipo abstrato de dados
definido, leia dois polinômios $p$ e $q$, calcule
o produto $r$ de $p$ e $q$, imprima o polinômio resultante, 
leia um certo número real $x$, calcule o valor de $r(x)$ e o imprima. 

\end{enumerate}
