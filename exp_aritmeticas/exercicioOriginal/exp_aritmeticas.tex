\item Considere o seguinte programa incompleto em \emph{Pascal}:

\begin{lstlisting}
program tipos;
var 
     A: <tipo>;
     B: <tipo>;
     C: <tipo>;
     D: <tipo>;
     E: <tipo>;
begin
     A := 1 + 2 * 3;
     B := 1 + 2 * 3 / 7;
     C := 1 + 2 * 3 div 7;
     D := 3 div 3 * 4.0;
     E := A + B * C - D
end.
\end{lstlisting}

Você deve completar este programa indicando, para cada variável de $A$ até
$E$, qual é o tipo correto desta variável. Algumas delas podem ser tanto inteiras 
como reais, enquanto que algumas só podem ser de um tipo específico.
Para resolver este exercício você precisa estudar sobre os operadores
inteiros e reais e também sobre a ordem de precedência de operadores
que aparecem em uma expressão aritimética. Sua solução estará correta se seu 
programa compilar.

\item Escreva um programa em \emph{Pascal} que leia 
6 valores reais para as variáveis $A, B, C, D, E, F$ e 
imprima o valor de $X$ após o cálculo 
 da seguinte expressão aritmética:

\[ 
X = \frac{\frac{A + B}{C - D}E}{\frac{F}{AB} + E}
\]

Seu programa deve assumir que nunca haverá divisões por zero
para as variáveis dadas como entrada. Note que neste programa
a variável $X$ deve ser do tipo \emph{real}, enquanto que 
as outras variáveis podem ser tanto da família \emph{ordinal}
(\emph{integer, longint, etc}) como também podem ser do tipo
\emph{real}.

\begin{center}
\begin{tabular}{|l|l|} \hline
Exemplo Entrada & Saída esperada \\ \hline
1 2 3 4 5 6     & -1.8750000000000000E+000  \\ \hline
1 -1 1 -1 1 -1  & 0.0000000000000000E+000   \\ \hline
3 5 8 1 1 2     & 1.0084033613445378E+000   \\ \hline
\end{tabular}
\end{center}

\item Escreva em \emph{Pascal} as seguintes expressões
aritméticas usando o mínimo possível de parênteses.
Para resolver este exercício você precisa estudar sobre precedência
de operadores em uma expressão aritmética. Dica: para elevar um número
ao quadrado multiplique este número por ele mesmo ($x^2 = x * x$).

\begin{enumerate}
\item 
     \[ \frac{W^2}{Ax^2 + Bx +C} \]

\item 
     \[ \frac{\frac{P_1 + P_2}{Y - Z}R}{\frac{W}{AB} + R} \]
\end{enumerate}

\noindent
Observe que os compiladores não suportam o uso de subescritos, que são
utilizados na notação matemática. Então no lugar de $P_1$ e $P_2$, você
pode dar os nomes para as variáveis de $p1$ e $p2$ respectivamente.

\item Faça um programa em \emph{Pascal}
que some duas horas. A entrada deve ser feita lendo-se
dois inteiros por linha, em duas linhas, e a saída
deve ser feita no formato especificado no exemplo
abaixo:

\begin{center}
\begin{tabular}{|l|l|} \hline
Exemplo Entrada & Saída esperada \\ \hline
12 52           &                \\ 
7 13            & 12:52 + 7:13 = 20:05 \\ \hline
20 15           &                \\ 
1 45            & 20:15 + 1:45 = 22:00  \\ \hline
0 0             &                \\ 
8 35            & 0:0 + 8:35 = 8:35  \\ \hline
\end{tabular}
\end{center}

Você deve observar que o comando de impressão deve imprimir os espaços em branco e
os símbolos ``+'' e ``='' conforme o enunciado exige.

\item Dado um número inteiro que representa uma quantidade de segundos,
   determinar o seu valor equivalente em graus, minutos e segundos. Se
   a quantidade de segundos for insuficiente para dar um valor em graus,
   o valor em graus deve ser 0 (zero). A mesma observação vale em
   relação aos minutos e segundos. 

\begin{center}
\begin{tabular}{|l|l|} \hline
Exemplo Entrada & Saída esperada \\ \hline
3600            & 1, 0, 0        \\ \hline
3500            & 0, 58, 20      \\ \hline
7220            & 2, 0, 20       \\ \hline
\end{tabular}
\end{center}

\item Fazer um programa em \emph{Pascal} que troque o conteúdo de duas 
variáveis. Exemplo:

\begin{center}
\begin{tabular}{|l|l|} \hline
Exemplo Entrada & Saída esperada \\ \hline
3 7             & 7 3            \\ \hline
-5 15           & 15 -5          \\ \hline
2 10            & 10 2           \\ \hline
\end{tabular}
\end{center}

\item (*) Desafio:
Fazer um programa em \emph{Pascal}
que troque o conteúdo de duas variáveis \emph{inteiras}
sem utilizar variáveis auxiliares. Pense em fazer contas
de adição e/ou subtração com os números. 


