\item  Aqui temos uma forma peculiar de realizar uma multiplicação entre
dois números: multiplique o primeiro por 2 e divida o segundo por 2
até que o primeiro seja reduzido a 1. Toda vez que o primeiro for
impar, lembre-se do segundo. Não considere qualquer fração durante
o processo. O produto dos dois números é igual a soma dos números
que foram lembrados. Exemplo: $53 \times 26 =$

\begin {tabbing}
00012345\=10002345\=10002345\=10002345\=12000345\=12000345\=12000345\=12300045\=12345\=12345\=12345 \kill
53 \> 26 \> 13 \> 6 \> 3 \> 1 \\
26 \> 52 \> 104 \> 208 \> 416 \> 832 \\
\\
26 + \> \> 104 + \> \> 416 + \> 832 = 1378
\end {tabbing}

Fazer um programa em \emph{Pascal}
que receba dois números inteiros e retorne o produto deles
do modo como foi especificado acima.

\begin{minipage}{5cm}
\begin{verbatim}
Entrada:
53
26
\end{verbatim}
\end{minipage} \
\begin{minipage}{5cm}
\begin{verbatim}
Saída:
1378
\end{verbatim}
\end{minipage}