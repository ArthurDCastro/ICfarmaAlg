\item (*) Escrever um programa em \emph{Pascal} que leia do teclado
uma sequência de números inteiros até que seja lido um número
que seja o dobro ou a metade do anteriormente lido.
O programa deve imprimir na saída os seguintes valores:
\begin{itemize}
\item a quantidade de números lidos;
\item a soma dos números lidos;
\item os dois valores lidos que forçaram a parada do programa.
\end{itemize}

Exemplo 1:
\begin{verbatim}
Entrada:
-549 -716 -603 -545 -424 -848
Saída:
6 -3685 -424 -848
\end{verbatim}

Exemplo 2:
\begin{verbatim}
Entrada
-549 -716 -603 -545 -424 646 438 892 964 384 192
Saída
11 679 384 192
\end{verbatim}