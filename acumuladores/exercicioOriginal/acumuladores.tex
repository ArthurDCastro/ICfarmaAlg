
\item Fazer um programa em \emph{Pascal} para ler do teclado
um número inteiro $m$ e em seguida
uma sequência de $m$ números reais e imprimir a média aritmética deles.
Isto é, dados os números $N_1, N_2, \ldots, N_m$, calcular:

\[
\frac{N_1 + N_2 + \ldots + N_m}{m}
\]

\item Fazer um programa em \emph{Pascal} para calcular
o produto dos números ímpares de $A$ até $B$,
onde $A \le B$ são lidos do teclado. Considere que $A$ e $B$ são
sempre ímpares. Isto é, calcular:

\[
A \times (A+2) \times (A+4) \times \ldots \times B
\]

\item Fazer um programa em \emph{Pascal} para
     calcular o  valor da  soma dos  quadrados dos  primeiros  50 inteiros
     positivos não nulos.

\[
\sum_{i=1}^{50}{i^2} = 1^2 + 2^2 + 3^2 + \ldots + 50^2
\]

\item Ler um inteiro positivo N diferente de zero e calcular a soma:
   $1^{3} + 2^{3} + ... + N^{3}$.

\item Fazer um programa em \emph{Pascal} que,
    dados dois números inteiros positivos, determine quantas vezes o
    primeiro  divide exatamente  o segundo.  Se o  primeiro não  divide o
    segundo o número de vezes é zero. Por exemplo, 72 pode ser dividido
    exatamente por 3 duas vezes.

\begin{minipage}{5cm}
\begin{verbatim}
Entrada:
72 3
\end{verbatim}
\end{minipage} \
\begin{minipage}{5cm}
\begin{verbatim}
Saída:
2
\end{verbatim}
\end{minipage}

\item
Uma agência governamental deseja conhecer a distribuição da população do
país por faixa salarial. Para isto, coletou dados do
último censo realizado e criou um arquivo contendo, em cada linha, a
idade de um cidadão particular e seu salário. As idades variam de zero a
110 e os salários variam de zero a 19.000,00 unidades da moeda local
(salário do seu dirigente máximo). Considere o salário mínimo igual a 450,00 unidades da moeda local.

As faixas salariais de interesse são as seguintes:
\begin{itemize}
\item de 0 a 3 salários mínimos
\item de 4 a 9 salários mínimos
\item de 10 a 20 salários mínimos
\item acima de 20 salários mínimos.
\end{itemize}
Fazer um programa em \emph{Pascal} que leia o arquivo de entrada e produza como
saída os percentuais da população para cada faixa salarial
de interesse. A última linha, que não deve ser processada, contém dois zeros.

\begin{minipage}{5cm}
\begin{verbatim}
Entrada:
25 240.99
48 2720.77
37 4560.88
34 19843.33
23 834.15
90 315.87
78 5645.80
44 150.33
56 2560.00
67 2490.05
0 0.00  
\end{verbatim}
\end{minipage} \
\begin{minipage}{5cm}
\begin{verbatim}
Saída:
4%
3%
2%
1%
\end{verbatim}
\end{minipage}

\item (*) Escrever um programa em \emph{Pascal} que leia do teclado
uma sequência de números inteiros até que seja lido um número
que seja o dobro ou a metade do anteriormente lido.
O programa deve imprimir na saída os seguintes valores:
\begin{itemize}
\item a quantidade de números lidos;
\item a soma dos números lidos;
\item os dois valores lidos que forçaram a parada do programa.
\end{itemize}

Exemplo 1:
\begin{verbatim}
Entrada:
-549 -716 -603 -545 -424 -848
Saída:
6 -3685 -424 -848
\end{verbatim}

Exemplo 2:
\begin{verbatim}
Entrada
-549 -716 -603 -545 -424 646 438 892 964 384 192
Saída
11 679 384 192
\end{verbatim}

\item  Aqui temos uma forma peculiar de realizar uma multiplicação entre
dois números: multiplique o primeiro por 2 e divida o segundo por 2
até que o primeiro seja reduzido a 1. Toda vez que o primeiro for
impar, lembre-se do segundo. Não considere qualquer fração durante
o processo. O produto dos dois números é igual a soma dos números
que foram lembrados. Exemplo: $53 \times 26 =$

\begin {tabbing}
00012345\=10002345\=10002345\=10002345\=12000345\=12000345\=12000345\=12300045\=12345\=12345\=12345 \kill
53 \> 26 \> 13 \> 6 \> 3 \> 1 \\
26 \> 52 \> 104 \> 208 \> 416 \> 832 \\
\\
26 + \> \> 104 + \> \> 416 + \> 832 = 1378
\end {tabbing}

Fazer um programa em \emph{Pascal}
que receba dois números inteiros e retorne o produto deles
do modo como foi especificado acima.

\begin{minipage}{5cm}
\begin{verbatim}
Entrada:
53
26
\end{verbatim}
\end{minipage} \
\begin{minipage}{5cm}
\begin{verbatim}
Saída:
1378
\end{verbatim}
\end{minipage}

\item Fazer um programa em \emph{Pascal} para
    ler  uma massa de dados onde  cada linha contém dois  valores numéricos
    sendo o primeiro do tipo real e o segundo do tipo inteiro. O segundo
    valor é o peso atribuído ao primeiro valor. O
    programa deve calcular a média ponderada dos diversos valores lidos.
    A última
    linha de  dados contém os  únicos números zero.  Esta linha não  deve ser
    considerada no cálculo da média. Isto é, calcular o seguinte, supondo
    que $m$ linhas foram digitados:

\[
\frac{N_1 \times P_1 + N_2 \times P_2 + \ldots + N_m \times P_m}{P_1 + 
 P_2 + \ldots P_m}
\]
\begin{minipage}{5cm}
\begin{verbatim}
Entrada:
60 1
30 2
40 3
0 0
\end{verbatim}
\end{minipage} \
\begin{minipage}{5cm}
\begin{verbatim}
Saída:
40
\end{verbatim}
\end{minipage}


