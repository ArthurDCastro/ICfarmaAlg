\item
Uma agência governamental deseja conhecer a distribuição da população do
país por faixa salarial. Para isto, coletou dados do
último censo realizado e criou um arquivo contendo, em cada linha, a
idade de um cidadão particular e seu salário. As idades variam de zero a
110 e os salários variam de zero a 19.000,00 unidades da moeda local
(salário do seu dirigente máximo). Considere o salário mínimo igual a 450,00 unidades da moeda local.

As faixas salariais de interesse são as seguintes:
\begin{itemize}
\item de 0 a 3 salários mínimos
\item de 4 a 9 salários mínimos
\item de 10 a 20 salários mínimos
\item acima de 20 salários mínimos.
\end{itemize}
Fazer um programa em \emph{Pascal} que leia o arquivo de entrada e produza como
saída os percentuais da população para cada faixa salarial
de interesse. A última linha, que não deve ser processada, contém dois zeros.

\begin{minipage}{5cm}
\begin{verbatim}
Entrada:
25 240.99
48 2720.77
37 4560.88
34 19843.33
23 834.15
90 315.87
78 5645.80
44 150.33
56 2560.00
67 2490.05
0 0.00  
\end{verbatim}
\end{minipage} \
\begin{minipage}{5cm}
\begin{verbatim}
Saída:
4%
3%
2%
1%
\end{verbatim}
\end{minipage}