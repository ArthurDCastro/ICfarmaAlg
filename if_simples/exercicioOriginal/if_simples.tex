
\item Faça um programa em \emph{Pascal} que leia um número $n$ do teclado
e decida se ele é positivo ou negativo. Seu programa deve imprimir
a mensagem ``par'' ou ``impar'' conforme o caso. Exemplo:

\begin{center}
\begin{tabular}{|l|l|} \hline
Exemplo Entrada & Saída esperada \\ \hline
5                & impar         \\ \hline
4                & par           \\ \hline
15               & impar         \\ \hline
\end{tabular}
\end{center}

\item Faça um programa em \emph{Pascal} que leia dois números $n, m$ do teclado
e decida se ele o primeiro é maior do que o segundo. Seu programa deve imprimir
a mensagem ``primeiro eh maior'' ou ``segundo eh maior ou igual'' conforme o caso. Exemplo:

\begin{center}
\begin{tabular}{|l|l|} \hline
Exemplo Entrada & Saída esperada \\ \hline
5 2             & primeiro eh maior               \\ \hline
2 5             & segundo eh maior ou igual       \\ \hline
5 5             & segundo eh maior ou igual       \\ \hline
\end{tabular}
\end{center}

\item Faça um programa em \emph{Pascal} que leia três números $x, y, z$ 
do teclado e decida se $x \le y < z$. 
Seu programa deve imprimir a mensagem ``esta no intervalo'' 
ou ``nao esta no intervalo'' conforme o caso. Exemplo:

\begin{center}
\begin{tabular}{|l|l|} \hline
Exemplo Entrada & Saída esperada \\ \hline
3 5 8           & esta no intervalo \\ \hline
3 8 8           & nao esta no intervalo \\ \hline
4 12 5          & nao esta no intervalo \\ \hline
\end{tabular}
\end{center}

\item Faça um programa em \emph{Pascal} que leia três números $x, y, z$ 
do teclado e decida se $x > y$ ou se $y < z$. 
Seu programa deve imprimir a mensagem ``sim'' em caso afirmativo e  
``nao'' caso contrário. Exemplo:

\begin{center}
\begin{tabular}{|l|l|} \hline
Exemplo Entrada & Saída esperada \\ \hline
3 5 8                & sim                \\ \hline
3 8 8                & nao               \\ \hline
4 12 5               & nao               \\ \hline
\end{tabular}
\end{center}

\item Escreva um programa em \emph{Pascal} que leia
6 valores reais para as variáveis $A, B, C, D, E, F$ e
imprima o valor de $X$ após o cálculo
 da seguinte expressão aritmética:

     \[ X = \frac{\frac{A + B}{C - D}E}{\frac{F}{AB} + E} \]

Seu programa deve imprimir a mensagem ``divisao por zero''
caso o denominador seja zero. Caso isso não ocorra seu programa
irá abortar neste caso, o que não é correto.

Exemplos de entrada e saída:

\begin{center}
\begin{tabular}{|l|l|} \hline
Exemplo Entrada & Saída esperada \\ \hline
1 2 3 4 5 6     & -1.8750000000000000E+000  \\ \hline
0 0 0 0 0 0     & divisao por zero   \\ \hline
1 1 2 2 1 3     & divisao por zero   \\ \hline
\end{tabular}
\end{center}

