\section{Exercícios}


\begin{enumerate}

% atribuicoes dentro de repeticoes

\item Dados  dois números inteiros  positivos determinar o valor  da maior
    potência do primeiro que divide  o segundo. Se o primeiro não divide
    o segundo, a maior potência é definida como sendo igual a 1. Por exemplo, a
    maior potência de 3 que divide 45 é 9.

\begin{minipage}{5cm}
\begin{verbatim}
Entrada:
3 45
\end{verbatim}
\end{minipage} \
\begin{minipage}{5cm}
\begin{verbatim}
Saída:
9
\end{verbatim}
\end{minipage}


% laco dentro de if

\item Dadas as populações $P_A$ e $P_B$ de duas cidades $A$ e $B$ em 2009, e
suas respectivas taxas de crescimento anual $X_A$ e $X_B$, faça um
programa em \emph{Pascal} que receba estas informações como entrada e
determine:
\begin{itemize}
\item
se a população da cidade de menor população ultrapassará a de maior
população;
\item
e o ano em que isto ocorrerá.
\end{itemize}

% if dentro de while

\item Um inteiro positivo $N$ é perfeito se for igual a soma de seus
divisores positivos diferentes de $N$.

Exemplo: 6 é perfeito pois $1 + 2 + 3 = 6$ e $1, 2, 3$ são todos os
divisores positivos de 6 e que são diferentes de 6.

Faça um programa em \emph{Pascal} que recebe como entrada um número
positivo $K$ e mostre os $K$ primeiros números perfeitos.


\item Faça um  programa em \emph{Pascal} que dado um inteiro positivo $n$, escreva
todos os termos, do primeiro ao $n$-ésimo, da série abaixo.  Você pode
assumir que o usuário nunca digita valores menores que 1 para $n$.

\[ 5,6,11,12,17,18,23,24, \ldots\]

% laco com varios ifs aninhados

\item Faça um programa em \emph{Pascal} que, dada uma sequência de números naturais positivos terminada por $0$ (zero), imprimir o histograma da sequência dividido em quatro faixas (o histograma é a contagem do número de elementos em cada faixa):
\begin{itemize}
 \item Faixa 1: $1$ -- $100$;
 \item Faixa 2: $101$ -- $250$;
 \item Faixa 3: $251$ -- $20000$;
 \item Faixa 4: acima de $20001$.
\end{itemize}

Exemplo:
\begin{verbatim}
Entrada: 347 200 3 32000 400 10 20 25 0
Saída:  Faixa 1: 4
        Faixa 2: 1
        Faixa 3: 2
        Faixa 4: 1
\end{verbatim}

%%% laço duplo

\item Fazer um programa em \emph{Pascal} que leia uma sequência de
 pares de números inteiros quaisquer, sendo dois inteiros por linha de
entrada. A entrada de dados termina quando os dois números lidos forem nulos.
Este par de zeros não deve ser processado e serve para marcar o
término da entrada de dados.

Para cada par $A,B$ de números lidos, se $B$ for maior do que
$A$, imprimir a sequência $A,A+1, \ldots, B-1,B$. Caso contrário,
imprimir a sequência $B,B+1, \ldots,A-1,A$.

Exemplos:
\begin{verbatim}
Entrada      Saida
4 6          4 5 6
-2 1         -2 -1 0 1
2 -3         -3 -2 -1 0 1 2
0 0 
\end{verbatim}


\item  Fazer um programa em \emph{Pascal} que receba um
   número inteiro $N$ como entrada e imprima cinco linhas
contendo as seguintes somas, uma em cada linha:

\begin{verbatim}
   N
   N + N
   N + N + N
   N + N + N + N
   N + N + N + N + N
\end{verbatim}

Exemplo:

\begin{minipage}{5cm}
\begin{verbatim}
Entrada:
3
\end{verbatim}
\end{minipage} \
\begin{minipage}{5cm}
\begin{verbatim}
Saída:
   3
   6
   9
   12
   15
\end{verbatim}
\end{minipage}

\item Fazer um programa em \emph{Pascal} que imprima
  exatamente a saída especificada na figura 1 (abaixo) de maneira que,
  em todo o programa fonte, não apareçam mais do que três comandos de impressão.

\begin{center}
\begin{minipage}{5cm}
\begin{verbatim}
1
121
12321
1234321
123454321
12345654321
1234567654321
123456787654321
12345678987654321

    Figura 1
\end{verbatim}
\end{minipage}
\end{center}

\item Fazer um programa em {\emph Pascal} que imprima
  exatamente a mesma saída solicitada no exercício anterior, mas que
  use exatamente dois comandos de repetição.

\item Adaptar a solução do exercício anterior para que a saída seja
  exatamente conforme especificada na figura 2 (abaixo).

\begin{center}
\begin{minipage}{5cm}
\begin{verbatim}
        1
       121
      12321
     1234321
    123454321
   12345654321
  1234567654321
 123456787654321
12345678987654321

    Figura 2
\end{verbatim}
\end{minipage}
\end{center}

% laço duplo com if dentro

\item Leia do teclado uma sequência de  $N > 0$ números quaisquer. Para cada
valor lido, se ele for positivo, imprimir os primeiros 10 múltiplos
dele.

% estrutura complexa com varios whiles e ifs

\item Sabe-se que um número da forma $n^3$ é igual a soma de $n$ números
ímpares consecutivos.

Exemplos:
\begin{itemize}
\item
$1^3 = 1$
\item
$2^3 = 3 + 5$
\item
$3^3 = 7 + 9 + 11$
\item
$4^3 = 13 + 15 + 17 + 19$
\end{itemize}

Dado $M$, escreva um program em \emph{Pascal} que determine os ímpares
consecutivos cuja soma é igual a $n^3$ para $n$ assumindo valores de
1 a $M$.

\item Faça um programa em {\em  \emph{Pascal}} que, dados
dois números naturais  $m$ e $n$  determinar,  entre todos  os pares de
números naturais $(x,y)$ tais que $x<=m$ e $y<=n$, um  par para o qual
o valor da expressão $xy - x^2 + y$ seja máximo e calcular também esse
máximo.

\item (*) Escreva um programa em \emph{Pascal} para ler uma sequência de
números inteiros, terminada em $-1$. Para cada número inteiro lido, o programa
deve verificar se este número está na base binária, ou seja, se é composto
somente
pelos dígitos $0$ e $1$. Caso o número esteja na base binária, o programa deve
imprimir seu valor na base decimal. Caso contrário, deve imprimir uma
mensagem indicando que o número não é binário. Ao final do programa deve ser
impresso, em formato decimal, o maior número válido (binário) da sequência.

\vspace*{0.25cm}

\noindent Dica: dado o número $10011$ em base binária, seu valor correspondente
em base decimal será dado por
\begin{equation*}
1.2^4 + 0.2^3 + 0.2^2 + 1.2^1 + 1.2^0 = 19 
\end{equation*}


\vspace*{0.25cm}

\noindent {\bf Exemplo:}
\begin{verbatim}
Entrada:    Saida:
10011       19
121         numero nao binario
1010        10
101010101   341
0           0
-1

O maior numero binario foi 341
\end{verbatim}




\end{enumerate}

