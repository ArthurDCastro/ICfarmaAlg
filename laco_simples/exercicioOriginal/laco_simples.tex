
\item Fazer um programa em \emph{Pascal} para
    ler uma massa de dados onde cada  linha da entrada contém um número  par.
    Para
    cada número lido,  calcular o seu sucessor par,  imprimindo-os dois a
    dois em  listagem de saída. A  última linha de dados  contém o número
    zero, o qual não deve ser processado e serve apenas para indicar o final
    da leitura dos dados. Exemplo:

\begin{center}
\begin{tabular}{|l|l|} \hline
Exemplo Entrada & Saída esperada \\ \hline
12 6 26 86 0    & 12 14          \\ 
                & 6 8            \\ 
                & 26 28          \\ 
                & 86 88          \\ \hline
-2 -5 -1 0      & -2 0           \\
                & -5 -3          \\ 
                & -1 1           \\ \hline
1 2 3 4 5 0     & 1 3            \\
                & 2 4            \\
                & 3 5            \\
                & 4 6            \\
                & 5 7            \\ \hline
\end{tabular}
\end{center}

\item Fazer um programa em \emph{Pascal} para
    ler  uma massa de  dados contendo a  definição de várias  equações do
    segundo grau da forma $Ax^{2} + Bx + C = 0$. Cada linha de dados contém a
    definição de uma equação por meio dos valores de $A$, $B$ e $C$ do conjunto
    dos  números reais.  A última  linha  informada ao  sistema contém  3
    (três) valores  zero (exemplo  0.0 0.0 0.0).  Após a leitura  de cada
    linha o  programa deve tentar calcular  as duas raízes  da equação. A
    listagem de saída, em cada  linha, deverá conter
    os valores das duas raízes reais. Considere
    que o usuário entrará somente com valores $A$, $B$ e $C$ tais que a equação
    garantidamente tenha duas raízes reais. 

\begin{center}
\begin{tabular}{|l|l|} \hline
Exemplo Entrada & Saída esperada \\ \hline
1.00 -1.00 -6.00 & -3.00 2.00 \\
1.00 0.00 -1.00  & -1.00 1.00 \\
1.00 0.00 0.00   & 0.00 0.00  \\ 
0.00 0.00 0.00   &            \\ \hline
\end{tabular}
\end{center}

\item Fazer um programa em \emph{Pascal} que receba dois números inteiros $N$ e
  $M$ como entrada e retorne como saída $N \ mod \ M$ (o resto da
  divisão inteira de $N$ por $M$) usando para isto apenas
  operações de subtração. O seu programa
  deve considerar que o usuário entra com $N$ sempre maior do que $M$.

\begin{center}
\begin{tabular}{|l|l|} \hline
Exemplo Entrada & Saída esperada \\ \hline
30 7            & 2              \\ \hline
3 2             & 1              \\ \hline
12 3            & 0              \\ \hline
\end{tabular}
\end{center}

\item Fazer um programa em \emph{Pascal} que leia um número $n > 0$ do
teclado e imprima a tabuada de $n$ de 1 até 10.

\begin{center}
\begin{tabular}{|l|l|} \hline
Exemplo Entrada & Saída esperada \\ \hline
5               & 5 x 1 = 5 \\
                & 5 x 2 = 10 \\
                & 5 x 3 = 15 \\
                & 5 x 4 = 20 \\
                & 5 x 5 = 25 \\
                & 5 x 6 = 30 \\
                & 5 x 7 = 35 \\
                & 5 x 8 = 40 \\
                & 5 x 9 = 45 \\
                & 5 x 10 = 50 \\ \hline
\end{tabular}
\end{center}
