\section{Exercícios}

\begin{enumerate}

\item Dado um número de três dígitos, construir outro número de quatro
   dígitos com a seguinte regra: a) os três primeiros  dígitos,
   contados da esquerda para a direita, são iguais aos do número dado;
   b) o quarto dígito e' um dígito de controle calculado da seguinte forma:
   primeiro dígito + 3*segundo dígito + 5*terceiro dígito; o dígito de
   controle é igual ao resto da divisão dessa soma por 7.

\item Dado um  número inteiro  de  cinco dígitos  representando um  número
   binário, determinar seu valor equivalente em decimal. Por exemplo, se a
   entrada for 10001, a saída deve ser 17.

\item Considere o programa feito para resolução do cálculo do número neperiano
      (seção~\ref{sec_neperiano}.
     Quantas operações de multiplicação sao executadas no seu programa?
\begin{enumerate}
\item Considerando 20 termos.
\item Considerando N termos.
\end{enumerate}

\item Considere a progressão geométrica 1, 2, 4, 8, 16, 32, ... e um inteiro 
   positivo N. Imprimir os N primeiros termos desta PG e a soma deles.

\item Imprimir os N primeiros termos das sequências definidas pelas relações
   de recorrência:

\begin{enumerate}
\item $Y(k+1) = Y(k) + k, k = 1,2,3..., Y(1)=1$
\item $Y(k+1) = Y(k) + (2k+1), k = 0,1,2,3..., Y(0)=1$
\item $Y(k+1) = Y(k) + (3k^2 + 3k + 1), k = 0,1,2,3..., Y(0)=1$
\item $Y(k+1) = 2Y(k), k = 1,2,3..., Y(1)=1$
\end{enumerate}

\item Dado um número  inteiro $N$,  tabelar $N[k]$  para $k$  variando de  1 até
     $N$.  Considere  que,  por definição,  $X[k]=X(X-1)(X-2)(X-3)...(X-k+1)$,  
     $X$
     sendo um número  real, $k$ um natural diferente de  zero e $X[0]=1$. 
     Observe que se $X=N=k$, então $N[N]=N!$. 

\item Sabe-se  que o valor do coseno  de 1 (um) radiano  pode ser calculado
    pela série infinita abaixo:


\begin{center}
\[
coseno(x) = \frac{1}{0!} - \frac{x^2}{2!} + \frac{x^4}{4!} - \frac{x^6}{6!} 
           + \ldots
\]
\end{center}

    Fazer um  programa que calcule o  valor do cosseno de $x$ 
    obtido  pela série acima considerando somente os primeiros 14  
    termos da mesma.

\item Considere o conjunto $C$ de todos os números inteiros com quatro algarismos
distintos, ordenados segundo seus valores, em ordem crescente:
\begin{displaymath}
C = \{1023, 1024, 1025, 1026, 1027, 1028, 1029, 1032, 1034, 1035, \dots\}
\end{displaymath}

\vspace{-.2cm}
Faça um programa em \emph{Pascal} que leia um número $N$, pertencente a este
conjunto, e imprima a posição deste número no conjunto.

Exemplos:

\vspace{.1cm}
\begin{minipage}{2in}
\begin{itemize}
\item
Entrada: 1026\\
Saída: 4
\vspace{-.2cm}
\item
Entrada: 1034\\
Saída: 9
\end{itemize}
\end{minipage}
\begin{minipage}{2in}
\begin{itemize}
\item
Entrada: 9876\\
Saída: 4536
\vspace{-.2cm}
\item
Entrada: 1243\\
Saída: 72
\end{itemize}
\end{minipage}

\item Faça um programa em \emph{Pascal} que calcule e imprima o valor de $f(x)$, onde $x\in\Re$ é lido no teclado e:
\begin{equation*}
 f(x) = \frac{5x}{2!} - \frac{6x^2}{3!} + \frac{11x^3}{4!} - \frac{12x^4}{5!} + \frac{17x^5}{6!} - \frac{18x^6}{7!} + \ldots
\end{equation*}
O cálculo deve parar quando $abs(f(x_{n+1})-f(x_n)) < 0.00000001$, onde $abs(x)$ é a função em \emph{Pascal} que retorna o valor absoluto de $x$.

\item O número áureo $\varphi$ (1,6180339...) pode ser calculado através de
expressões com séries de frações sucessivas do tipo:
\begin{eqnarray*}
   \varphi_1 &=& 1 + \frac{1}{1}\ =\ 2 \\
   \varphi_2 &=& 1 + \frac{1}{1 + \frac{1}{1}}\ =\ 1,5 \\
   \varphi_3 &=& 1 + \frac{1}{1 + \frac{1}{1 + \frac{1}{1}}}\ =\ 1,666 \\
   \varphi_4 &=& 1 + \frac{1}{1 + \frac{1}
                {1 + \frac{1}{1 + \frac{1}{1}}}}\ =\ 1,6
\end{eqnarray*}
onde $\varphi_i$ indica a aproximação do número áureo com $i$ frações
sucessivas. Estes valores variam em torno do número áureo, sendo maior ou
menor alternadamente, mas sempre se aproximando deste quando o número
de frações cresce.

Faça um programa em \emph{Pascal} que leia um número $N$ e imprima o
valor da aproximação do número áureo $\varphi_N$, que usa uma série de
$N$ frações sucessivas.

\item Dado um inteiro positivo $N$ e dada uma sequência de $N$ números reais $x_1, \ldots, x_n$
faça um programa em \emph{Pascal} que calcule o quociente da soma dos reais pelo seu produto. Isto é:

\[
q=\frac{\sum_{i=1}^{N}{x_i}}{\prod_{i=1}^{N}x_i}
\]

Como não pode haver divisão por zero, seu programa deve parar tão logo esta situação seja
verificada indicando uma mensagem apropriada para o usuário.

\item Em \emph{Pascal} o tipo \textit{CHAR} é enumerável, e portanto está na classe dos tipos
chamados de \textit{ordinais}, conforme o guia de referência da linguagem estudado em
aula. A ordem de cada caracter é dada pela tabela ASCII. Assim é possível, por exemplo, escrever
trechos de códico tais como:

\vspace*{\baselineskip}

\begin{center}
\begin{minipage}{8cm}
\begin{verbatim}
IF 'A' > 'B' THEN
     WRITE ('A eh maior que B')
ELSE
     WRITE ('A não eh maior que B');
\end{verbatim}
\end{minipage}
\end{center}

\vspace*{\baselineskip}

\noindent que  produziria a mensagem ``A não eh maior que B'',
pois na tabela ASCII o símbolo ``A'' tem ordem 64 enquanto que ``B'' tem ordem 65.

Ou ainda:

\vspace*{\baselineskip}

\begin{center}
\begin{minipage}{8cm}
\begin{verbatim}
FOR i:= 'a' TO 'z' DO
     WRITE (i);
\end{verbatim}
\end{minipage}
\end{center}

\vspace*{\baselineskip}

\noindent que produziria como saída ``abcdefghijklmnopqrstuvxwyz''.

Faça um programa em \emph{Pascal} que leia seu nome completo (nomes completos em geral) constituídos
por apenas letras maiúsculas entre ``A'' e ``Z'' e espaços em branco terminadas em ``.'' e que
retorne o número de vogais e consoantes neste nome. Exemplos:

\vspace*{\baselineskip}

\begin{minipage}{8cm}
\begin{verbatim}
Entrada: FABIANO SILVA.
Saída: 
     Vogais: 6
     Consoantes: 6

Entrada: MARCOS ALEXANDRE CASTILHO.
Saída: 
     Vogais: 9
     Consoantes: 14

\end{verbatim}
\end{minipage}

\item Faça um  programa em \emph{Pascal} que leia um inteiro positivo $n$, e escreva
a soma dos $n$ primeiros termos da série:

\[ \frac{1000}{1} - \frac{997}{2} + \frac{994}{3} - \frac{991}{4} + \ldots\]

\item Dizemos que uma sequência de inteiros é {\bf $k$-alternante} se for
composta alternadamente por segmentos de números pares de tamanho $k$ e
segmentos de números ímpares de tamanho $k$.

Exemplos:

\noindent
A sequência 1 3 6 8 9 11 2 4 1 7 6 8 é 2-alternante. \\
A sequência 2 1 4 7 8 9 12 é 1-alternante.  \\
A sequência 1 3 5 é 3-alternante.

Escreva um programa \emph{Pascal} que verifica se uma sequência de tamanho $n$ é
10-alternante.  O programa deve ler $n$, o tamanho da sequência, no inicío do
programa e aceitar somente valores  múltiplos de 10.
A saída do programa deve ser a mensagem  ``{\tt A sequencia eh
10-alternante}'' caso a sequência seja 10-alternante e ``{\tt A sequencia
nao eh 10-alternante}'', caso contrário.


\item Faça um  programa em \emph{Pascal} que calcule
 o resultado da seguinte série:

\[
S = \frac{x^0}{2!} - \frac{x^4}{6!} + \frac{x^8}{10!} - \frac{x^{12}}{14!} + \frac{x^{16}}{18!} - \ldots
\]


\item Faça um programa em \emph{Pascal} que
receba como entrada um dado
inteiro $N$ e o imprima como um produto de primos. Exemplos:
$45 = 3 \times 3 \times 5$. $56 = 2 \times 2 \times 2 \times 7$.

\item Faça um programa em \emph{Pascal} que calcule e escreva o
 valor de $S$ assim definido:

 \[ 
        S = \frac {1}{1!} - \frac {2}{2!} + \frac {4}{3!}  - \frac
        {8}{2!} + \frac {16}{1!} - \frac {32}{2!} + \frac {64}{3!} -
        \cdots 
 \]

\item Fazer  um  programa em  \emph{Pascal}  que calcule e escreva o valor de S:
\[
    S = \frac {37 \times 38}{1} + \frac {36 \times 37}{2} + \frac {35 \times 36}{3} + 
        \cdots + \frac{1 \times 2}{37}
\]


\end{enumerate}
