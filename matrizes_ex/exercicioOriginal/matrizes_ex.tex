\subsection{Exercícios}


\begin{enumerate}

\item Dada uma matriz real A com $m$ linhas e $n$ colunas e um vetor real $V$ 
com $n$ elementos, determinar o produto  de $A$ por $V$. 

\item Um  vetor real $X$ com $n$  elementos é apresentado como  resultado 
de um sistema de  equações lineares $Ax=B$ cujos  coeficientes são representados
em uma  matriz real $A  (m \times n)$  e os lados  direitos das equações  em um
vetor real $B$ de $m$ elementos.  Verificar se o vetor $X$ é realmente solução
do sistema dado. 

\item Dizemos que uma  matriz inteira $A (n \times n)$ é 
uma matriz de permutação
se em cada linha e em cada  coluna houver $n-1$ elementos nulos e um único
elemento igual a 1.  Dada uma matriz inteira $A (n \times  n)$ verificar se 
$A$ é de permutação. Exemplos:

\begin{center}
\begin{tabular}{cccc}
0 & 1 & 0 & 0 \\ 
0 & 0 & 1 & 0 \\
1 & 0 & 0 & 0 \\
0 & 0 & 0 & 1 \\
\end{tabular}
\end{center}

é de permutação, enquanto que esta outra não é:

\begin{center}
\begin{tabular}{cccc}
0 & 1 & 0 & 0 \\
0 & 0 & 1 & 0 \\
1 & 0 & 0 & 0 \\
0 & 0 & 0 & 2 \\
\end{tabular}
\end{center}

\item Dada uma matriz  $A (n \times m)$ 
imprimir o número de  linhas e o número de
colunas nulas da matriz. Exemplo: a matriz abaixo 
tem duas linhas e uma coluna nulas. 

\begin{center}
\begin{tabular}{cccc}
0 & 0 & 0 & 0 \\
1 & 0 & 2 & 2 \\
4 & 0 & 5 & 6 \\
0 & 0 & 0 & 0 \\
\end{tabular}
\end{center}



\item  Dizemos  que uma matriz  quadrada inteira é  um quadrado mágico  se a
soma dos elementos de cada linha,  a soma dos elementos de cada coluna e
a  soma dos  elementos das  diagonais principal  e secundária  são todos
iguais.  Exemplo:

\begin{center}
\begin{tabular}{ccc}
8 &  0 & 7 \\
4 &  5 & 6 \\
3 & 10 & 2 \\
\end{tabular}
\end{center}

é um quadrado mágico pois 
8+0+7 = 4+5+6 = 3+10+2 = 8+4+3 = 0+5+10 = 7+6+2 = 8+5+2 = 3+5+7 = 15.

\begin{enumerate}
\item Dada  uma matriz  quadrada  $A (n \times m)$,  
verificar  se $A$  é um  quadrado mágico. 
\item Informe quantas são e quais são as submatrizes não triviais
      (isto é, não pode ser a matriz constituída por unicamente um elemento, 
      uma linha e uma coluna) que definem quadrados mágicos. 
      Por exemplo, a matriz do exemplo acima
      tem 4 submatrizes de tamanho $2 \times 2$, duas submatrizes de 
      tamanho $2 \times 3$, etc, e uma única submatriz de dimensão
      $3 \times 3$;
\begin{itemize}
\item Armazene de alguma maneira as 
      informações necessárias sobre a localização precisa de cada uma das 
      submatrizes não triviais que definem quadrados mágicos;
\item Imprima a qualquer tempo algum quadrado mágico armazenado;
\item Dado uma dimensão qualquer, digamos $N$, imprima todas os 
      quadrados mágicos de dimensão $N$ contidos na matriz original.
\end{itemize}
\end{enumerate}


\item Implemente o quadrado ``quase magico''.  
Um quadrado quase magico é aquele em que as somas das linhas e a somas das
colunas resultam em um mesmo valor, mas a soma dos elementos das 
diagonais não.  O programa deve pedir a dimensão do
quadrado a ser impresso, que deve ser um número ímpar entre 1 e 99.



\item Um jogo  de palavras cruzadas pode ser representado  por uma matriz $A
(n \times  m)$ onde cada posição da  matriz corresonde a um  quadrado do jogo,
sendo  que 0  indica  um quadrado  em  branco e  -1  indica um  quadrado
preto.  Colocar as  numerações de  início de  palavras  horizontais e/ou
verticais  nos   quadrados  correspondentes  (substituindo   os  zeros),
considerando que uma palavra deve ter pelo menos duas letras. 

Exemplo: Dada a matriz:
\begin{center}
\begin{tabular}{cccccccc}
 0 & -1 &  0 & -1 & -1 &  0 & -1 &  0 \\
 0 &  0 &  0 &  0 & -1 &  0 &  0 &  0 \\
 0 &  0 & -1 & -1 &  0 &  0 & -1 &  0\\
-1 &  0 &  0 &  0 &  0 & -1 &  0 &  0\\
 0 &  0 & -1 &  0 &  0 &  0 & -1 & -1\\
\end{tabular}
\end{center}

A saída deveria ser:
\begin{center}
\begin{tabular}{cccccccc}
 1 & -1 &  2 & -1 & -1 &  3 & -1 &  4 \\
 5 &  6 &  0 &  0 & -1 &  7 &  0 &  0 \\
 8 &  0 & -1 & -1 &  9 &  0 & -1 &  0\\
-1 & 10 &  0 & 11 &  0 & -1 & 12 &  0\\
13 &  0 & -1 & 14 &  0 &  0 & -1 & -1\\
\end{tabular}
\end{center}

\item Uma matriz $D (8 \times 8)$ 
pode representar a posição atual de um jogo de damas, 
sendo que 0 indica uma casa vazia, 1 indica uma casa ocupada por uma peça branca
e -1 indica uma casa ocupada por uma peça preta. Supondo que as peças pretas
estão se movendo no sentido crescente das linhas da matriz $D$, determinar
as posições das peças pretas que:

\begin{itemize}
\item podem tomar peças brancas;
\item podem mover-se sem tomar peças brancas;
\item não podem se mover.
\end{itemize}

\item Deseja-se atualizar as contas correntes dos clientes de uma agência
bancária. É dado o cadastro de $N$ clientes contendo para cada cliente
o número de sua conta e seu saldo. O cadastro está ordenado pelo número 
da conta. Em seguida é dado o número de operações realizadas no dia, e, 
para cada operação, o número da conta, uma letra $C$ ou $D$ indicando se a
operação é de crédito ou débido, e o valor da operação. Emitir o cadastro
de clientes atualizado. Pode ser modelado como uma matriz $N \times 2$.


\item Reordenar a matriz do exercício anterior 
por ordem de saldo, do maior para o menor. 


\item Os elementos $M[i,j]$ de uma matriz $M (n \times n)$ 
representam os custos de
transporte da cidade $i$ para a cidade $j$. Dados $n$ itinerários lidos
do teclado, cada um
com $k$ cidades, calcular o custo total para cada itinerário. Exemplo:

\begin{verbatim}
4 1 2 3
5 2 1 400
2 1 3 8
7 1 2 5
\end{verbatim}

O custo do itinerário 1 4 2 4 4 3 2 1 é: 
M[1,4] + M[4,2] + M[2,4] + M[4,4] + M[4,3] + M[3,2] + M[2,1] =
3 + 1 + 400 + 5 + 2 + 1 + 5 = 417.


\item Considere $n$ cidades numeradas de 1 a $n$ que estão interligadas por
uma série de estradas de mão única. As ligações entre as cidades são 
representadas pelos elementos de uma matriz quadrada $L (n \times n)$ cujos
elementos L$[i,j]$ assumem o valor 0 ou 1 conforme exista ou não estrada
direta que saia da cidade $i$ e chegue na cidade $j$. Assim, os elementos
da $i$-ésima linha indicam as estradas que saem da cidade $i$ e os 
elementos da $j$-ésima coluna indicam as estradas que chegam à cidade $j$.
Por convenção, $L[i,i]=1$. A figura abaixo ilustra um exemplo para $n=4$.

\begin{center}
\begin{tabular}{ccccc}
  & A & B & C & D \\
A & 1 & 1 & 1 & 0 \\
B & 0 & 1 & 1 & 0 \\
C & 1 & 0 & 1 & 1 \\
D & 0 & 0 & 1 & 1 \\
\end{tabular}
\end{center}

Por exemplo, existe um caminho direto de $A$ para $B$ mas não de $A$ para $D$.

\begin{enumerate}
\item Dado $k$, determinar quantas estradas saem e quantas chegam à cidade $k$.
\item  A qual das cidades chega o maior número de estradas?
\item Dado $k$, verificar se todas as ligações diretas entre a cidade $k$ e 
   outras são de mão dupla;
\item  Relacionar as cidades que possuem saídas diretas para a cidade $k$;
\item Relacionar, se existirem:
  \begin{itemize}
   \item As cidades isoladas, isto é, as que não têm ligação com nenhuma outra;
   \item As cidades das quais não há saída, apesar de haver entrada;
   \item As cidades das quais há saída sem haver entrada;
  \end{itemize}
\item Dada uma sequência de m inteiros cujos valores estão entre 1 e $n$,
   verificar se é possível realizar o roteiro correspondente. No 
   exemplo dado, o roteiro representado pela sequência ($m=5$) 3 4 3 2 1
   é impossível;
\item Dados $k$ e $p$, determinar se é possível ir da cidade $k$ até a cidade 
   $p$
   pelas estradas existentes. Você consegue encontrar o menor caminho
   entre as duas cidades?
\item Dado $k$, determinar se é possível, partindo de $k$, passar por todas as
   outras cidades uma única vez e retornar a $k$.
\end{enumerate}

\item Uma matriz transposta $M^T$ é o resultado da troca de linhas por colunas em uma determinad
a matriz $M$.
Escreva um programa que leia duas matrizes ($A$ e $B$), e testa se $B = A +
 A^T$.

\item Fazer procedimentos que recebam três parâmetros:  uma matriz e dois
   inteiros representando as dimensões da matriz. Retornar: 
  \begin{enumerate} 
    \item a transposta de matriz; 
    \item a soma das duas matrizes;
    \item a multiplicação das duas matrizes.
  \end{enumerate} 

 \item Faça um programa que leia duas matrizes $A$
 e $B$ quaisquer e imprima a transposta de  $B$ se a transposta de $A$
 for igual a $B$.



\item Fazer uma função que receba  como parâmetros:  dois  vetores de
   reais    e   dois   inteiros    representando   as    dimensões   dos
   vetores. Retornar o produto escalar dos dois vetores (real). 
   Refazer a multiplicação de matrizes usando esta função. 


\item Faça um programa que, dadas $N$ datas em uma
matriz DATAS$_{N\times3}$, onde a primeira coluna corresponde ao dia, a segunda ao
mês e a terceira ao ano, coloque essas datas em ordem cronológica
crescente. Por exemplo:

\[
DATAS=\left( \begin{array}{ccc}
5&1&1996\\
25&6&1965\\
16&3&1951\\
15&1&1996\\
5&11&1965\\
\end{array}
\right)
DATAS=\left( \begin{array}{ccc}
16&3&1951\\
25&6&1965\\
5&11&1965\\
5&1&1996\\
15&1&1996\\
\end{array}
\right)
\]


\item Verifique se a matriz $A$ \'e sim\'etrica, isto \'e, se $A[i,j]=A[j,i], \forall i, j \le M$.
Fa\c ca uma fun\c c\~ao que retorne 1 em caso afirmativo, 0 caso contr\'ario.

\item Uma matriz $B$ é dita inversa da matriz $A$ quando $A \times B = I$, 
onde $I$ é a matriz identidade e $\times$ é a operação de multiplicação
de matrizes. A matriz identidade é a matriz quadrada onde os elementos da 
diagonal principal são 1 e os demais 0 ($I[i,j] = 1$ se $i=j$ e 
$I[i,j] = 0$ se $i \neq j$).  Escreva um programa em \emph{Pascal} que 
leia duas matrizes e testa se a segunda é a inversa da primeira.





\item Considere uma matriz $M$ de tamanho $N \times M$
utilizada para representar gotas de água (caractere $G$) em uma
janela. A cada unidade de tempo $T$, as gotas descem uma posição na
matriz, até que atinjam a base da janela e desapareçam. Considere que
a chuva parou no momento em que seu programa iniciou.

Exemplo:
\begin{small}
\begin{verbatim}
         Passo T=0           Passo T=1               Passo T=4
     -----------------   -----------------       ----------------- 
     |   G        G  |   |               |       |               |
     |       G       |   |   G        G  |       |               |
     |               |   |       G       |       |               |
     |     G   G     |   |               |  ...  |               |  ...
     |               |   |     G   G     |       |   G        G  |
     |               |   |               |       |       G       |
     |               |   |               |       |               |
     +++++++++++++++++   +++++++++++++++++       +++++++++++++++++
\end{verbatim}
\end{small}

Faça um programa em que:

\begin {enumerate}

\item 
Leia as coordenadas iniciais das gotas de água na matriz. O canto
superior esquerdo da matriz (desconsiderando as bordas) possui
coordenada $(1,1)$. A coordenada $(0,0)$ indica o término da
leitura. Coordenadas inválidas devem ser desconsideradas.

Exemplo de entrada para a matriz acima (em $T=0$):
\begin{small}
\begin{verbatim}
    1 4
    1 13
    4 6
    2 8
    100 98
    4 10
    0 0
\end{verbatim}
\end{small}
Note que a entrada $(100,98)$ deve ser descartada pois é inválida para
a matriz do exemplo.

\item 
Imprima, a cada unidade de tempo $T$, o conteúdo da matriz $M$,
atualizando a posição das gotas $G$ até que não reste nenhuma gota na
janela.
\end{enumerate}


\item Modifique seu programa da questão anterior de modo que as gotas que estão
inicialmente na primeira linha da janela desçam com o dobro da
velocidade das outras gotas. Ou seja, as gotas que iniciam na primeira
linha descem duas linhas na matriz a cada instante $T$. As gotas mais
rápidas podem encontrar gotas mais lentas pelo caminho, neste caso a
gota mais lenta desaparece ficando somente a mais rápida.

\item Modifique novamente o programa da questão anterior considerando que,
desta vez, a 
cada unidade de tempo $T$, $NG$ novas gotas são inseridas na matriz. Além
disso, as gotas descem na matriz até que atinjam a base da janela e desapareçam.
Inicialmente não há gotas na janela, pois a chuva começa quando $T=1$.

\vspace*{0.3cm}
Exemplo:
\begin{small}
\begin{verbatim}
         Passo T=1           Passo T=2             Passo T=1000
     -----------------   -----------------       ----------------- 
     |   G        G  |   |        G      |       |               |
     |       G       |   |   G        G  |       |  G            |
     |               |   |       G       |       |               |
     |     G   G     |   |               |  ...  |         G     |
     |               |   |     G   G     |       |   G        G  |
     |               |   |               |       |       G       |
     |               |   |  G            |       |               |
     +++++++++++++++++   +++++++++++++++++       +++++++++++++++++
\end{verbatim}
\end{small}

Faça um programa em  que:

\begin {enumerate}
\item 
Leia o número de linhas ($L$) e o número de colunas ($C$) da
matriz $M$, a quantidade de novas gotas a serem criadas a cada
iteração ($NG$), e o número de iterações ($TMAX$) do programa.

Exemplo de entrada para a matriz acima:
\begin{small}
\begin{verbatim}
    7 15 5 1000
\end{verbatim}
\end{small}

\item 
A cada unidade de tempo $T$, insira $NG$ novas gotas na matriz. A posição de 
uma nova gota é dada por um procedimento cujo protótipo é:
\begin{verbatim}
    Procedure coordenada_nova_gota(L,C:integer; VAR x,y:integer);
\end{verbatim}
Este procedimento 
recebe quatro parâmetros: os dois primeiros indicam o número de linhas e 
colunas da matriz $M$ ($L,C$). Os dois últimos retornam as coordenadas 
($x,y$) da nova gota na matriz.

\item
A cada unidade de tempo $T$, imprima o conteúdo da matriz $M$,
atualizando a posição das gotas $G$ seguindo os seguintes critérios:

        \begin{enumerate}
        \item
        Quando uma gota cai sobre outra, forme-se uma gota ``dupla'', ou seja, ela desce
        duas posições a cada instante $T$. Caso uma nova gota caia sobre uma gota
        ``dupla'', surge uma gota ``tripla'', que desce três posições a cada instante
        $T$, e assim por diante. 
        
        \item
        As gotas mais rápidas podem encontrar gotas mais lentas pelo caminho, neste caso
        a velocidade delas é somada.
        \end{enumerate}

\end{enumerate}



\item Considere o tipo PGM para imagens como definido na seção~\ref{pgm}.
Faça um programa que leia da entrada padrão 
(teclado) duas imagens no formato PGM: imagem original ($imgO$) e a imagem 
do padrão ($imgP$).
 
Em seguida, o programa deve procurar se a imagem $imgP$ está contida na 
imagem $imgO$ e imprimir na tela as coordenadas $(coluna,linha)$ do canto 
superior esquerdo de \textbf{cada ocorrência} da imagem $imgP$ encontrada na 
imagem $imgO$.

Observações:
\begin{itemize}
    \item A imagem $imgP$ pode aparecer mais de uma vez na imagem $imgO$;
    \item Na imagem $imgP$, pontos com o valor $-1$ devem ser ignorados, isto é, represent
am pontos transparentes da imagem e não devem ser comparados com a imagem $imgO$.
    \item Estruture seu código. A solução parcial ou a indicação de chamadas a funções não
 implementadas serão consideradas.
\end{itemize}

Exemplo:
\begin{itemize}
\vspace*{-2mm}      
    \item \textbf{Imagem original}:
    \begin{verbatim}
    P2
    11 10
    40
    40 5  5  5  5  5  5  5  5  40 0 
    5  20 20 5  5  5  5  5  5  5  5
    5  5  20 5  5  5  0  0  0  0  0
    5  5  20 20 5  5  20 20 0  0  5
    5  5  5  5  5  5  0  20 0  0  0
    5  5  5  5  5  5  0  20 20 0  5
    5  5  5  5  11 11 11 0  0  0  0
    5  5  5  5  20 20 11 5  5  5  5
    5  5  5  5  11 20 11 5  5  5  0
    40 5  5  5  11 20 20 5  5  40 5
    \end{verbatim}

\vspace*{-4mm}      
    \item \textbf{Imagem do padrão}:
    \begin{verbatim}
    P2
    3 3
    20
    20 20 -1
    -1 20 -1
    -1 20 20
    \end{verbatim}
    
\vspace*{-4mm}      
    \item \textbf{Resultado do Programa}:
    \begin{verbatim}
    2 2
    7 4
    5 8   
    \end{verbatim}    
\end{itemize}

\item Modifique o programa anterior de forma que, ao invés de imprimir as 
coordenadas, seja impressa uma nova imagem, que consiste de uma cópia da 
imagem original $imgO$ na qual as ocorrências da imagem $imgP$ estejam 
circunscritas por uma borda de um ponto de largura, com o valor máximo da 
imagem $imgO$ (3ª linha do arquivo PGM). Você não precisa se preocupar com 
possíveis sobreposições das bordas.

Exemplo da nova saída para a entrada original:
\begin{itemize}
\vspace*{-2mm}      
    \item \textbf{Imagem resultante}:
    \begin{verbatim}
    P2
    11 10
    40
    40 40 40 40 40 5  5  5  5  40 0
    40 20 20 5  40 5  5  5  5  5  5
    40 5  20 5  40 40 40 40 40 40 0
    40 5  20 20 40 40 20 20 0  40 5
    40 40 40 40 40 40 0  20 0  40 0
    5  5  5  5  5  40 0  20 20 40 5
    5  5  5  40 40 40 40 40 40 40 0
    5  5  5  40 20 20 11 40 5  5  5
    5  5  5  40 11 20 11 40 5  5  0
    40 5  5  40 11 20 20 40 5  40 5
    \end{verbatim}
\end{itemize}

\item Uma matriz é chamada de \textit{esparsa} quando 
possui uma grande quantidade de elementos que valem zero. Por exemplo,
a matriz de ordem $5 \times 4$ seguinte é esparsa, pois contém somente
4 elementos não nulos.

\begin{center}
\begin{tabular}{c|c|c|c|c|} 
\multicolumn{1}{c}{} & \multicolumn{1}{c}{1} & \multicolumn{1}{c}{2} & \multicolumn{1}{c}{
3} & \multicolumn{1}{c}{4} \\ \cline{2-5}
1 & 0  & 17 & 0   & 0 \\ \cline{2-5}
2 & 0  & 0  & 0   & 0 \\ \cline{2-5}
3 & 13 & 0  & -12 & 0 \\ \cline{2-5}
4 & 0  & 0  & 25  & 0 \\ \cline{2-5}
5 & 0  & 0  & 0   & 0 \\ \cline{2-5}
\end{tabular}
\end{center}

Obviamente, a representação computacional padrão para matrizes é
ineficiente em termos de memória, pois gasta-se um espaço inútil para
se representar muitos elementos nulos. 

Nesta questão, vamos usar uma representação alternativa que vai permitir
uma boa economia de memória. 

A proposta é representar somente os elementos não nulos. Para isto usaremos
três vetores, dois deles ($L$ e $C$) 
para guardar as coordenadas dos elementos não nulos
e o terceiro (D) para guardar os valores dos elementos daquelas coordenadas. 
Também
usaremos três variáveis para representar o número de linhas e colunas
da matriz completa e o número de elementos não nulos da matriz.

Considere as seguintes definições de tipos:
\begin{lstlisting}
CONST
     MAX = 6;      (* um valor bem menor que 5 x 4, dimensao da matriz *) 
TYPE 
     vetor_coordenadas = array [1..MAX] of integer;  (* coordenadas    *)
     vetor_elementos   = array [1..MAX] of real;     (* dados          *)
VAR
     L, C: vetor_coordenadas; (* L: linhas, C: colunas                 *)
     D: vetor_elementos;      (* D: dados                              *)
     N_lin, N_col: integer;   (* para armazenar as dimensoes da matriz *)
     N_elementos: integer     (* numero de elementos nao nulos         *)
\end{lstlisting}


\newtheorem{definicao}{Definição}

\begin{definicao}
\label{def1}
Um elemento M[i,j] da matriz completa pode ser 
obtido da representação compactada: 

\begin{itemize}
\item se existe um k tal que L[k] = i e C[k] = j, então M[i,j] = D[k];
\item caso contrário, M[i,j] = 0.
\end{itemize} 
\end{definicao}

A matriz do exemplo anterior pode então ser assim representada:

\begin{verbatim}
N_elementos:= 4; N_lin:= 5; N_col:= 4;
\end{verbatim}

\begin{center}
\begin{tabular}{c|p{.6cm}|p{.6cm}|p{.6cm}|p{.6cm}|p{.6cm}|p{.6cm}|} 
\multicolumn{1}{c}{} & \multicolumn{1}{c}{1} & \multicolumn{1}{c}{2} & \multicolumn{1}{c}{
3} & \multicolumn{1}{c}{4} & \multicolumn{1}{c}{5} & \multicolumn{1}{c}{6}\\ \cline{2-7}
L & 1  & 3  & 3   & 4 & & \\ \cline{2-7}
\end{tabular}


\begin{tabular}{c|p{.6cm}|p{.6cm}|p{.6cm}|p{.6cm}|p{.6cm}|p{.6cm}|} 
\multicolumn{1}{c}{} & \multicolumn{1}{c}{} & \multicolumn{1}{c}{} & \multicolumn{1}{c}{} 
& \multicolumn{1}{c}{} & \multicolumn{1}{c}{} & \multicolumn{1}{c}{}\\ \cline{2-7}
C & 2  & 1  & 3   & 3 & & \\ \cline{2-7}
\end{tabular}

\begin{tabular}{c|p{.6cm}|p{.6cm}|p{.6cm}|p{.6cm}|p{.6cm}|p{.6cm}|} 
\multicolumn{1}{c}{} & \multicolumn{1}{c}{} & \multicolumn{1}{c}{} & \multicolumn{1}{c}{} 
& \multicolumn{1}{c}{} & \multicolumn{1}{c}{} & \multicolumn{1}{c}{}\\ \cline{2-7}
D & 17  & 13  & -12   & 25 & & \\ \cline{2-7}
\end{tabular}
\end{center}

\begin {enumerate}


\item Fazer um procedimento que leia da entrada padrão: 
\begin{itemize}
\item dois inteiros, representando as dimensões da matriz (linha, coluna);
\item trincas de elementos l, c, d, onde l e c são inteiros e d é real,
representando respectivamente a linha, a coluna
e o valor de um elemento não nulo da matriz. A leitura termina quando for
lido uma trinca 0, 0, 0. Para cada trinca, devem ser criados os três
vetores que representam a matriz conforme descrito acima. Veja o exemplo
de entrada de dados, abaixo.
\end{itemize}

Exemplo para a entrada de dados:

\begin{verbatim}
5 4
1 2 17
3 1 13
3 3 -12
4 3 25
0 0 0 
\end{verbatim}


\item Fazer uma função que, dada uma coordenada (l, c), respectivamente
para uma linha e coluna,
retorne o valor de elemento M[l,c], conforme a definição \ref{def1}.

\item Fazer um procedimento que,  dadas duas matrizes no formato compactado
descrito acima, obtenha uma terceira matriz compactada que é a soma
das duas primeiras. 

\item Fazer um procedimento que, dada uma matriz no formato compactado,
imprima na tela uma matriz no formato padrão, contendo os zeros.

%\item Fazer uma função que testa se uma entrada dada conforme o item 1
%acima corresponde de fato a uma matriz esparsa. Considere que a matriz
%é esparsa se no máximo 15\% dos elementos forem não nulos.
\end{enumerate}




\item Declare uma matriz $M \times N$ de caracteres do
  tipo \textsf{char}.  Implemente quatro funções que, dados como
  parâmetros a matriz, uma palavra do tipo \textsf{string} e um par de
  coordenadas na matriz, isto é, dois inteiros representando uma linha e
  uma coluna,  descubram se,  na matriz, a  palavra ocorre  iniciando na
  posição indicada pelas coordenadas. 
A primeira função procura na horizontal, da esquerda para direita;
a segunda função procura na horizontal, da direita para esquerda;
a terceira função procura na vertical, da cima para baixo;
a quarta função procura na vertical, da baixo para cima.




\item Os incas construiam pirâmides de base quadrada em
que a única forma de se atingir o topo era seguir em espiral pela
borda, que acabava formando uma escada em espiral.  Escreva um
programa que leia do teclado uma matriz quadrada $N \times N$ de
números inteiros e verifica se a matriz é inca; ou seja, se partindo
do canto superior esquerdo da matriz, no sentido horário, em espiral,
a posição seguinte na ordem é o inteiro consecutivo da posição
anterior.  Por exemplo, as matrizes abaixo são incas:

\begin{verbatim}
   1   2   3   4        1   2   3   4   5
  12  13  14   5       16  17  18  19   6        
  11  16  15   6       15  24  25  20   7
  10   9   8   7       14  23  22  21   8
                       13  12  11  10   9
\end{verbatim}

O programa deve ler do teclado a dimensão da matriz 
(um inteiro $N$, $1 \leq N \leq 100$) e em cada uma das próximas $N$ linhas, 
os inteiros correspondentes às  entradas da matriz naquela linha.  
A saída do programa deve ser
``A matriz eh inca'' ou ``A matriz nao eh inca''.

\item Escreva um programa em \emph{Pascal} que leia do teclado uma matriz A 
($N \times M$)
de inteiros e imprima uma segunda matriz $B$ de mesma dimensões em que
cada elemento $B[i,j]$ seja constituído pela soma de todos os 8 elementos
vizinhos do elemento $A[i,j]$, excetuando-se o próprio $A[i,j]$.


\item Nesta questão você terá que providenciar ligações par-a-par 
entre diversos pontos distribuídos ao longo de uma rota qualquer.
A entrada de dados consiste de um conjunto de pares $(x,y), 1\leq x,y \leq
MAX$, sendo
que o último par a ser lido é o (0,0), que não deve ser processado.

Para cada par $(x,y)$ dado como entrada, você deve providenciar uma
conexão física entre eles. 
As linhas de uma matriz podem representar a ``altura'' das
linhas de conexão, enquanto que as colunas da matriz podem representar
os pontos $(x,y)$ sendo conectados. Um símbolo de ``\verb#+#'' pode ser 
usado para se representar alteração na direção de uma conexão. 
O símbolo ``\verb#|#'' pode ser usado para representar um trecho de 
conexão na vertical. Finalmente o símbolo ``\verb#-#'' pode ser usado
para se representar um trecho de conexão na direção horizontal.
Quando um cruzamento de linhas for inevitável, deve-se usar o
símbolo ``\verb#x#'' para representá-lo. Considere que não existem trechos 
de conexões na diagonal.

Por exemplo, suponha que a entrada é dada pelos seguintes pares:

\begin{verbatim}
3 5
2 9
0 0
\end{verbatim}

Uma possível saída para seu programa seria a impressão da seguinte
matriz:

\begin{verbatim}
4
3         
2  +-------------+
1  | +---+       |
 1 2 3 4 5 6 7 8 9
\end{verbatim}

Outra possível matriz solução para este problema seria esta:
\begin{verbatim}
4
3         
2    +---+  
1  +-x---x-------+
 1 2 3 4 5 6 7 8 9
\end{verbatim}

Note que nesta última versão foi preciso inserir dois cruzamentos.

Ainda como exemplo, se o par (6,8) também fosse dado como entrada no 
exemplo anterior, a saída do programa poderia ser assim exibida:

\begin{verbatim}
4
3          +---+ 
2  +-------x---x-+
1  | +---+ |   | |
 1 2 3 4 5 6 7 8 9
\end{verbatim}

Você deve implementar um programa em \emph{Pascal} que seja
capaz de ler uma sequência de pares terminada em $(0,0)$ (como no
exemplo acima) e que imprima o desenho das conexões como saída,
também conforme o diagrama acima. 


\item 
Modifique o programa anterior com o objetivo de minizar o número de cruzamentos
da matriz gerada como solução do problema anterior. Assim,
a matriz ideal para ser dada como resposta do último exemplo
seria a seguinte:

\begin{verbatim}
4
3           
2  +-------------+
1  | +---+ +---+ |
 1 2 3 4 5 6 7 8 9
\end{verbatim}

\item Considere o seguinte programa:

\begin{lstlisting}
program prova_final;

CONST MAX=100;
TYPE matriz = array [1..MAX,1..MAX] of integer;
VAR n_lin, n_col: integer; (* dimensoes da matriz *)
    m: matriz;             (* matriz *)

(* espaco reservado para os procedimentos *)

begin
   read (n_lin, n_col);
   le_matriz (m, n_lin, n_col);
   acha_maior_sequencia (m, n_lin, n_col, l_ini, c_ini, l_fim, c_fim);
   writeln ('A maior sequencia de numeros repetidos na matriz ');
   writeln ('inicia na coordenada ', l_ini, c_ini);
   writeln (' e termina na coordenada ', l_fim, c_fim);
end.
\end{lstlisting}


Implemente os procedimentos indicados para 
que o programa leia uma matriz de inteiros e imprima as coordenadas
de início e término da maior sequência de números repetidos da matriz.
Esta sequência pode estar tanto nas linhas quanto nas colunas. No caso
de existir mais de uma sequência repetida de mesmo tamanho, você pode
imprimir as coordenadas de qualquer uma delas, desde que imprima as de uma só.

\begin{minipage}{7cm}
\begin{verbatim}
Exemplo 1:
Entrada:      Saída
4 3           1 2 
1 2 3         3 2
2 2 1
3 2 5
\end{verbatim}
\end{minipage} \  
\begin{minipage}{7cm}
\begin{verbatim}
Exemplo 2:
Entrada:      Saída
4 5           2 2
1 2 3 1 2     2 4
1 2 2 2 3
2 3 4 5 6
8 7 6 4 2
\end{verbatim}
\end{minipage}



\item Faça um programa para:
\begin{itemize}
    \item ler uma sequência de polinômios 
    $P_i(x)=a_{i_0} + a_{i_1}x + a_{i_2}x^2 + ... + a_{i_n}x^n, i=1,2,...,k$; 

    \item A leitura deve considerar que cada linha de entrada    
    contém um polinômio $P_i$. A primeira informação é o seu 
    respectivo grau ($n_i$). As outras informações são os
    $n_i$ coeficientes ($a_{i_0}, a_{i_1}, ..., a_{i_n}$);
    
    Exemplo:
    
    $P(x) = 8.1 -3.4x + x^2  \Longrightarrow $ \verb|2  8.1  -3.4  1.0|
    
    \item A sequência de polinômios se encerra quando for fornecido um 
    polinômio de grau zero; 
    
    \item Após a leitura de todos os polinômios, o programa deve ler uma 
    sequência de números reais $x$. Para cada número real lido, o programa 
    deve imprimir o resultado de $P_i(x)$, para todos os polinômios lidos 
    anteriormente ($i=1,2,...,k$);
    
    \item A sequência de números reais se encerra quando for lido o número 
    $0.0$, para o qual não se deve calcular os valores de $P_i(x)$.
\end{itemize}    
    
Exemplo:
\begin{verbatim}
Entrada:
2  -1.0 0.0 1.0
3  1.0  2.0  0.0  -1.0
0
4.5
1.0
0.0

Saída:
P_1(2.0) = 3.0
P_2(2.0) = -3.0
P_1(1.0) = 0.0
P_2(1.0) = 2.0
\end{verbatim}


\item Faça um programa para:

\begin{itemize}
\item ler um inteiro $N$ e uma matriz quadrada de ordem $N$ contendo apenas
      0's e 1's.
\item encontrar a maior submatriz quadrada da matriz de entrada
      que contém apenas 1's.
\item imprimir as coordenadas dos cantos superior esquerdo e
      inferior direito da submatriz encontrada no item anterior. Havendo mais 
      de uma submatriz máxima, imprimir as coordenadas de qualquer uma delas.
\end{itemize}


\noindent Exemplo: Considere a seguinte matriz quadrada de 
ordem 6:

\vspace*{\baselineskip}

\begin{center}
\begin{tabular}{c|c|c|c|c|c|c|}
\multicolumn{1}{c}{}  & \multicolumn{1}{c}{1} & \multicolumn{1}{c}{2} & \multicolumn{1}{c}{3} & \multicolumn{1}{c}{4} & \multicolumn{1}{c}{5} & \multicolumn{1}{c}{6} \\ \cline{2-7}
1 & 0 & 1 & 0 & 1 & 1 & 1 \\ \cline{2-7}
2 & 0 & 1 & 1 & 1 & 1 & 0 \\ \cline{2-7}
3 & 0 & 1 & 1 & 1 & 0 & 1 \\ \cline{2-7}
4 & 1 & 1 & 1 & 1 & 0 & 1 \\ \cline{2-7}
5 & 0 & 0 & 1 & 0 & 1 & 0 \\ \cline{2-7}
6 & 0 & 1 & 0 & 1 & 0 & 1 \\ \cline{2-7}
\end{tabular}
\end{center}

\vspace*{\baselineskip}

A título de ilustração, esta matriz tem:

\begin{itemize}
\item 22 submatrizes quadradas de ordem 1 que contém apenas 1's;
\item 5 submatrizes quadradas de ordem 2 que contém apenas 1's. Por exemplo, 
      para duas delas: uma é dada pelas coordenadas (1,4) e (2,5) e outra 
      pelas coordenadas (2,2) e (3,3);
\item 1 submatriz quadrada de ordem 3 que contém apenas 1's, as coordenadas são (2,2) e (4,4).
\end{itemize}

Como a maior submatriz quadrada que contém apenas 1's é a de ordem 3,
então a saída do programa deve imprimir, para este exemplo, as coordenadas
(2,2) e (4,4).



\item Escreva um programa que, dado um tabuleiro e uma lista de sub-partes 
retangulares do tabuleiro, retorna o número de posições que não pertencem 
a nenhuma sub-parte. Quando uma posição não pertence a nenhuma sub-parte
dizemos que ela está \emph{perdida}.

\vspace*{\baselineskip}

\begin{center}
\textbf{Entrada}
\end{center}


A entrada consiste de uma série de conjuntos de teste. 

Um conjunto de teste começa com uma linha com três números $W$, $H$ e $N$, 
indicando, respectivamente, a largura e a altura do tabuleiro 
e o número de sub-partes deste. 
%
Estes valores satisfazem as seguintes restrições: 
$1 \leq W$, $H \leq 500$ e $0 \leq N \leq 99$. 

Seguem $N$ linhas, compostas de quatro inteiros 
$X_1$, $Y_1$, $X_2$ e $Y_2$, 
tais que $(X_1, Y_1)$ e $(X_2, Y_2)$ 
são as posições de dois cantos opostos de uma sub-parte. 
%
Estes valores satisfazem as seguintes restrições: 
$1 \leq X_1$, $X_2 \leq W$ e $1 \leq Y_1$, $Y_2 \leq H$. 

O fim da entrada acontece quando $W=H=N=0$. 
Esta última entrada não deve ser considerada como um conjunto de teste.

\begin{center}
\textbf{Saída}
\end{center}

O programa deve imprimir um resultado por linha, seguindo o formato 
descrito no exemplo de saída.

\begin{verbatim}
Exemplo

Entrada:
1 1 1
1 1 1 1                {fim do primeiro conjunto de testes}
2 2 2
1 1 1 2
1 1 2 1                {fim do segundo conjunto de testes }
493 182 3
349 148 363 146
241 123 443 147
303 124 293 17         {fim do terceiro conjunto de testes}
0 0 0                  {fim do conjunto de testes}

Saída
Não há posições perdidas. 
Existe uma posição perdida.
Existem 83470 posições perdidas.
\end{verbatim}


\end{enumerate}
