\section{Exercícios}


\begin{enumerate}

\item Fazer uma função em \emph{Pascal} que receba como parâmetro 
   dois números inteiros não nulos e retorne TRUE se um for o contrário 
   do outro e FALSE em caso contrário. Isto é, se os parâmetros forem
   123 (cento e vinte e três) e 321 (trezentos e vinte e um), deve-se 
   retornar TRUE. Usar apenas operações sobre inteiros. 

\item Fazer uma  função  em \emph{Pascal}  
   que  receba como  parâmetro um  número
   inteiro  representando   um  número  binário  e   retorne  seu  valor
   equivalente em decimal. Por exemplo, se a entrada for 10001, a saída 
   deve ser 17. 

\item Fazer uma  função  em  \emph{Pascal}
   que  receba como  parâmetro um  número
   inteiro  e   retorne  TRUE  se  ele   for  primo  e   FALSE  em  caso
   contrário.  Use esta  função para  imprimir todos  os  números primos
   entre 0 e 1000. 

\item Implemente funções para seno e cosseno conforme definidos em 
   capítulos anteriores e use-as em uma terceira função que calcule a
   tangente. O programa principal deve imprimir os valores de $tg(x)$ 
   para um certo valor fornecido pelo usuário.

\item Faça uma função em \emph{Pascal} 
que some dois números representando horas. 
                  A entrada deve ser feita da seguinte maneira: \\
                        12 52 \\
                        7 13 \\
                  A sa\'{\i}da deve ser assim: \\
                  12:52 + 7:13 = 20:05
 
\item Faça uma função que receba como parâmetros seis variáveis 
DIA1, MES1 e ANO1, DIA2, MES2 e ANO2, todas do tipo integer. 
Considerando que cada trinca de dia, 
                  mês e ano representa uma data, a função deve retornar 
\textsf{true} se a primeira 
                  data for anterior à segunda e \textsf{false} 
caso contrário.



\end{enumerate}
